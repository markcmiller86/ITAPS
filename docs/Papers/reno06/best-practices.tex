
\section{Programming Using the TSTT Mesh Interface\label{sec:Programming}}

Early experience with the TSTT interface shows that there are significant
performance penalties created by certain programming practices, including
some practices that naturally arise when translating an existing library-specific
code to use TSTT instead. This section highlights recommended best
practices that come from our early experiences with the interface; more
work in this area is ongoing.

\begin{description}
\item [Use~direct~array~access.]Many TSTT calls return data in SIDL
arrays. Because TSTT uses only one-dimensional SIDL arrays, access to
data in these arrays is simpler and more efficient when using direct
array access rather than calls to the SIDL \texttt{get()} and
\texttt{set()} functions. To use direct array access, the address of the
actual array within the SIDL array structure must be exposed so that it
can be treated directly as a simple array by the application code.
\item [Retrieve~data~in~chunks.]Whenever possible, retrieve data using
the TSTT array access calls instead of entity by entity calls. Experiments
indicate that chunks containing as few as 20 entities were sufficient
to reduce interoperability overhead to less than 5\%, especially when
combined with direct array access~\cite{mcinnes05}.
\item [Re-use~SIDL~arrays.]Especially in inner loops, the time spent
creating and destroying SIDL arrays can represent a significant overhead.
In many cases, this can easily be prevented by moving array creation
and destruction outside the inner loop. In cases where an array is
created within a function call in the original code, the SIDL array
can be made static or created outside the function call and passed
in.
\end{description}

