\subsection{Definitions and Properties}   


\textbf{Definition: Process and Process Rank} \footnote{Process or
  processor? ITAPS\cite{ITAPS} uses term 'processor' instead of 'process'. If
  we assume one process per processor(one core), we can use term
  'processor'.} 

In computing, a process is an instance of a computer program that is
being executed. A program is a collection of instructions, and a
process is the actual execution of those instructions. Multiple
processes may be associated with the same program, but execute
independently. A process has its own variables, stack, and local
memory allocation. At any time, multiple processes may exist on one
physical processor, but cannot be split across multiple processors. 

For an application program run with m processes, the processes are
numbered 0 through m-1. And the assigning number is the rank. The
process with process rank i in the execution is denoted
$P_{r\_i}$.\footnote{To be different from part denotation of $P_i$.}   


\textbf{Definition: Entity}

Entity is the basic unit in the problem domain that the application
program works
with. It has a dimension, and entity of higher dimension can be
bounded by other entities of lower dimension.  
For example, consider a finite element
mesh, an entity is a mesh entity (vertex, edge, face,
or region). A mesh region of tetrahedron
type can be bounded by 4 mesh faces, 6 mesh edges, and 4 mesh
vertices. 

\textbf{Definition: Part Entity}

Part entity is an entity that does not bound any other entities of
higher dimension. The definition of part mesh entity is presented in
section5. 

\textbf{Definition: Entity Group}

Entity group is a group of part entities that are required to stay
together during the process of data partition and migration. The
definitions and operations of mesh entity group are defined in
Section5. 


\textbf{Definition: Part Object} 

A part object is the basic unit of data to which a destination part
can be assigned during the the process of data partition and
migration. It can be a part entity or an entity group. 
 For example, consider a finite element
 mesh, a part object can be a part mesh entity, or a mesh entity
 group. 


\textbf{Definition: Partition}

A partition is a distribution of data into sub-sets called parts
according to some distribution rule, such that one part object is
assigned to a single part. Within a partition, part objects are
assigned to
different parts, and parts are assigned to different processes. The
number of parts is independent from the number of processes. All these
assignments can be changed dynamically. For example, in finite element
space, a mesh partition can be a distribution of a mesh into
sub-meshes.  


\textbf{Definition: Part} 

A part is a collection of subset data. A part is assigned to one and
only one process at any time, i.e, a part can not span across
processes \cite{ITAPS}. A process can contain multiple parts. For example,
consider a finite element mesh, a part is a collection of mesh
entities.  

\textbf{Definition: Part Id}  

For each part, a globally unique part id (short for global part
identifier) can be given over a set of processes within a 
partition. The part with part id i can be denoted $P_i$. \footnote{To
  be 
  consistent with the situation of one part per process, in which the
  value of part id equals to that of the process rank for each part,
  part id will be implemented as a globally unique integer within the
  partition.}  

The application program can access a part though its part id. The
process rank of a process that a part resides can be decided through
the part id. Except the situation of one part per process, the value
of part id is always different from the value of process rank for each
part.  

\textbf{Definition: Part Boundary} 

 A part boundary is a set of data shared by multiple parts, which
 reside in one process or multiple processes. The data on a part
 boundary is in the closure of the part, not belonging to the interior
 of the part. A part can have multiple part boundaries shared by
 different parts. It is also called inter-part boundary. 


\textbf{Definition: Part Boundary Entity}\footnote{Once a part
  boundary
  entity is defined, the information of partition classification and
  residence parts is stored only for a small part of mesh entities.}

Part boundary entity is short for an entity that resides in a part
boundary. Maintaining the information of part boundary entity is very
important to perforam parallel partition and migration within the
partition. For example, in a finite element mesh, the $i^{th}$ part
boundary mesh entity of dimension d is denoted as $M^d_{B\_i}$.


\textbf{Definition: Residence Part} 

For a part object, the residence part is the part where it
resides. For a part boundary entity, the residence part(s) is the
union of residence part(s) where the part object(s) that the part
boundary entity bounds resides. 


\subsection{Functional Requirement Definitions for Partitions}

For a part boundary entity, it must be aware where it is
duplicated. 

\textbf{Definition: Remote Copy} 

Given a part boundary entity $V^d_{B\_i}$ on one part $P_i$, the
memory location of
the part boundary entity duplicated on parts other than part $P_i$. 

The data of a part boundary entity duplicated on
different parts must be consistent at any time. Once a part boundary
entity is
modified on one part, its remote copies must be modified
accordingly.

% remote copy = entity pointer+ part id   

For a part boundary entity, it is beneficial to assign a specific copy
as the owner of the others and let the owner be in charge of
communication or computation between the copies \cite{Seol}. 

\textbf{Definition: Owner Part}

For a part boundary entity, one of the parts sharing the part boundary
entity
will be assigned as its owner part according to some rule dynamically,
such as poor-to-rich ownership, which assigns the poorest part to be
the owner part, where the poorest part is the part that has the least
number of part objects among residence parts of the entity
\cite{Seol}.\footnote{The poor-to-rich assignment may have some problem, since
the poorest part may be not on the the poorest process that has the
least work load.}

\textbf{Definition: Residence Part Operator $\mathcal{P}$} 

An operator that returns a set of part id(s) where part object or part
boundary entity exists. 

For a part object $PO_i$ \footnote{Please refer symbol list in Section4.}, $\mathcal{P}[PO_i]={p}$ where p is the
id of a part where $PO_i$ exists. The part object $PO_i$ can be a part
entity $V^d_{P\_j}$, or an entity group $V^d_{G\_j}$.

For a part boundary entity $V^d_{B\_i}$, 

\begin{equation}
\mathcal{P}[V^{d_i}_{B\_i}]=\{\mathcal{P}[V^{d_j}_{P\_j}|V^d_{B\_i}\in\{\theta(V^{d_j}_{P\_j})]\}\cup\{\mathcal{P}[V^{d_j}_{G\_j}|V^d_{B\_i}\in\{\theta(V^{d_j}_{G\_j})]\}.  
\end{equation}
