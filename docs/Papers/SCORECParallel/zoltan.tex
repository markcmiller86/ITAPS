% This is section{Using Zoltan to Do Partition}

FMDB uses Zoltan load-balancing interface functions to perform
partition and repartition of the part objects across the processes. In
order To use Zoltan in FMDB, three steps are needed. 

1. Assign a global identifier to each part object.
2. Set options to control the behavior of Zoltan, the application
program can choose any option to do multilevel
partitioning  
  geometric, graph or hypergraph based partitioner;
  coarse or refine partition; 
  tolerant imbalancings.  
 
3. Provide necessary query functions to provide part object
information to Zoltan (geometric, graph, and hypergraph query functions). 


\subsection{Define global identifier for part object}   
% part object(po) pointer + process rank 

Each part object in the data is required to have global identifier (ID) in
Zoltan partition. And each part object can be represented as local
memory allocation plus the process rank. This global
ID is easy to build and rebuild, and would save the time to search the
corresponding part object based on a global ID. 

% From FMDB's aspect, to desribe how to provide options and query
% functions to Zoltan. 

\subsection{Set Zoltan's Parameters} % I am not sure that I should use
				% graph node or part object. Even
				% though they have the same meaning. 

A specific structure of partition options is provided by FMDB to the
application program, which can set Zoltan's partition parameters
through the partition options. The structure of the partition options
is listed as follows. Currently, FMDB only uses Zoltan graph-based
load balancing interface functions.  

\begin{itemize}

\item{lb\_method:} The partition algorithm used by Zoltan. The default
  value is 'GRAPH'.

\item{lb\_approach:} The desired load balancing approach. 'PARTITION',
  'REPARTITION' or 'REFINE' \cite{Zoltan}. The default value is 'PARTITION'.  

\item obj\_weight\_dim: The number of weights associated with a graph
  node. The default value is zero, indicates all graph nodes have
  equal weight.  

\item edge\_weight\_dim: The number of weights associated with a graph
  edge. The default value is zero, indicates all graph edges have
  equal weight. 

\item num\_global\_partitions:   The total number of partitions to be
  generated in the partition result. Integer values greater than zero
  are accepted \cite{Zoltan}. The default value is the number of processes. 

\item num\_local\_partitions: 
  The number of partitions to be generated on this processor in the
  partition result. Integer values greater than zero
  are accepted. If any process is set with the option, the value on
  processes not setting this option is assumed to be zero \cite{Zoltan}. The
  default value is one for each process.   

\item imblance\_tol: The amount of load imbalance the partitioning
  algorithm should deem acceptable. The default value is 1.1, which
  ndicates that 10\% imbalance is OK; that is, the maximum over the
  average shouldn't exceed 1.1  \cite{Zoltan}.    

\item debug\_level: An integer indicating how much debugging
  information is printed by Zoltan. Higher values produce more output
  and potentially slow down Zoltan's computation \cite{Zoltan}. 

\item timer: The timer with which the application program wish to measure time. Valid
  choices are wall, cpu, and user  \cite{Zoltan}. 


\end{itemize}


\subsection{Provide Query Functions to Zoltan}

In order to use Zoltan graph-based load-balacing interface functions,
FMDB defines a class  'pmZoltanCallbacks' to interact with Zoltan, and provides four query
functions to Zoltan in that class as follows. Please see Appendix for
the details of the class  'pmZoltanCallbacks'. 

\begin{itemize}

\item \textbf{get\_num\_po} a query function returns the number of
  graph nodes that are currently assigned to the processor.

  
\item \textbf{get\_po\_list} a query function returns the information about
  the graph nodes  currently assigned to the processor, including an
  array of global ID(s) for all graph nodes,  and optionally
  their weights. 

\item \textbf{get\_num\_edges\_multi} a query function returns the number of
  graph edges for connected to each graph node returned by query
  function get\_po\_list. 

\item \textbf{get\_edge\_list\_multi} a query function returns arrays of global
  ID(s), process ID(s), and optionally edge weights for graph nodes
  connected to each graph node returned by query
  function get\_po\_list. 

\end{itemize}


\subsection{Construct Graph Nodes and Graph Edges}   %\input{graph.tex}
