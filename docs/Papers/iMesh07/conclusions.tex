
\section{Current Status and Ongoing Work\label{sec:Conclusions}}

The iMesh interface has been implemented into several existing
meshing tools, including FMDB (RPI), GRUMMP (UBC), MOAB (SNL), Frontier
(SUNY SB), Overture (LLNL), and NWGrid (PNNL).  The interface is used by
several ITAPS-developed mesh services tools including a mesh quality
improvement toolkit (Mesquite)\cite{Mesquite03}, a face- and
edge-swapping service\cite{TSTT-swap-tool}, and a mesh adaptation service.  In
addition, the iMesh interface is used in several
application codes.  The most notable of these is the joint work between
the ITAPS consortium and researchers at the Stanford Linear Accelerator
Center (SLAC).  In this work, researchers are developing the ITAPS-based
mesh services needed in design optimization of accelerator cavities and
to insert a mesh adaptation loop into SLAC's linear accelerator design
code\cite{GeLe04}.

To increase the dissemination of iMesh-compliant tools, we are now
working to establish component-level compliance with the standards of
the Common Component Architecture (CCA) Forum\cite{cca-forum}.  The CCA
Forum is defining a component architecture tailored to address all
aspects of high-performance scientific computing.  As part of this
effort, they are creating the Rapid Application Development environment,
in which iMesh implementations will play a key role.

Work continues to improve the functionality and ease of use of the iMesh
interface.  We are working to adapt our current unit test suite for full
iMesh implementations for use by those requiring correct behavior for
only a subset of functionality to be able to use a particular service.
Also, we expect to leverage current Babel research aimed at improving
software component compliance and usage through interface-level software
contracts\cite{dahlgren:sehpcs04,dahlgren:study04,dahlgren:sehpcs05}.

More information on the iMesh interface, including complete documentation,
can be found at http://www.itaps-scidac.org.

