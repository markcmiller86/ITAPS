%\documentclass[12pt,letterpaper]{article}
% JOURNAL STUFF
\documentclass[global]{svjour}
\journalname{Engineering with Computers}

%\usepackage{imr}
%\usepackage{amssymb}
\usepackage{epsfig}
\usepackage{url}
\usepackage{float}
\floatstyle{ruled}
\newfloat{algorithm}{tbp}{loa}
\floatname{algorithm}{Algorithm}

%\def\thepage {}
%\bibliographystyle{imr}
%\newtheorem{Pseudo}{Pseudo-Code}[section]
\begin{document}

\title{Toward Interoperable Mesh, Geometry and Field Components for PDE Simulation Development}

%% \author{Kyle K. Chand \and 
%%   Lori Freitag Diachin \and 
%%   Xiaolin Li \and
%%   Carl Ollivier-Gooch\and
%%   Mark Shephard \and
%%   Timothy Tautges \and
%%   Harold Trease}
% JOURNAL STUFF
\titlerunning{Interoperable Components for Simulation Development}
\author{Kyle K. Chand\inst{1} \and 
  Lori Freitag Diachin\inst{1} \and 
  Xiaolin Li\inst{2} \and
  Carl Ollivier-Gooch\inst{3}\and
  E. Seegyoung Seol \inst{4}\and
  Mark S. Shephard\inst{4} \and
  Timothy Tautges\inst{5} \and
  Harold Trease\inst{6}}
\institute{Center for Applied Scientific Computing, Lawrence Livermore National Laboratory \and
  Dept. of Applied Mathematics and Statistics, SUNY Stonybrook \and
  Advanced Numerical Simulation Laboratory, University of British Columbia \and
  Scientific Computation Research Center, Rensselaer Polytechnic Institute \and
  Mathematics and Computer Science Division, Argonne National Laboratory \and
  Pacific Northwest National Laboratory}

\maketitle

\begin{abstract}
Mesh-based PDE simulation codes are becoming increasingly
sophisticated and rely on advanced meshing and discretization tools.
Unfortunately, it is still difficult to interchange or interoperate
tools developed by different communities to experiment with various
technologies or to develop new capabilities.  To address these
difficulties, we have developed component interfaces designed to
support the information flow of mesh-based PDE simulations.  We
describe this information flow and discuss typical roles and services
provided by the geometry, mesh, and field components of the
simulation.  Based on this delineation for the roles of each
component, we give a high-level description of the abstract data model
and set of interfaces developed by the Department of Energy's
Interoperable Tools for Advanced Petascale Simulation (ITAPS)
center.  These common interfaces are critical to our interoperability
goal, and we give examples of several services based upon these
interfaces including mesh adaptation and mesh improvement.
\end{abstract}

\section{Introduction}
\section*{List of Abbreviations}
\begin{multicols}{3}
\begin{lyxlist}{Topo}
\item[AI]{Adjacency information enum}
\item[EH]{Entity handle}
\item[ES]{Entity set handle}
\item[ET]{Error type enum}
\item[iter]{Iterator over entities}
\item[SH]{Set handle}
\item[SO]{Storage order enum}
\item[TH]{Tag handle}
\item[TVT]{Tag value type enum}
\item[Topo]{Entity topology enum}
\item[Type]{Entity type enum}
\item[VH]{Vertex handle}
\end{lyxlist}
\end{multicols}

\section{Introduction\label{sec:Introduction}}

Creating simulation software for problems described by partial
differential equations is a relatively common but very time-consuming
task. Much of the effort of developing a new simulation code goes into
writing infrastructure for tasks such as interacting with mesh and
geometry data, equation discretization, adaptive refinement, design
optimization, etc. Because these infrastructure components are common to
most or all simulations, re-usable software for these tasks would
significantly reduce both the time and expertise required to create a
new simulation code.

Currently, libraries are the most common mechanism for software re-use
in scientific computing, especially the highly-successful libraries for
numerical linear algebra\cite{petsc,BaGr97,eispack,lapack,linpack}.
The drawback to software re-use through libraries is the difficulty in
changing from one to another. When a user wishes to add functionality or
simply experiment with a different implementation of the same
functionality in another library, all calls within an application must
be changed to the other API, which likely will not package functionality
in precisely the same way. Another significant challenge with library
use, especially in the context of meshing and geometry libraries, is
that data structures used within the libraries may be radically
different, making changes from one library to another even more onerous.
This time-consuming conversion process can be a significant diversion
from the central scientific investigation, so many application
researchers are reluctant to undertake it. This can lead to the use of
sub-optimal strategies.  For example, new advances developed by the
meshing research community often take years to become incorporated into
application simulations.

To address these issues, the Interoperable Tools for Advanced Petascale
Simulation (ITAPS) center is working to develop interoperable software
tools for meshes, domain geometry, and
discretization\cite{tstt:overview}.  The present paper will discuss our
work in developing a mesh interface.  The most prominent example of
prior research in defining interfaces for meshing is the Unstructured
Grid Consortium, a working group of the AIAA Meshing, Visualization, and
Computing Environments Technical Committee\cite{UGC-web}.  The first
release of the UGC interface\cite{UGC-v1} was aimed at high level mesh
operations, including mesh generation and quality assessment.
Recognizing a need for lower-level functionality, the UGC has developed
a low-level query and modification interface for mesh databases, as well
as an interface for defining generic high-level
services\cite{UGC-v2:paper}.

The ITAPS mesh interface, called iMesh, has a broader scope than the UGC
interface.  In addition to supporting low-level mesh manipulation, the
iMesh interface is also designed to support the requirements of solver
applications, including the ability to define mesh subsets and to attach
arbitrary user data to mesh entities.  In addition, the iMesh interface
is intended to be both language and data structure independent.  In
summary, our initial target is to support low-level interaction between
applications programs --- both meshing and solution applications --- and
external mesh databases regardless of the data structures and
programming language used by each.  In the long term, we expect to also
support high-level operations, including mesh generation, typically as
services built using the iMesh interface.

The fundamental challenge in developing this interface has been the
tension between generality and compactness: our goal has been to define
a set of operations addressing all common uses of mesh data while
minimizing redundancy and avoiding idioms peculiar to a particular
underlying mesh representation.  A common theme in many design decisions
while developing the iMesh interface has been to support common
constructs as simply, directly, and efficiently as possible while still
allowing more sophisticated, less common constructs to be expressed in
the data model and interface.

We began by defining a general abstract data model, focusing on the ways
in which mesh data is used in simulations rather than on how mesh data
is stored by meshing tools.  The data model, described in detail in
Section~\ref{sec:Data-Model}, includes fundamental mesh entities ---
vertices, faces, elements, etc --- and the topological relationships
between them, as well as the concepts of general mesh subsets and
arbitrary data associated with mesh entities.

The mesh interface is built on this data model.  The interface
(Section~\ref{sec:Interface}) supports global and local mesh query, mesh
modification, and collections and tagging of mesh entities.  
% This needs to be said somewhere, but where exactly?  Is this the right
% place? 
The iMesh interface is built on a client-server model, with the
explicit assumption that the client (application) and server (mesh
database) may be written in different programming languages.  To address
cross-language issues, especially with arrays and strings, the iMesh
interface is defined using the Scientific Interface Description Language
(SIDL)\cite{babel:site05,babel:usersguide05}.  This language neutral
description is then processed by an existing interpreter, Babel, to
produce a language-specific client API and server skeleton, as well as
glue code that mediates language translation issues.

Performance data from early usage of the interface suggests there is a
preferred coding style for using the iMesh interface (see
Section~\ref{sec:Programming}).  The interface is already in use in
various meshing tools and simulation applications and on-going
development continues to improve the usability and accessibility of
iMesh-compliant software (see Section ~\ref{sec:Conclusions}).



\section{Information Flow in Mesh-based Simulation}\label{sec:infoFlow}
In mesh-based analysis procedures the PDEs are solved approximately
through a double discretization process in which the problem's physical
domain is discretized into a set of piecewise components (e.g. a mesh)
and the PDEs to be solved are discretized over the mesh in an
appropriate manner. The discretizations of the PDEs over the mesh are
assembled into a fully discrete system that is solved. The result of
this process yields the construction of a set of discretized solution
fields. Methods covered by such approaches include finite difference,
finite volume, finite element, boundary element and partition of unity
(so-called meshfree) methods.

\subsection{Problem Definition}

To qualify the operations and information involved in executing a
mesh-based simulation, we begin with a qualification of the problem
definition which includes:

\begin{itemize}
\item The domain over which the simulation is to be solved. For the
classes of simulations being considered here the domain includes a
spatial component that is one-, two-, or three-dimensional. The problem
can also be defined over time in which case the domain also includes a
temporal component.

\item The mathematical form governing the simulation (PDE�s,
variational principle, weak form).

\item Specification of the parameters, referred to as physical
attributes, associated with the governing mathematical equations that
includes:

\begin{itemize}
\item material properties
\item forcing functions
\item boundary conditions
\item initial conditions
\end{itemize}
\end{itemize}

From the viewpoint of supporting a numerical simulation, the
domain representation must be able to:

\begin{itemize}
\item Support the construction of a mesh that represents the domain
level discretization used in the simulation.

\item Support the ability to address any geometry interrogation required.

\item Support the proper association of the physical and mathematical
attributes with the mesh.

\item Support the domain evolution in the cases where the domain
changes as part of the solution process.
\end{itemize}

There are multiple sources for the high level definitions of the
spatial component of the domain with CAD models, image data and
cell-based (mesh-based) being the most common. Each of these sources
has one or more representational forms. Historically, CAD systems use
boundary representations. Image data use a volumetric
form such as voxels or octrees. Depending on the configuration of the
cells a variety of implicit and explicit boundary or volumetric
representations have been used.

Except in cases of image data and when all aspects of the simulation
process can be effectively defined in terms of volume entities, it
is generally accepted that the use of a boundary
representation is well suited for the spatial domain definition. There
is a substantial computer-aided design literature on the various
boundary representations. Common to these representations is
the use the abstraction of topological entities and their adjacencies
to represent the model entities of different dimensions. In a
boundary representation the information defining the shape of
the topological entities can be thought of as attribute information
associated with the appropriate entities. The ability to interact with
topological entities provides an
effective means to develop abstract interfaces
allowing the easy integration to multiple domain definition
sources.

An important consideration in the selection of a boundary
representation is its ability to represent the classes of domain
needed. In the case of numerical simulations the domains to be meshed
can be general combinations of 0-, 1-, 2- and 3-D entities in general
configurations. Figure 1 shows a typical analysis domain that may be
used for the structural analysis of a portion of a piping
system. The analysis domain is an idealization
where portions of the pipes are idealized by beams,
the support bracket idealized by a plate
and a full 3-D solid used for the pipe juncture.



Figure 1. Example of a non-manifold model used in simulation.


The proper representation of such geometric domains, as
well as others like multi-material domains, are
referred to as a non-manifold boundary representations \cite{GuCh90,We88}. In
the case of non-manifold models the representation must indicate
how topological entities are used by bounding higher order
entities. For example, each side of a face may be used by a different
region. Therefore, faces have two uses. Another terminology for the
use of a topological entity by higher order topological entities is
co-entities \cite{Ta00}.

Geometric modeling systems maintain tolerance information on how
numerically well the entities fit together.
This is necessitated by the fact that to function properly geometric
modeling systems must employ finite tolerances.
The algorithms and methods within the geometric
modeling system use the tolerance information to
effectively define and maintain a consistent representation of the
geometric model. (What many geometry-based
applications have referred to as dirty geometry is caused by a lack of
knowledge and proper use of the tolerance information \cite{BeWa04}.)

The abstraction of topology provides an effective means to develop
functional interfaces to boundary-based modelers. The ability to
generalize these interfaces is further enhanced by the fact that
the geometry shape information needed by most all
simulation procedures consists of pointwise interrogations 
that can be easily answered in a method independent of 
the modeler shape representation.

The developer of CAD systems support
geometry-based applications through general APIs.
These geometric modeling APIs have been successfully
used to developed automate finite element modeling processes
\cite{BeWa04,ShGe92}.

In some cases the only domain representation is a
mesh. In these cases it is still desirable
to construct a high level topological representation of the problem
domain. In this case the
process of constructing, or updating, the topological entities
associated with the domain geometric model is focused
on determining the appropriate sets of mesh regions, faces, edges, and
vertices to associate with the model regions, faces, edges and
vertices respectively. Algorithms to do this based on mesh based
geometry and/or simulation contact or fracture information
have been developed \cite{KrOr01,PaOr02,WaSh05}.
Once the model topology has been set,
the geometric shape information can be defined in terms of
the mesh facets, or can be made higher order \cite{CiOr00,OwWh01}.

An examination of the properties of analysis attributes indicates
they are tensorial quantities \cite{BeSo83} that
must be defined with respect to a coordinate system. Generalized 
structures and methods can define analysis attributes and associate them
geometric model entities \cite{OBSh02}.

\subsection{Domain Discretization}

The mesh, is a piecewise decomposition of the space/time domain. 
The specifics of the definition of the mesh is a function of the 
methods used to discretize the equations over the mesh entities.

It is common to employ different discretizations for the spatial and
temporal domains.  Since the definition of the spatial mesh is 
typically the more complex of the two, it is the focus of
this discussion. The requirements of the mesh are:

\begin{itemize}
\item To have the appropriately defined �union� of the mesh
entities represent the domain of interest.

\item To maintain, or have access to, the geometric shape information
needed for processes such as differentiation and integration.

\item To support the PDE discretization process over the mesh entities.

\item To maintain relationships of the mesh entities needed to support
the assembly of the complete discrete system and construction of the
solution fields.
\end{itemize}

One common mesh form is the conforming mesh where the intersections of
two mesh entities is null and the intersections of their closure is
either null or the closure of a common boundary mesh entity (face,
edge or vertex). A variant of this is the non-conforming mesh where
the intersections of the closure of two mesh entities is null, or
different parts of the boundaries of the two mesh entities. Other mesh
structures employ mesh patches that can interact in a variety of
ways. Finally, other methods are defined in terms of overlapping
regions (e.g., spheres or cubes). In each of these cases there are
rules on how the mesh entities interact, how
equation discretizations are performed over them, and how the complete
discrete system is assembled.

The geometric shape of the mesh entities must be
understood to support the equation discretization process. In
many methods the mesh geometry is implied from a discrete set of
parameters, which is satisfactory for fixed mesh simulations. An
alternative is to maintain
a linkage to the high level domain geometry. This alternative tends to
be expensive so
a mesh based geometric definition is typically used. However,
in the case of adaptive mesh improvements it is necessary to use the
links back to the original domain geometry to ensure the mesh
geometric approximation improves in a manner consisted with the order
of accuracy provided by the equation discretization process. For
example, as piecewise linear elements approximating curved portions of
the geometry are refined, the new mesh vertices need to be placed on
the curved boundary, or as the polynomial order
of an element is increased, the geometric approximation of the closure
of that entity must be increased to the correct order.

The data model for the mesh must maintain an association with the
domain definition, the discretization functions, the assembled
discrete system and the solution fields. From the perspective of
maintaining its relationship to the geometric domain, the use
of topological entities and their adjacency is ideal
\cite{BeSh97,DeOB01,Ta00}. In this manner it is possible to associate
the mesh entities to the domain entities to obtain needed attributes
and geometric information. This association between the two model
structures is referred to as classification below.

In other cases, such a representation is not ideal.
For example, something like an octree, or some other
spatially-based structure, is appropriate for the partition of unity
(so call meshfree) methods. In the case of structured meshes
maintaining a complete topology down to the individual cell entities
would be overkill.
However, in both of these cases there is information that is well
suited to a topological representation. For example, the boundaries
of the mesh patches in an structured mesh are ideally
defined in terms of a topological structure augmented with the rules
of mesh patch interaction. In the case of partition of unity
methods, maintaining topological entities of the cells of an
octree effectively supports the needed operations \cite{KlSh00}.

Consider the case of using a topological structure for the mesh.
Under the assumption that each topological mesh
entity of dimension $d$, $M^d_i$, is bounded by a set of topological
mesh entities of dimension $d-1$, 
$\left\{ M^d_i \left\{ M^{d-1} \right\} \right\}$, 
the full set of mesh topological entities are:

\begin{equation}
T_M = \left\{ \left\{ M \left\{ M^0 \right\} \right\},~
\left\{ M \left\{ M^1 \right\} \right\},~
\left\{ M \left\{ M^2 \right\} \right\},~
\left\{ M \left\{ M^3 \right\} \right\} \right\}
\end{equation}

where $\left\{ M \left\{ M^{d} \right\} \right\}$, $d=0,1,2,3$, are
respectively the set of vertices, edges, faces and regions which
define the topological entities of the mesh domain. It is
possible to limit the mesh representation to just these entities under
the following restrictions \cite{BeSh97}.

\begin{enumerate}
\item Regions and faces have no interior holes.

\item Each entity of order $d_i$ in a mesh, $M^{d_i}$, may use a particular entity of
lower order, $M^{d_j}$, $d_j<d_i$, at most once.

\item	For any entity $M^{d_i}_i$ there is a unique set of entities of order $d_i-1$,
$\left \{ M^{d_i}_i \left\{M^{d_{i-1}} \right\} \right\}$  that 
are on the boundary of $M^{d_i}_i$.
\end{enumerate}

The first restriction means that regions may be directly represented
by the faces that bound them, faces may be represented by the edges
that bound them, and edges may be represented by the vertices that
bound them. The second restriction allows the orientation of an entity
to be defined in terms of its boundary entities.
For example, the orientation of an edge,
$M^1_i$ bounded by vertices $M^0_j$ and $M^0_k$ is uniquely defined as
going from $M^0_j$ to $M^0_k$ only if $j \neq k$.

The third restriction means that a mesh entity is uniquely specified
by its bounding entities. Most representations including that used in
this paper employ that
requirement. There are representational schemes where this condition
only applies to interior entities; entities on the boundary of the
model may have a non-unique set of boundary entities \cite{BeSh97}.

A key component of supporting mesh-based simulations is the
association of the mesh with respect to the geometric model
\cite{BeSh97,ShGe92}. This association is referred to as
classification in which the mesh topological entities are classified
with respect to the geometric model topological entities upon which
they lie.

{\bf Definition: Classification} - {\it The unique association of mesh
topological entities of dimension $d_i$, $M^{d_i}_i$ to the
topological entity of the geometric model of dimension $d_j$,
$G^{d_j}_j$ where $d_i \leq d_j$, on which it lies is termed
classification and is denoted $M^{d_i}_i \sqsubseteq G^{d_j}_j$
where the classification symbol, $\sqsubseteq$,
indicates that the left hand entity, or set, is classified on the
right hand entity.}

{\bf Definition: Reverse Classification} - {\it For each model
entity, $G^d_j$ , the set of equal order mesh entities classified on that
model entity define the reverse classification information for that
model entity. Reverse classification is denoted as:}

\begin{equation}
RC(G^d_j) = \left\{ M^d_i | M^d_i \sqsubseteq G^d_j \right\}
\end{equation}

The concept of mesh entity classification to a higher level
model can be extended to include additional levels of model
decomposition. Two important cases of this are parallel mesh
partitions and structured mesh partitions. In the cases when these
partitions are non-overlapping the associations are obvious.
The concepts can be extended to the case of
overlapping partitions through the definition of an appropriate
rules of the interaction of entities in the different models.

Mesh shape information can be effectively associated with the
topological entities defining the mesh. In many cases this is limited
to the coordinates of the mesh vertices and, if they exist, higher
order nodes associated with mesh edges, faces or regions. In addition,
it is possible to associate other forms of geometric information with
the mesh entities. For example, the association of Bezier curves and
surface control points with mesh edges and faces for use in p-version
finite elements \cite{LuSh02}. The mesh classification can be can be
used to obtain other needed geometric information such as the
coordinates of a new mesh vertex caused by splitting a mesh edge
classified on a model face.

\subsection{Equation Discretization and the Definition of Solution Fields}

The PDEs being solved are written in terms of dependent
variables that are functions of the
space/time domain. For purposes of this discussion, consider the set of
PDEs being solved are written in the form:

\begin{equation}
{\cal{D}}({\bf u}, \sigma) - f = 0
\end{equation}

where 

\begin{itemize}
\item $\cal{D}$ represents the appropriate differential operators.

\item $\bf{u} (\bf{x},t)$ represents one of more vector dependent variables which are
functions of the independent variables of space, $\bf{x}$, and time, $t$ .

\item $\sigma$ represents one of more scalar dependent variables which are
functions of the independent variables of space, $\bf{x}$, and time, $t$.

\item $f$ represents the forcing functions.
\end{itemize}

(Note that the complete statement of a PDE problem must include a set
of boundary and, for time dependent problems, initial conditions.)

In the double discretization process used in mesh-based PDE solvers,
the dependent variables are discretized over the individual, or
groups of, mesh entities, either by direct operator discretization
(e.g., difference equations) or in terms of a set of basis
function. In both cases this process specifies a set of distribution
functions defining how the discretized variables vary over the mesh
entities and a set of yet to be determined multipliers, called degrees
of freedom (dof). The dof can always be
associated with a single mesh entity while the distribution functions
are associated with one or more mesh entities.
Three common cases that employ different combinations
of interactions between the mesh entities, the dof and the
distributions are:

\begin{itemize}
\item Finite difference based on a vertex stencil: In this case
the distribution functions are difference stencils
written in terms of dof that are the value of the field at
specific neighboring points. The dof are associated with mesh
vertices. The difference stencil is defined over the mesh entities 
that link the vertices involved with the stencil.

\item Finite volume methods: Finite
volume methods are constructed in terms of distribution function
written over individual mesh entities, referred to as cells. In most
cases the field being defined is $C^{-1}$ and the dof are not shared between
neighboring mesh entities. In this case the dof are associated
with the mesh entity the distribution is written over. The
coupling of the dof from different mesh entities is then through
operators acting over common boundary mesh entities.

\item Finite elements with common dof between neighboring elements:
Finite element distribution functions, referred to as shape functions,
are written over individual mesh entities, referred to as elements. In
cases where $C^m,~ m \geq 0$, continuity is required, the
shape functions associated with neighboring elements are made $C^m,~m
\geq 0$, continuous by having common dof associated with the bounding
mesh entities common to the neighboring elements. For example,
a $C^0$ field between two neighboring quadratic
elements in 2-D can be obtained by using the values of the field at one
point on the common edge and at the two
vertices bounding that edge as dof. In this
case the full set of dof used by the element distribution function can
be dof associated with any of the mesh entities in the closure of the
mesh entity of the element. There are other means to meet even higher
order continuity requirements, all
of which require sharing dof on the common boundaries.
\end{itemize}

The process of applying the discretization operation over the
appropriate mesh entities will produce a local contribution to
the complete fully discrete system. The
processes can be stated symbolically as:

\begin{equation}
\cal{D} (\bf{D}^c, \bf{d}^c) - \bf{f}^c = 0 
\end{equation}

where: 

\begin{itemize}
\item $\cal{D}$ represents the discretized differential operators written in terms
of appropriate distribution functions, $D^c$, over the domain of the
contributor $C$ and $\bf{d}^c$ represents the vector of dof associated with that
contributor.

\item $\bf{f}^c$ represents the discretized representation of the known
``forcing functions'' and boundary conditions for that
contributor.
\end{itemize}

The result of the discretization process yields a discrete
representation of the original PDEs that can be written as:

\begin{equation}
\bf{k}^c \bf{d}^c = \bf{f}^c
\label{eq:contrib_matrix}
\end{equation}

where $\bf{k}^c$ is a matrix of parameters for contributor $C$ that multiple the
vector of dof associated with that contributor, $\bf{d}^c$.

The construction of the system contributors can be controlled by the
appropriate traversal of information in the high level problem
definition, or at a level above the mesh such as the mesh patch level
for structured methods.

Note that the solution fields represent the variations of the tensor
variables over the domain of the problem. These fields
must be maintained is a form useful for the application of queries
and manipulation as needed for operations that include:

\begin{itemize}
\item The accurate transfer of the fields to other meshes to provide
input in a multiphysics analysis step, or to maintain the description
of the mesh on an adapted field.

\item The construction of new fields through operations
that may project them on new distribution with higher order
continuity, combine with other fields, etc.
\end{itemize}

\subsection{Discretized System Construction and Solution}

The relationship of the contributor level discretization given in
(\ref{eq:contrib_matrix})
to the complete discrete system is dictated by contributor level
mappings that ``map'' the contributor level dof to the
``assembled'' vector of the dof for the complete system, $\bf{d}$. The
process of constructing the complete system from the contributors is
referred to as the assembly process. Symbolically the complete
discrete system can be written as:

\begin{equation}
\bf{K} \bf{d} = \bf{F}
\end{equation}

where

\begin{itemize}
\item $\bf{K}$ is a system level matrix of parameters.
\item $\bf{d}$ is the complete vector of dof 
\item $\bf{F}$ is the complete right hand side vector.
\end{itemize}

Symbolically the relationship between the contributor level and system
level matrices and vectors can be depicted as:

\begin{equation}
\bf{d} ~=~ A^{N_c}_{c=1}(\bf{d}^c),~K~=~A^{N_c}_{c=1}(\bf{k}^c),~\bf{F}~=~A^{N_c}_{c=1}(\bf{f}^c)
\end{equation}

where 
\begin{itemize}
\item $N_c$ is the number of contributors in the complete system
\item $A^{N_c}_{c=1}$ indicates an assembly operator that is applied to each
contributors contributions and properly maps it to the complete
discrete system.
\end{itemize}

There are a variety of specific representational forms for the
complete systems matrices.  The specific form used is function of the
methods used to perform the computationally intensive process of
solving the discrete system to determine the values of the system dof.
The global algebraic equations are solved to produce the values of the
system dofs. Once the system level dof are determined, the mappings
between the contributor level and system level dof can be used to
complete the specification of the solution fields.





\section{The ITAPS Interface Definition efforts}\label{sec:tsttdef}
To support the flow of information in mesh-based simulations, a number
of tools and technologies have been developed by different research
groups in academia, industry, and the government labs.  For these tools
to have maximum impact, it is important that they be interoperable,
interchangeable, and easily inserted into existing application
simulation codes.  Accomplishing this goal will allow easier
experimentation with different, but functionally similar, technologies
to determine which is best suited for a given application.  In
addition, it will provide mechanisms for combining technologies
together to create hybrid solution techniques that use multiple
advanced tools.  To accomplish this goal, we have defined an abstract
data model that encompasses a broad spectrum of mesh types and usage
scenarios {\it and} a set of common interfaces that are implementation
and data structure neutral.  Our goal has been to keep the interfaces
small enough to encourage adoption but also flexible enough to support
a broad range of mesh types.

The ITAPS data model partitions the data required by a simulation
into three {\it core data types}: the geometric data, the mesh
data, and the field data.  Interfaces to the data represented by these
abstractions channel the flow of information throughout the
simulation.  For example, ITAPS adaptive mesh refinement services
access solution information for error estimation via the field
interface; modify the mesh using the mesh interface; and query the
geometry interface when creating mesh entites on domain boundaries.
These core data types are associated with each other through {\it data
relation managers}. The data relation managers control the
relationships among two or more of the core data types, resolve cross
references between entities in different groups, and can provide
additional functionality that depends on multiple core data types.
In addition, there are a number of basic functionalities and concepts
that are common to all three of the core data types, for example,
entities, creating sets of entities, and attaching user-defined data
to entities.  We discuss these concepts in Section \ref{sec:utilities}.
Work on the mesh data model and application programming interface
(API) has progressed the farthest, and we describe it in some detail
in Section \ref{sec:mesh}.  Preliminary work on the geometry and field
data model and interfaces are discussed as well in Sections \ref{sec:geom} and
\ref{sec:fields}.

A key aspect of the ITAPS approach is that we do not enforce any
particular data structure or implementation with our interfaces,
requiring only that certain questions about the geometry, mesh, or
field data can be answered through calls to the interface.  To
encourage adoption of the interface, we aim to create a small set of
interfaces that existing mesh and geometry packages can support.  The
latter point is critical.  The DOE, NSF, DoD and other federal
agencies have invested hundreds of person-years in the development of
a wide variety of geometry, mesh generation and mesh management
toolkits.  These software packages will not be rewritten from scratch
to conform to a common API, rather the API must be data structure
neutral and allow for a broad range of underlying mesh, geometry, and
field representations. However, only a small set of functionalities
can be covered by a 'core' set of interface functions.  To increase
the functionality of the ITAPS interface, we define additional,
optional, interfaces for which we will provide reference
implementations based on the core interface methods.  Developers can
incrementally adopt the interface by implementing the optional
functions on their own mesh database as needed.

One of the most challenging aspects of this effort remains balancing
performance of the interface with the flexibility needed to support a
wide variety of mesh types.  Performance is critical for kernel
computations involving mesh and geometry access.  To address this
need, we provide a number of different access patterns including array
and iterator-based.  The user may choose the access pattern that is
best suited for their application; the underlying implementation must
provide both styles of access even though only one is likely to be
native.  Further challenges arise when considering the support of many
different scientific programming languages.  This aspect is addressed
through our joint work with the Common Component Architecture Forum
\cite{cca-forum} to provide
language independent interfaces by using their SIDL/Babel technology
\cite{babel}.  

%%{\tt Insert example here}

\subsection{The ITAPS Basic Interface}
\label{sec:utilities}

The ITAPS data models for mesh, geometry and fields all make use of the
concepts of {\it entities}, {\it entity sets}, and {\it tags}, and we
describe these now in some detail.

ITAPS {\it entities} are used to represent atomic pieces of information
such as vertices in a mesh or edges in a geometric model.  To allow
the interface to remain data structure neutral, entities (as well as
entity sets and tags) are uniquely represented by opaque handles.
Unless entities are added or removed, these handles must be
invariant through different calls to the interface in the lifetime of
the ITAPS interface, in the sense that a given entity will always have
the same handle.  This is required to ensure consistency among the
several different calls that use and return entity handles and to
allow for easy entity handle comparison.  Entities do not have
interface functionality that is separate from mesh, geometry or field
interfaces, and we describe these functionalities in more detail in
the sections that follow.

Entity adjacency relationships define how the entities connect to
each other and both first-order and second-order adjacencies are
supported for the mesh and geometry interfaces.
\begin{itemize}
\item {\it First-order adjacencies}: For an entity of dimension $d$,
first-order adjacencies return all of the entities of dimension
$q$, which are either on the closure of the entity ($d > q$, downward
adjacency), or which it is on the closure of ($d < q$, upward
adjacency).  

\item {\it Second-order adjacencies}:  Many applications require not
only information about first-order adjacencies, but also about the next
level of neighbors. Although such information can always be determined
from the appropriate first-order adjacencies, their application is
common enough that supporting a second-order adjacency function is
useful. A second-order adjacency determines the set of topological
entities of a given type adjacent to entities that share common
boundary entities of the specified type. An example would be
the set of regions that share a bounding edge with the given region.
\end{itemize}

An ITAPS {\it entity set} is an arbitrary collection of ITAPS entities
that have uniquely defined entity handles.  Each entity set may be an
unordered set or it may be a (possibly non-unique) ordered list of
entities.  When an ITAPS interface is first created in a simulation, a
{\it Root Set} is created.  The root set can be populated by string
name using the {\tt load} function call.  The action taken by {\tt
load} is implementation specific and can range from reading mesh data
from a file to generating a mesh on the fly from a named CAD file.

Two primary relationships among entity sets are supported:

\begin{itemize}
\item Entity sets may {\it contain} one or more entity sets.  An
entity set contained in another may be either a subset or an element
of that entity set.  The choice between these two interpretations is
left to the application; ITAPS supports both interpretations. If entity
set A is contained in entity set B, a request for the contents of B
will include the entities in A and the entities in sets contained in A
if the application requests the contents recursively.  We note that
the {\it Root Set} cannot be contained in another entity set.

\item {\it Parent/child relationships} between entity sets are used to
represent relations between sets, much like directed edges connecting nodes in
a graph.  This relationship can be used to indicate that two meshes
have a logical relationship to each other, including multigrid and
adaptive mesh sequences. Because we distinguish between parent and
child links, this is a directed graph. Also, the meaning of cyclic
parent/child relationships is dubious, at best, so graphs must be
acyclic. No other assumptions are made about the graph.
\end{itemize}

Users are able to query entity sets for their entities and entity
adjacency relationships.  Both array- and iterator-based access
patterns are supported.  In addition, entity sets also have "set
operation" capabilities; in particular, existing ITAPS entities may be
added to or removed from the entity set, and sets may be subtracted,
intersected, or united.  

ITAPS {\it tags} are used as containers for user-defined opaque data that
can be attached to ITAPS entities and entity sets.  Tags can be
multi-valued which implies that a given tag handle can be associated
with many different entities.  In the general case, ITAPS tags do not
have a predefined type and allow the user to attach any opaque data to
ITAPS entities.  To improve ease of use and performance, we support
three specialized tag types: integers, doubles, and entity handles.  Tags
have and can return their string name, size, handle and data.  Tag
data can be retrieved from ITAPS entities by handle in an agglomerated
or individual manner.  The ITAPS implementation is expected to allocate
the memory as needed to store the tag data.

\subsection{The ITAPS Mesh Interface}
\label{sec:mesh}

ITAPS {\it mesh entities} are the fundamental building blocks of the
ITAPS mesh interface and correspond to the individual pieces of the
domain decomposition (mesh).  Under the assumption that each
topological mesh entity of dimension $d$, $M^d_i$, is bounded by a set
of topological mesh entities of dimension $d-1$, $\left\{ M^d_i
\left\{ M^{d-1}
\right\} \right\}$, the full set of mesh topological entities are:
\begin{equation}
T_M = \left\{ \left\{ M \left\{ M^0 \right\} \right\},~
\left\{ M \left\{ M^1 \right\} \right\},~
\left\{ M \left\{ M^2 \right\} \right\},~
\left\{ M \left\{ M^3 \right\} \right\} \right\}
\end{equation}
where $\left\{ M \left\{ M^{d} \right\} \right\}$, $d=0,1,2,3$, are
respectively the set of vertices, edges, faces and regions which
define the topological entities of the mesh domain. It is
possible to limit the mesh representation to just these entities under
the following restrictions \cite{BeSh97}.
\begin{enumerate}
\item Regions and faces have no interior holes.
\item Each entity of order $d_i$ in a mesh, $M^{d_i}$, may use a particular entity of
lower order, $M^{d_j}$, $d_j<d_i$, at most once.
\item	For any entity $M^{d_i}_i$ there is a unique set of entities of order $d_i-1$,
$\left \{ M^{d_i}_i \left\{M^{d_{i-1}} \right\} \right\}$  that 
are on the boundary of $M^{d_i}_i$.
\end{enumerate}

The first restriction means that regions may be directly represented
by the faces that bound them, faces may be represented by the edges
that bound them, and edges may be represented by the vertices that
bound them. The second restriction allows the orientation of an entity
to be defined in terms of its boundary entities.  For example, the
orientation of an edge, $M^1_i$ bounded by vertices $M^0_j$ and
$M^0_k$ is uniquely defined as going from $M^0_j$ to $M^0_k$ only if
$j \neq k$. The third restriction means that a mesh entity is uniquely
specified by its bounding entities. Most representations including
that used in this paper employ that requirement. There are
representational schemes where this condition only applies to interior
entities; entities on the boundary of the model may have a non-unique
set of boundary entities \cite{BeSh97}.

Specific examples of mesh entities include, for example, a hexahedron,
tetrahedron, edge, triangle and vertex.  Mesh entities are classified
by their entity type (topological dimension) and entity topology
(shape).  Just as for geometric entities, allowable mesh entity types
are vertex (0D), edge (1D), face (2D), and region (3D).  Allowable
entity topologies are point (0D); line segment (1D); triangle,
quadrilateral, and polygon (2D); and tetrahedron, pyramid, prism,
hexahedron, septahedron, and polyhedron (3D); each of these topologies
has a unique entity type associated with it.  Mesh entity geometry and
shape information is associated with the individual mesh entities. For
example, the vertices will have coordinates associated with them.
Higher-dimensional mesh entities can also have shape information
associated with them. For example the coordinates of higher-order
finite-element nodes can be associated with mesh edges, faces, and
regions.
 
Higher-dimensional entities are defined by lower-dimensional entities
with shape and orientation defined using canonical ordering
relationships.  To determine which adjacencies are supported by an
underlying implementation, an adjacency table is defined which can be
returned by a query through the interface.  The implementation can
report that adjacency information is always, sometimes, or never
available; and to be available at a cost that is constant, logarithmic
(i.e., tree search), or linear (i.e., search over all entities) in the
size of the mesh.  The use of a table allows the implementation to
provide separate information for each upward and downward adjacency
request.  If adjacency information exists, entities must be able to
return information in the canonical ordering using both individual and
agglomerated request mechanisms.

ITAPS {\it mesh entity sets} are extensively used to collect mesh
entities together in meaningful ways, for example, to represent the
set of all faces classified on a geometric face, or the set of regions
in a domain decomposition for parallel computing.  For some
computational applications, it is useful for entity sets to comprise a
valid computational mesh.  The simplest example of this is a
nonoverlapping, connected set of ITAPS region entities, for example,
the structured and unstructured meshes commonly used in finite element
simulations.  Collections of entity sets can compose, for example,
overlapping and multiblock meshes. In both of these examples,
supplemental information on the interactions of the mesh sets will be
defined and maintained by the application.  We note that in other
cases, for example, smooth particle hydrodynamic (SPH) applications,
molecular dynamics, or mesh-free methods, one can use meshes that
consist of a collection of ITAPS vertices with no connectivity or
adjacency information.

The mesh interface, including the use of mesh entity sets, is extendable to
include ``modification operators'' that change the geometry and topology.
Capabilities include changing vertex coordinates and adding or deleting
entities. No validity checks are provided with this basic interface so that
care must be taken when using these interfaces.  These interfaces are intended
to support higher-level functionality such as mesh quality improvement,
adaptive schemes, front tracking proceedures, and basic mesh generation
capabilities, all of which would provide validity checking.  Modifiable meshes
require interactions with the underlying geometric model including classifying
entities.  

Several implementations of the ITAPS mesh interface are well underway and are
supported by mesh management toolkits such as FMDB (RPI)~\cite{ReSh03},
MOAB (SNL)~\cite{moab:http}, NWGrid
(PNNL)~\cite{nwgrid:http}, and GRUMMP (University of British
Columbia)~\cite{GRUMMP:http}.  In addition to the development of underlying
implementations, the ITAPS mesh interface has also been used in a variety of
contexts as well.  In particular, it serves as the interface to the Mesquite
mesh quality improvement and Frontier front tracking tools (see Section 4).


\subsection{Geometry}




\subsection{The ITAPS Fields Interface}
\label{sec:fields}

Simulation fields represent tensor quantities defined in terms of
numerical analysis discretizations in a form useful to support queries
and operations by other functions or simulations. Common examples
where fields are used are (i) multiphysics analysis where the solution
fields from each physics analysis represents a forcing function or
boundary condition for another, (ii) the construction of external
adaptive control loops where the solution fields are used by error
estimation procedures to obtain estimates of the discretization errors
and to construct new mesh size field, and (iii) visualization and 
analysis/postprocessing.

Tensor quantities used in the quantification of problems of
mathematical physics are of order zero or greater and are defined over
a physical space or space/time domain.  Knowledge of the order of a
tensor and the dimension of the spatial domain over which it is defined,
gives the number of components needed to uniquely define the tensor
\cite{BeSo83}. The symmetries, for tensors of order 2 or greater,
define those components that are identical to, or the negative of
(antisymmetric), other components. The components of the tensor are,
in general, functions of the domain parameters as well as other
problem parameters. The ability to understand and use a tensor at any
particular instant requires knowledge of the coordinate system in
which the components of the tensor are referred.

The qualification of a tensor over a domain is called a field. The
field inherits the tensor order and spatial domain dimension from the
tensor along with any symmetries and constraints. The field
discretizes the tensor component values over the domain with
distributions and degrees of freedom (DOFs). The distributions are
defined over the mesh entities (and temporal discretization entities
as needed) and give the variation of the components of the
field. Thus, they must have the same functional domain that the
components of the tensor have.  The DOFs multiply the distributions and
set the magnitude of the variation of the individual distributions.

A complex simulation process can involve a number of fields defined
over various portions of the domain of the simulation. A single field
can be used by a number of different analysis routines that interact,
and the field may be associated with multiple meshes and have
a different relationship with each one.
In addition, different
distributions can be used by a field to discretize its associated
tensor. The ability to have a specific tensor defined over multiple
meshes and/or discretized in terms of multiple distributions is
handled by supporting multiple instances. A field instance has a
single set of distributions over a given mesh. These distributions are
defined over mesh entities which are of same dimension as the tensor
it is discretizing. A field instance can exist in an evaluated form
where the DOF have been determined, or in an unevaluated form where
the DOF are not yet determined.

ITAPS is currently defining interoperable field functions to:
\begin{itemize}
\item construct/load/save a field over a mesh,
\item interrogate the field at specific points and over mesh entities,
\item transform a field from one coordinate system to another,
\item project a field to a different set of basis functions (e.g., projecting a discontinuous 
stress field onto a set of continuous shape functions), and
\item transfer fields between different meshes including the use of different distributions. 
\end{itemize}

%% NOTE - IN THE BIBTEX FILE THIS REFERENCE MEEDS TO HAVE THE INITIALS 
%% P. P. ADDED TO THE 3RD AUTHOR.

%% @book{BeSo83,
%%  author="Beju, I. and Soos, E. and Teodorescu, P.P.", 
%%  title="Euclidean Tensor Calculus with Applications", 
%%  publisher="Abacus Press",
%%  year=1983






\section{ITAPS interface use cases}\label{sec:apps}

The ITAPS data model and interfaces have been defined and
implementations are underway at many different institutions.
In particular, the mesh interface is the most mature and in this
section we give several examples of its use in adaptive loop
construction for two different applications and in mesh quality improvement
tools.  The appendix provides code for two elementary examples illustrating 
some simple uses of the mesh interface.

\subsection{Adaptive Loop Construction}

Although mesh-based PDE 
codes are capable of providing results to the required levels of
accuracy, the vast majority lack the ability to automatically control
the mesh discretization errors through the application of adaptive
methods \cite{AiOd00,BaSt01,BaRa03}, thus leaving it to the user to
attempt to define an appropriate mesh.

One approach to support the application of adaptive analysis is to
alter the analysis code to include the error estimation and mesh
adaptation methods needed. The advantage of this approach is that the
resulting code can minimize the total computation and data
manipulation time required. The disadvantage is the amount of code
modification and development required to support mesh adaptation
is extensive since it requires extending the data structures
and all the procedures that interact with them. The expense and time
required to do this to existing fixed mesh codes is large and in most
cases considered prohibitive.

The alternative approach is to leave the fixed mesh analysis code
unaltered and to use the interoperable mesh, geometry and field
components to control the flow of information between the analysis
code and a set of other needed components. This approach has been used
to develop multiple adaptive analysis capabilities in which the
interoperable mesh, geometry and field components are used as follows:

\begin{itemize}
\item The geometry interface supports the integration with multiple
CAD systems. The interoperable API of the modeler enables interactions
with mesh generation and mesh modification to obtain all domain
geometry information needed \cite{BeWa04}.

\item The mesh interface provides the services for storing and
modifying mesh data during the adaptive process. The
Algorithm-Oriented Mesh Database \cite{ReSh03} was used for the examples given
here.

\item The field interface \cite{BeSh99} provides the functions to obtain
the solution information needed for error estimation and to support
the transfer of solution fields as the mesh is adapted.
\end{itemize}

One approach to support mesh adaptation is to use error estimators
to define a new mesh size field that is provided to an
automatic mesh generator that creates an entirely new mesh
of the domain. Although a popular approach, it has two
disadvantages. The first is the computational cost of an entire mesh
generation each time the mesh is adapted. The second is that in the
case of transient and/or non-linear problems, it requires global
solution field transfer between the old and new meshes. Such solution
transfer is not only computationally expensive, it can introduce
additional error into the solution which can dictate the ability of
the procedure to effectively obtain the level of solution accuracy
desired. An alternative approach to mesh adaptation is to apply local
mesh modifications \cite{LiSh05} that can range from standard templates, to
combinations of mesh modifications, to localized remeshing. Such
procedures have been developed that ensure the mesh's approximation to
the geometry is maintained as the mesh is modified \cite{LiSh03}. This is the
approach used to adaptive the mesh in the examples presented here.

\subsubsection{Adaptive Loop for Accelerator Design }

SLAC's eigenmode solver Omega3P, which is used in the design of next
generation linear accelerators, has been integrated with adaptive mesh
control \cite{GeLe04} to improve the accuracy and convergence of wall
loss (or quality factor) calculations in accelerator cavities. The
simulation procedure consists of interfacing Omega3P to solid models,
automatic mesh generation, general mesh modification, and error
estimator components to form an adaptive loop. The accelerator
geometries are defined as ACIS solid models \cite{spatial}. Using
functional interfaces between the geometric model and meshing
techniques, the automatic mesh generator MeshSim \cite{simmetrix}
creates the initial mesh. After Omega3P calculates the solution
fields, the error indicator determines a new mesh size field, and the
mesh modification procedures \cite{LiSh05} adapt the mesh.

The adaptive procedure has been applied to a Trispal 4-petal
accelerator cavity. Figure 1 shows the mesh and wall loss distribution
on the cavity surface for initial, first and final adaptive
meshes. The procedure has been shown to reliably produce results of
the desired accuracy for approximately one-third the number of
unknowns the previous user controlled procedure produced \cite{GeLe04}.



Figure 1. Adaptive analysis of a Trispal 4-petal accelerator cavity.


\subsubsection{Metal Forming Simulation}

In 3D metal forming simulations the workpiece undergo large
plastic deformations that result in major changes in the
domain geometry. The meshes of the deforming parts typically need to
be frequently modified to continue the analysis due to large element
distortions, mesh discretization errors and/or geometric approximation
errors. In these cases, it is necessary to replace the deformed mesh
with an improved mesh that is consistent with the current geometry.
Procedures to determine a new mesh size field
considering each of these factors has been developed and used in
conjunction with local mesh modification \cite{WaSh05}. The procedure includes
functions to transfer history dependent field variables as each mesh
modification is performed \cite{WaSh05}.

Figure 2 shows the set-up, initial mesh and final adapted meshes for a
steering link manufacturing problem solved using the DEFORM-3D
analysis engine \cite{Fl04} within a mesh modification-based adaptive
loop. A total stroke of 41.7mm is taken in the simulation. The initial
workpiece mesh consists 28,885 elements. The simulation is completed
with 20 mesh modification steps producing a final mesh with 102,249
elements.



Figure 2. Metal forming example.


\subsection{Mesh Quality Improvement}\label{sec:quality-improvement}

Mesh quality improvement techniques can be applied based on {\it a
priori} geometric quality metrics or {\it a posteriori} solution-based
metrics improvements.  Low-level mesh improvement operations include
vertex relocation, topology modification, vertex insertion, and vertex
deletion.   

The ITAPS center is supporting the development of a stand-alone mesh
quality improvement toolkit, called Mesquite~\cite{Mesquite03}.
Mesquite currently provides state-of-the-art algorithms for vertex
relocation and is flexible enough to work on a
wide array of mesh types ranging from structured meshes to unstructured
and hybrid meshes and a number of different two-and three-dimensional
element types.

% Question for Lori:  Why does Mesquite need to know the number of
% elements of a given type or topology?

Vertex relocation schemes must operate on the surface of the geometric
domain as well as in the interior of the domain to fully optimize the
mesh.  As such, the software must have functional access to both the
high level description of the geometric domain and to individual mesh
entities such as element vertices.  In particular, to operate on
interior vertices, Mesquite queries an ITAPS implementation for vertex
coordinate information, adjacency information, and the number of
elements of a given type or topology.  After determining the optimal
location for a vertex, Mesquite requests that the ITAPS implementation
update vertex coordinate information.  To operate on the surface mesh,
Mesquite must also use ITAPS geometric queries to determine the surface
normal and the closest point on the surface.  Explicit classification
of the mesh vertex against a geometric surface is required, as there
are some cases for which the closest point query will return a point
on the wrong surface, resulting in inverted or invalid meshes.

The ITAPS center is also supporting the development of a simplicial mesh
topology modification tool, which performs face and edge swapping
operations.\cite{TSTT-swap-tool}  This tool has been implemented using
the ITAPS mesh interface, enabling swapping in any ITAPS implementation
supporting triangles (2D) or tetrahedra (3D).

In gathering enough information to determine whether a swap is
desirable, any mesh topology modification scheme must make extensive use
of the ITAPS entity adjacency and vertex coordinate retrieval functions.
Reconfiguring the mesh, when this is appropriate, requires deletion of
old entities and creation of new entities through the ITAPS interface.
In addition, classification operators are again essential.  For
instance, reconfiguring tetrahedra that are classified on different
geometric regions results in tetrahedra that are not classified on
either region, so this case must be avoided.  Likewise, classification
checks make it easy to identify and disallow mesh reconfigurations that
would remove a mesh edge classified on a geometric edge.

In addition to basic geometry, topology and classification information,
a ITAPS implementation must provide additional information for mesh
improvement schemes to operate effectively and efficiently.  For
example, even for simple mesh improvement schemes, the implementation
must be able to indicate which entities may be modified and which may
not.  For mesh improvement schemes to operate on an entire mesh rather
than simply accepting requests entity by entity, an ITAPS implementation
must support some form of iterator.  Furthermore, advanced schemes may
allow the user to input a desired size, orientation, degree of
anisotropy, or even an initial reference mesh; exploiting such features
will require the implementation to associate many different types of
information with mesh entities and pass that information to the mesh
improvement scheme when requested.


\section{Concluding Remarks}
A simulation's information flow provides a conceptual framework for
designing interoperable tools for geometry management, mesh generation
and discretization.  Using this framework, the Interoperable Tools for
Advanced Petascale Simulation center has developed a set of language
independent interfaces to geometry, mesh, and solution field
information.  Several groups have successfully implemented the ITAPS
mesh interface with a diverse range of technologies ranging from
structured composite grids to fully unstructured infrastructures.
These implementations have provided mesh services for ITAPS-based
interoperable components providing technologies such as mesh
adaptation, design optimization and mesh improvement.  In principle,
any implementation of the the ITAPS mesh interface now has access to
these advanced technologies without requiring new source code
development.

While the utility of these interfaces has been demonstrated in a
number of applications, the current work remains a proof of principle.
Advancing the interface definition to a reliable standard requires
further investigation and demonstration.  For example, the nascent
solution field and operator interfaces have yet to be completed and
implemented.  In addition, more implementations of the interfaces need
to be created and exercised to make extensive interoperability and
interchangeability a possibility.  Furthermore, the performance
ramifications of using these interfaces must be carefully examined in
order to assure that applications built upon this infrastructure are
not plagued by low performance.  Collaborations with the Common
Component Architecture group~\cite{cca-forum} will lead to the
development of higher level components that advanced services to
simulation code developers.

More information on the ITAPS center can be found at
http://www.itaps-scidac.org/.

%\section*{Acknowledgments} 
% JOURNAL STUFF
\begin{acknowledgement}

We would like to thank Tamara Dahlgren of LLNL for her insightful
comments on the ITAPS interface design.  

This work was performed under the auspices of the U.S. Department of
Energy by the University of California Lawrence Livermore National
Laboratory under contract No. W-7405-Eng-48 (UCRL-JRNL-213577); 
the Canadian Natural Sciences and Engineering Research 
Council under Special Research Opportunities Grant SRO-299160;
and by Rensselaer Polytechnic Institute under DOE grant 
number DE-FC02-01ER25460. 

%WE NEED auspices for SANDIA, STONYBROOK, RPI , and PNNL

% JOURNAL STUFF
\end{acknowledgement}

\bibliographystyle{plain}
\bibliography{../../biblio/tstt.bib}

\appendix
\section[Usage Examples]{Elementary Examples of ITAPS Mesh Interface
  Usage}\label{sec:usage-examples}

This appendix presents very simple examples illustrating usage of the
ITAPS mesh interface.  These examples are meant to be illustrative
rather than exhaustive; much of the functionality of the mesh interface
is not showcased here.  The examples are written as stand-alone programs
that can be compiled and run with any ITAPS-compliant mesh database.

We note that the interface examples described here were developed
during the first round of SciDAC funding under the predecessor of the
ITAPS center, the Terascale Simulation Tools and Technologies (TSTT)
center.  With the advent of the SciDAC-2 program, the center was
renamed to ITAPS, but the team, philosophy and interface definition
efforts remain largely the same.  In the examples given here,
each interface is in the ITAPS namespace to avoid potential function
definition collisions.  The ``base'' functionality described in
Section 3.1, which includes tags, sets, and error handling is in the
iBase interface; the mesh functionality described in Section 3.2 is in
the iMesh interface.  

Full SIDL descriptions of the interfaces are available at
http://www.itaps-scidac.org/ under the Software link.  For those interested
in providing feedback on the interface definitions or participating in the
interface definition activity, please contact the ITAPS management team
at itaps-mgmt@lists.llnl.gov.

\subsection{Language Interoperability}

The ITAPS interface is designed to be not only data-structure neutral,
but also programming language neutral. That is, a mesh server can be
written in one language and client code in another. The ITAPS interface
is specified using an interface description language (SIDL), and
translated into language-specific interfaces through a tool called
Babel.\cite{babel,babel:site04} Babel also generates glue code that
mediates all inter-language issues, including function name mangling and
passage of string and array arguments.  As an example of how this works
in practice, consider the case of a request for mesh adjacency
information. An application code using the ITAPS interface makes an
adjacency request by calling a \emph{stub} function (auto-generated by
Babel) in the language of the application.  This function re-packages
function arguments and calls an \emph{internal object representation}
function (auto-generated by Babel, in C), which again repackages
arguments and calls a \emph{skeleton} function (auto-generated by Babel)
in the language of the server. This function, finally, calls the server
implementation of the original SIDL function. This approach eliminates
all language-specific issues, including name mangling schemes and the
treatment of strings and arrays, including dynamic array handling.  In
exchange, four versions of each SIDL function exist (three of which are
auto-generated), and a call from client code must pass through all these
layers. Not surprisingly, this complexity in call sequences can have a
significant impact on application efficiency.

As an example of the function signatures that Babel creates in various
languages, let us examine the mesh interface function for retrieving the
entities adjacent to a single entity.  The SIDL declaration for this
function is

\begin{verbatim}
package iMesh{
...
void getEntAdj(in opaque entity_handle,
               in EntityType entity_type_requested,
               inout array<opaque> adj_entity_handles,
               out int adj_entity_handles_size) throws iBase.Error;
}
\end{verbatim}

Clients call this function in different ways depending on the language
in which the client is written.  The C++ binding most nearly
duplicates the SIDL function declaration;
 
\begin{verbatim}
void iMesh::getEntAdj(void* entity_handle,
                      ::iMesh::EntityType entity_type_requested,
                      ::sidl::array<void*>& adj_entity_handles,
                      int32_t& adj_entity_handles_size)
    throw (::iBase::Error);
\end{verbatim}

In the C binding, the function name has been decorated to prevent naming
clashes between SIDL interfaces, and two arguments have been added.  One
of these ({\tt self}) is a handle for the iMesh data and the other ({\tt
\_ex}) is used to return exceptions. 
\begin{verbatim}
void iMesh_Entity_getEntAdj(iMesh_Entity self,
                            void* entity_handle,
                            enum iMesh_EntityType__enum entity_type_requested,
                            struct sidl_opaque__array** adj_entity_handles,
                            int32_t* adj_entity_handles_size,
                            sidl_BaseInterface *_ex);
\end{verbatim}

In Fortran77, all arguments are passed by address, and SIDL uses 64-bit
integers when passing handles.  Like the C binding, arguments have been
added for the iMesh data and exception return.
\begin{verbatim}
      subroutine iMesh_Entity_getEntAdj_f(self, entity_handle,
     &     entity_type_requested, adj_entity_handles,
     &     adj_entity_handles_size, exception)
      integer*8 self, entity_handle
      integer*4 entity_type_requested
      integer*8 adj_entity_handles
      integer*4 adj_entity_handles_size
      integer*8 exception
\end{verbatim}

Finally, the Fortran90 API is organized into modules and takes advantage
of user-defined types, in a manner quite similar to the C API.
\begin{verbatim}
recursive subroutine getEntAdj_s(self, entity_handle, entity_type_requested, &
    adj_entity_handles, adj_entity_handles_size, exception)
  implicit none
  type(iMesh_Entity_t) , intent(in) :: self
  integer (selected_int_kind(18)) , intent(in) :: entity_handle
  integer (selected_int_kind(9)) , intent(in) :: entity_type_requested
  type(sidl_opaque_1d) , intent(inout) :: adj_entity_handles
  integer (selected_int_kind(9)) , intent(out) :: adj_entity_handles_size
  type(sidl_BaseInterface_t) , intent(out) :: exception
\end{verbatim}


\subsection{Mesh Adjacency Example}

This example shows two ways in which entity adjacencies can be retrieved
using the ITAPS iMesh interface.  This example is written in C++; because the
ITAPS team uses Babel for interlanguage calls, the underlying 
implementation could be in any Babel-supported language.  

\begin{algorithm}
\begin{verbatim}
 1 #include <iostream>
 2 #include "iMesh.hh"
 3
 4 typedef void* EntityHandle;
 5 typedef void* EntitySetHandle;
 6 typedef void* IteratorHandle;
 7 int main( int argc, char *argv[] )
 8 {
 9   iMesh::Mesh mesh = iMesh::Factory::newMesh("");
10   std::string filename = argv[1];
11   EntitySetHandle rootSet = mesh.getRootSet();
12   mesh.load(rootSet, filename);
13
14   int vert_uses = 0;   // Iterate to access adjacencies
15   iMesh::Entity mesh_ent = mesh;
16   IteratorHandle iter;
17   mesh_ent.initEntIter(rootSet, iMesh::EntityType_REGION, 
18                        iMesh::EntityTopology_ALL_TOPOLOGIES, iter);
19   EntityHandle ent;
20   bool moreData = mesh_ent.getNextEntIter(iter, ent);
21   while (moreData) {
22     sidl::array<EntityHandle> verts;
23     int verts_size;
24     mesh_ent.getEntAdj(ent, iMesh::EntityType_VERTEX,
25                        verts, verts_size);
26     vert_uses += verts_size;
27     moreData = mesh_ent.getNextEntIter(iter, ent);
28   }
29
30   sidl::array<EntityHandle> ents;   // Block Retrieval
31   int ents_size;
32   mesh.getEntities(rootSet, iMesh::EntityType_REGION, 
33                    iMesh::EntityTopology_ALL_TOPOLOGIES,
34                    ents, ents_size);
35   sidl::array<EntityHandle> allverts;
36   sidl::array<int> offsets;
37   int allverts_size, offsets_size;
38   iMesh::Arr mesh_arr = mesh;
39   mesh_arr.getEntArrAdj(ents, ents_size, iMesh::EntityType_VERTEX,
40                         allverts, allverts_size,
41                         offsets, offsets_size);
42   std::cout << "Sizes did ";
43   if (allverts_size != vert_uses) std::cout << "not";
44   std::cout << " agree!" << std::endl;
45   return true;
46 }
\end{verbatim}
\caption{Example of adjacency retrieval using the ITAPS mesh interface.}\label{ex:adjacency}
\end{algorithm}

In line 9, a new mesh instance is created, using a factory.
This factory is implementation-specific, but its {\em interface} is not,
freeing an application from any compile-time dependence on a single
implementation.  The ITAPS implementation is supplied at link time or,
with dynamically-loaded libraries, at run time.  In lines 10--12, mesh
data is read from a file into the root set of the mesh.

Lines 14--28 iterate through all the three-dimensional entities
(regions) of the mesh, counting their total number of vertices.  The
iteration is controlled by an entity-by-entity iterator, initialized in
line 17.  Note that this iterator is not defined as part of the
iMesh::Mesh base interface but in a more specialized interface,
iMesh::Entity; line 15 casts the Mesh object to Entity.\footnote{While
C++ could handle the relationships among interfaces using inheritance,
not all languages can, so Babel does not use this idiom in C++ either.}
In line 19, the iterator provides both a boolean value indicating
whether more data is available and the handle of the next available
entity if there is one.  This syntax is admittedly somewhat awkward, but
if a mesh is modified, it is impossible in general to be certain whether
there will be another entity until one tries to retrieve the next one.
Line 24 is the heart of the adjacency retrieval loop, returning an array
of all vertices adjacent to the current region in the iteration.

Lines 30--41 illustrate block retrieval of entity adjacency information.
Line 32 first retrieves all regions in the mesh.  Then, in line 39, all
vertices adjacent to the entities whose handles are in {\tt ents} (i.e.,
all regions) are returned; the contents of {\tt offsets} identifies, for
each {\tt ent}, where its list of vertices begins in {\tt allverts}.

Finally, lines 42--44 report whether the total numbers of adjacent
vertices retrieved by these alternate approaches are consistent.

\subsection{Set and Tag Example}

This example shows simple retrieval of entity sets and identification of
tags attached to those sets.  Again, the underlying ITAPS
implementation could be in any Babel-supported language.  

\begin{algorithm}
\begin{verbatim}
 1 #include <iostream>
 2 #include <set>
 3 #include "iMesh.hh"
 4 #include "iBase.hh"
 5 
 6 typedef void* EntityHandle;
 7 typedef void* EntitySetHandle;
 8 typedef void* TagHandle;
 9 
10 int main( int argc, char *argv[] )
11 {
12   std::string filename = argv[1];
13   iMesh::Mesh mesh = iMesh::Factory::newMesh("");
14   EntitySetHandle rootSet = mesh.getRootSet();
15   mesh.load(rootSet, filename);
16 
17   sidl::array<EntitySetHandle> sets;
18   int sets_size;
19   iBase::EntSet mesh_eset = mesh;
20   mesh_eset.getEntSets(rootSet, 1, sets, sets_size);
21 
22   iBase::SetTag mesh_stag = mesh;  //Retrieve set tag info
23   std::set<TagHandle> tag_handles;
24   for (int i = 0; i < sets_size; i++) {
25     sidl::array<TagHandle> tags;
26     int tags_size;
27     mesh_stag.getAllEntSetTags(sets[i], tags, tags_size);
28     for (int j = 0; j < tags_size; j++) {
29       tag_handles.insert(tags[j]);
30     }
31   }
32
33   for (std::set<TagHandle>::iterator sit = tag_handles.begin(); 
34        sit != tag_handles.end(); sit++) {
35     std::string tag_name = mesh_stag.getTagName(*sit);
36     int tag_size = mesh_stag.getTagSizeBytes(*sit);
37     std::cout << "Tag name = '" << tag_name 
38               << "', size = " << tag_size << " bytes." << std::endl;
39   }
40   return true;
41 }
\end{verbatim}
\caption{Example of entity set and tag retrieval using the ITAPS mesh interface.}\label{ex:sets-tags}
\end{algorithm}
 
After reading a mesh as in the previous example, all the entity sets
defined for the mesh are retrieved (line 20).

Lines 22--31 retrieve tag information for the sets.  Specifically, line
27 retrieves all tags attached to a particular entity set, and the loop
from lines 28--30 populates a standard template library {\tt set} of tag
handles.

Finally, the loop from lines 33--39 output information about each tag
found, in order of increasing tag handle.  For each tag handle, the name
of the tag (retrieved in line 35) and its size in bytes (retrieved in
line 36) are output.



\end{document} 
