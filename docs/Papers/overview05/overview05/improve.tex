\subsection{Mesh Quality Improvement}\label{sec:quality-improvement}

Mesh quality improvement techniques can be applied based on {\it a
priori} geometric quality metrics or {\it a posteriori} solution-based
metrics improvements.  Low-level mesh improvement operations include
vertex relocation, topology modification, vertex insertion, and vertex
deletion.   

The ITAPS center is supporting the development of a stand-alone mesh
quality improvement toolkit, called Mesquite~\cite{Mesquite03}.
Mesquite currently provides state-of-the-art algorithms for vertex
relocation and is flexible enough to work on a
wide array of mesh types ranging from structured meshes to unstructured
and hybrid meshes and a number of different two-and three-dimensional
element types.

% Question for Lori:  Why does Mesquite need to know the number of
% elements of a given type or topology?

Vertex relocation schemes must operate on the surface of the geometric
domain as well as in the interior of the domain to fully optimize the
mesh.  As such, the software must have functional access to both the
high level description of the geometric domain and to individual mesh
entities such as element vertices.  In particular, to operate on
interior vertices, Mesquite queries an ITAPS implementation for vertex
coordinate information, adjacency information, and the number of
elements of a given type or topology.  After determining the optimal
location for a vertex, Mesquite requests that the ITAPS implementation
update vertex coordinate information.  To operate on the surface mesh,
Mesquite must also use ITAPS geometric queries to determine the surface
normal and the closest point on the surface.  Explicit classification
of the mesh vertex against a geometric surface is required, as there
are some cases for which the closest point query will return a point
on the wrong surface, resulting in inverted or invalid meshes.

The ITAPS center is also supporting the development of a simplicial mesh
topology modification tool, which performs face and edge swapping
operations.\cite{TSTT-swap-tool}  This tool has been implemented using
the ITAPS mesh interface, enabling swapping in any ITAPS implementation
supporting triangles (2D) or tetrahedra (3D).

In gathering enough information to determine whether a swap is
desirable, any mesh topology modification scheme must make extensive use
of the ITAPS entity adjacency and vertex coordinate retrieval functions.
Reconfiguring the mesh, when this is appropriate, requires deletion of
old entities and creation of new entities through the ITAPS interface.
In addition, classification operators are again essential.  For
instance, reconfiguring tetrahedra that are classified on different
geometric regions results in tetrahedra that are not classified on
either region, so this case must be avoided.  Likewise, classification
checks make it easy to identify and disallow mesh reconfigurations that
would remove a mesh edge classified on a geometric edge.

In addition to basic geometry, topology and classification information,
a ITAPS implementation must provide additional information for mesh
improvement schemes to operate effectively and efficiently.  For
example, even for simple mesh improvement schemes, the implementation
must be able to indicate which entities may be modified and which may
not.  For mesh improvement schemes to operate on an entire mesh rather
than simply accepting requests entity by entity, an ITAPS implementation
must support some form of iterator.  Furthermore, advanced schemes may
allow the user to input a desired size, orientation, degree of
anisotropy, or even an initial reference mesh; exploiting such features
will require the implementation to associate many different types of
information with mesh entities and pass that information to the mesh
improvement scheme when requested.
