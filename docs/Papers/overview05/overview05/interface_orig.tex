To support the flow of information in mesh-based simulations a number
of tools and technologies have been developed by different research
groups in academia, industry and the government labs.  For these tools
to have maximum impact it is important that they be interoperable,
interchangeable, and easily inserted into existing application
simulation codes.  Accomplishing this goal will allow easier
experimentation with different, but functionally similar, technologies
to determine which is best suited for a given application.  In
addition, it will provide mechanisms for combining technologies
together to create hybrid solution techniques that use multiple
advanced tools.

To accomplish this goal, we have defined an abstract data model that
encompasses a broad spectrum of mesh types and usage scenarios {\it
and} a set of common interfaces that are implementation and data
structure neutral.  The set of interfaces must be both small enough to
encourage adoption but also flexible enough to support a broad range
of mesh types.  {\it LAF should we move this figure and discussion to
section 2?} Figure \ref{fig:hieracrhy} shows the hierarchical
relationship between the geometric description of the computational
domain and first discretization step in PDE-based simulation.  At the
highest level is the geometric description of the computational
domain; in many cases this domain description will be a CAD boundary
representation.  This domain must be decomposed into a set of mesh
entities. This process may employ a hierarchical decomposition of the
geometric domain. For example, the first level may decompose the
domain into a set of sub-domains that can be meshed using various
meshing strategies.  In particular, hybrid meshes consisting of
different component meshes can be used to discretize different
portions of the geometric domain, or different full geometry meshes
can be used during different stages of the numerical solution.  Each
of these meshes is associated with the full geometry domain so that
any changes made there propogate properly to the associated meshes.
Each mesh can be further decomposed into partitions for solution on a
massively parallel computer.

The TSTT data model abstracts this simulation data hierachy and is
decomposed into three {\it core data types}: the geometric data, the
mesh data, and the field data.  These core data types are associated
with each other through {\it data relation managers}. The data
relation managers control the relationships among two or more of the
core data types, resolve cross references between entities in
different groups, and can provide additional functionality that
depends on multiple core data types.  Work on the mesh data model and
API has progressed the farthest and we describe it some detail in
Section \ref{sec:mesh}.  Preliminary work on the geometry and field
data model and interfaces are discussed as well.

A key aspect of this approach is that we do not enforce any particular
data structure or implementation with our interfaces, only that
certain questions about the mesh, geometry or field data can be
answered through calls to the interface.  To encourage adoption of the
interface we aim to create a small set of interfaces that existing
mesh and geometry packages can support.  The latter point is critical.
The DOE, NSF, DoD and other federal agencies have invested hundreds of
person-years in the development of a wide variety of geometry, mesh
generation and mesh management toolkits.  These software packages will
not be rewritten from scratch to conform to a common API, rather the
API must be data structure neutral and allow for a broad range of
underlying mesh representations. However, only a small set of
functionalities can be covered by a 'core' set of interface functions.
To increase the functionality of the TSTT interface, we define
additional, optional, interfaces for which we will provide reference
implementations based on the core interface methods.  Developers can
implement these functions on their own mesh database as needed.
The allows for incremental adoption of the interface.

One of the foremost challenges inherent in this type of effort include
balancing performance of the interface with the flexibility needed to
support a wide variety of mesh types.  Performance is critical for
kernel computations involving mesh access.  To address this need we
provide a number of different access patterns including array and
iterator-based.  The user may choose the access pattern that is best
suited for their application; the underlying implementation must
provide both styles of access even though only one is likely to be
native.  Further challenges arise when considering the support of many
different scientific programming languages.  This aspect is addressed
through our joint work with the Center for Component Technologies for
Terascale Simulation Science (CCTTSS) \cite{cca-forum} to provide
language independent interfaces by using their SIDL/Babel technology
\cite{babel}.  Preliminary results for the use of SIDL/Babel with
the TSTT mesh interface are given in Section \ref{sec:mesh_perf}.

\subsection{The TSTT Mesh Interface}

\subsubsection{The Mesh Data Model}
\label{sec:mesh}

The TSTT mesh data model is composed of two different types of
entities: mesh entities and entity sets.  To allow the interface to
remain data structure neutral, these entities are uniquely represented
by 32-bit opaque handle which may or may not be invariant through
different calls to the interface in the lifetime of the TSTT mesh.

{\bf Mesh Entity Definition}: TSTT mesh entities are the core of the
TSTT mesh interface and are defined by their entity type and entity
topology.  Allowable entity types are VERTEX (0D), EDGE (1D), FACE
(2D), and REGION (3D).  Allowable entity topologies are listed
WHERE??; each of these topologies has a unique entity type associated
with it.  Higher-dimensional entities are defined by lower-dimensional
entities using canonical ordering relationships.  Mesh geometry shape
information can be associated with the individual mesh entities. For
example, the vertices will have coordinates associated with
them. Higher order mesh entities can also have shape information
associated with them. For example the coordinates of higher-order
nodes can be associated with mesh edges, faces and regions.

%Vertices can return
%coordinate information in blocked or interleaved fashion.

Entity adjacency relationships define how the entities connect to
each other and both first-order and second-order adjacencies are
supported.
\begin{itemize}
\item {\it First-order adjacencies}: For an entity of dimension $d$,
first-order adjacencies return all of the mesh entities of dimension
$q$, which are either on the closure of the entity ($d > q$, downward
adjacency), or which it is on the closure of ($d < q$, upward
adjacency).  %If available, first-order adjacencies can be obtainable
%by either stored adjacencies, local traversal of stored adjacencies of
%an entity's neighborhood, or global mesh level traversal.  

\item {\it Second-order adjacencies}: There are times when applications 
would want not just what bounds an entity or the what an entity
bounds, but the next level of neighbors. Although such information can
always be determined from the appropriate first order adjacencies,
their application is common enough that supporting a second order
adjacency function is useful. A second order adjacency determines the
set of topological entities of a given order adjacent to entities that
that share common boundary entities of the specified order. An example
would be to determine the set of model regions that share a bounding
edge with the given region.

%For an entity of dimension $d$,
%second-order adjacencies describe all of the mesh entities of
%dimension $q$ that share any adjacent entities of dimension $b$, where
%$d \neq b$ and $b \neq q$.  Second-order adjacencies can be derived
%from first-order adjacencies.  Examples include for a given face, a
%set of regions adjacent to the face (first-order upward), a set of
%vertices bounding the face (first-order downward), a set of faces that
%share any vertex of the face (second-order).

\end{itemize}

To determine which adjacencies are supported by an underlying
implementation, an adjacency table is defined which can be
returned by a query through the interface.  Adjacencies are defined be
either immediately available, available through a local traversal,
available through a global traversal, or not available.  If adjacency
information exists, entities must be able to return both upward and
downward adjacency information in the canonical ordering using both
individual and agglomerated request mechanisms.

%%%% ***** MS Comment - To this points the sets are defined as discrete entity sets (as approsed to spaces, etc.) with some specific rules. These sets support specific standard operations. This is very useful and of critical importance to our ability to support applications.  However, I still have some indigestion the things we say sets can do and they "theory" aspects. We support basic Boolean operations (with some care needed) on sets of discrete entities - using the term "set theoretic sense" implies more than that - we should be specific. The use of the sets for most all the examples defined requires a fair amount of additional algorithm development and specific definitions that are not captured directly by basic the set operations. That's fine since we want a general tool that can support lots of things we have not considered and/or do not want to formalize. (There are some we may want to formalize that the current sets do poorly - but we can consider that when (and only when) we decide we want them. On the other hand, some stated examples go well beyond anything the sets come close to supporting - for example supporting the prolongation and restriction operators of multigrid methods with the mesh sets would actually be much harder than most of the normal mechanisms used for these processes. Based on this concern I have made some suggested edits to the text below to try to focus the discussion of sets on what I think we should tell people about them.

{\bf Entity Set Definition:} A TSTT entity set is an arbitrary
collection of TSTT entities that have uniquely defined entity handles.
Each EntitySet may be an unordered set or it
may a (possibly non-unique) ordered list of entities.  When the TSTT
mesh interface is first created in a simulation, a {\it Root Set} is
created and can be populated by string name using the load
functionality.  Example entity sets include a set of vertices, the set
of all faces classified on a geometric face, the set of regions in a
domain decomposition for parallel computing.%, the set of all entities
%in a given level of a multigrid mesh sequence.



Two primary relationships among EntitySets are supported:

\begin{itemize}
\item Entity sets may {\it contain} one or more entity sets.  An
entity set contained in another may be either a subset or an element
of that entity set.  The choice between these two interpretations is
left to the application; TSTT supports both interpretations. If entity
set A is contained in entity set B, a request for the contents of B
will include the entities in A and the entities in sets contained in A
if the application requests the contents recursively.  We note that
the {\it Root Set} cannot be contained in another entity set.

\item {\it Parent/child relationships} between entity sets are used to
represent relations between sets, much like edges connecting nodes in
a graph.  This relationship can be used to indicate that two meshes
have a logical relationship to each other, including multigrid and
adaptive mesh sequences. Because we distinguish between parent and
child links, this is a directed graph. Also, the meaning of cyclic
parent/child relationships is dubious, at best, so graphs must be
acyclic. No other assumptions are made about the graph.
\end{itemize}

Users are able to query entity sets for their entities and entity
adjacency relationships.  Both array- and iterator-based access
patterns are supported.  In addition, entity sets also have "set
operation" capabilities; in particular, you may add and remove
existing TSTT entities to the entity set and you may subtract,
intersect, or unite entity sets.  %In addition, subset and hierarchical
%parent/child relationships among EntitySets are supported.

We note that to be useful to computational simulations, entity sets
can comprise a valid computational mesh; the most simple example of
which is a nonoverlapping, connected set of TSTT entity regions, for
example, the structured and unstructured meshes commonly used in
finite element simulations.  Collections of entity sets can compose,
for example, overlapping and multiblock meshes. In both of these
examples, supplemental information on the interactions of the mesh sets
will be defined and maintained by the application.  Smooth particle
hydrodynamic (SPH) meshes can consist of a collection of TSTT vertices
with no connectivity or adjacency information.

%In addition, entity sets can also be extended to be "modifiable", in
%which case, basic operations that allow applications to change the
%geometry and topology are provided.  

The mesh interface, including the use of mesh sets, is extendable to
include ``modificiation operators'' that change the geometry and topology.
Capabilities include changing
vertex coordinates and adding or deleting entities. No validity checks
are provided with this basic interface so that care must be taken when
using these interfaces.  These interfaces are intended to support
higher-level functionality such as mesh quality improvement, adaptive
schemes, front tracking proceedures, and basic mesh generation
capabilities, all of which would provide validity checking.
Modifiable meshes require interactions with the underlying
geometric model including classifying entities.% and this interaction is described
%in Section \ref{sec:mesh_geom_class}.

Tags are used as containers for user-defined opaque data that can be
attached to TSTT entities and entity sets.  Tags can be
multi-valued which implies that a given tag handle can be associated
with many mesh entities.  In the general case, TSTT tags do not have a
predefined type and allow the user to attach any opaque data to mesh
entities.  To improve ease of use and performance, we support three
specialized tag types: integers, doubles, and Booleans.  Tags have and
can return their string name, size, handle and data (data retrieval is
done in the entity, mesh and entity set interfaces).  Tag data can be
retrieved from TSTT objects by handle in an agglomerated or individual
manner.  The implementation is expected to allocate the memory as
needed to store the tag data.

\subsubsection{Status of the mesh interface}

The TSTT mesh interface has been under development for approximately
two years and several revisions have been made in that time.  Several
implementations are well underway and are supported by mesh management
toolkits such as AOMD (RPI) \cite{ReSh03}, Overture (LLNL)
\cite{overture}, MOAB (SNL) \cite{moab}, NWGrid (PNL) \cite{nwgrid},
and GRUMMP (UBC) \cite{grummp}.  

In addition to the development of underlying implementations, the
TSTT mesh interface has also been used in a variety of contexts as
well.  In particular, it serves as the interface to the Mesquite 
mesh quality improvement and Frontier front tracking


\subsubsection{Preliminary performance results for the mesh interface}

\subsection{Geometry}




\subsection{The ITAPS Fields Interface}
\label{sec:fields}

Simulation fields represent tensor quantities defined in terms of
numerical analysis discretizations in a form useful to support queries
and operations by other functions or simulations. Common examples
where fields are used are (i) multiphysics analysis where the solution
fields from each physics analysis represents a forcing function or
boundary condition for another, (ii) the construction of external
adaptive control loops where the solution fields are used by error
estimation procedures to obtain estimates of the discretization errors
and to construct new mesh size field, and (iii) visualization and 
analysis/postprocessing.

Tensor quantities used in the quantification of problems of
mathematical physics are of order zero or greater and are defined over
a physical space or space/time domain.  Knowledge of the order of a
tensor and the dimension of the spatial domain over which it is defined,
gives the number of components needed to uniquely define the tensor
\cite{BeSo83}. The symmetries, for tensors of order 2 or greater,
define those components that are identical to, or the negative of
(antisymmetric), other components. The components of the tensor are,
in general, functions of the domain parameters as well as other
problem parameters. The ability to understand and use a tensor at any
particular instant requires knowledge of the coordinate system in
which the components of the tensor are referred.

The qualification of a tensor over a domain is called a field. The
field inherits the tensor order and spatial domain dimension from the
tensor along with any symmetries and constraints. The field
discretizes the tensor component values over the domain with
distributions and degrees of freedom (DOFs). The distributions are
defined over the mesh entities (and temporal discretization entities
as needed) and give the variation of the components of the
field. Thus, they must have the same functional domain that the
components of the tensor have.  The DOFs multiply the distributions and
set the magnitude of the variation of the individual distributions.

A complex simulation process can involve a number of fields defined
over various portions of the domain of the simulation. A single field
can be used by a number of different analysis routines that interact,
and the field may be associated with multiple meshes and have
a different relationship with each one.
In addition, different
distributions can be used by a field to discretize its associated
tensor. The ability to have a specific tensor defined over multiple
meshes and/or discretized in terms of multiple distributions is
handled by supporting multiple instances. A field instance has a
single set of distributions over a given mesh. These distributions are
defined over mesh entities which are of same dimension as the tensor
it is discretizing. A field instance can exist in an evaluated form
where the DOF have been determined, or in an unevaluated form where
the DOF are not yet determined.

ITAPS is currently defining interoperable field functions to:
\begin{itemize}
\item construct/load/save a field over a mesh,
\item interrogate the field at specific points and over mesh entities,
\item transform a field from one coordinate system to another,
\item project a field to a different set of basis functions (e.g., projecting a discontinuous 
stress field onto a set of continuous shape functions), and
\item transfer fields between different meshes including the use of different distributions. 
\end{itemize}

%% NOTE - IN THE BIBTEX FILE THIS REFERENCE MEEDS TO HAVE THE INITIALS 
%% P. P. ADDED TO THE 3RD AUTHOR.

%% @book{BeSo83,
%%  author="Beju, I. and Soos, E. and Teodorescu, P.P.", 
%%  title="Euclidean Tensor Calculus with Applications", 
%%  publisher="Abacus Press",
%%  year=1983




%% \subsection{Geometry Interface}
%% \begin{verbatim}
%%       - Data Model
%%       - Functionality
%%       - Status of interface/implementations
%% \end{verbatim}

%% \subsection{Field Interface}
