\documentclass[11pt]{article}
%\usepackage{imr}
%\usepackage{amssymb}
\usepackage{epsfig}

\def\thepage {}
%\bibliographystyle{imr}
%\newtheorem{Pseudo}{Pseudo-Code}[section]
\begin{document}

\title{\uppercase{Toward Interoperable Mesh, Geometry and Field Components for PDE Simulation Development}}
%\author{Lori Diachin$^1$ \and Mark Shephard$^2$}
%\thanks{
%$^1$ Center for Applied Scientific Computing, Lawrence Livermore National Laboratory, Livemore, CA 94551 \and
%$^2$ Scientific Computing Research Center, Rensellear Polytechnic Institute, Troy NY
%}
\date{Received: date / Revised version: date}

\maketitle
\begin{abstract}
abstract
\end{abstract}

\section{Introduction}
The development of effective simulation codes for solving problems in
mathematical physics using mesh-based techniques continues to require
significant effort.  Because the governing equations can be
non-linear, coupled and/or subject to continued refinement, a wide
variety of functions must be addressed for each new problem area.
These include, for example, mesh generation and adaptive mesh control,
equation discretization, the solution of large systems of linear or
non-linear equations, and, potentially, coupling results between
different codes. The execution of many of these steps can be supported
using existing technologies, but coupling technologies developed by
different groups can be quite challenging.  This paper overviews an
approach which addresses this challenge for the generation and control
of meshes as well as the interactions of simulation information
defined on those meshes through the development of interoperable
components.

A number of different approaches have been used to provide tools and
technologies to mesh-based simulation code developers. Depending on
the starting point and needs of the specific code development process,
each has been found to be useful. These approaches can be categorized
into the following four groups:

\begin{enumerate}
\item complete simulation codes that support the integration of specific
      user-defined modules,
\item simulation frameworks that support the overall development process,
\item libraries that support specific aspects of the simulation
      process, and
\item components that encapsulate specific functionalities.
\end{enumerate}

There are many examples of complete simulation codes that provide a
small set of predefined routines that allow users to add specific
capabilities. One such well known example is ABAQUS \cite{abaqus}
which supports a user-defined material routine and a user-defined
finite element routine which allows users to include their own
constitutive relationships and finite element type,
respectively. Although predefined user routine interfaces place
specific limits on the form that can be added, they are particularly
useful if the missing functionality needed is covered by that
routine. For example, the ABAQUS material routine has been
successfully used to integrate hundreds of new material models ranging
from simple curve fits through homogenized constitutive relations
constructed from multiscale analysis.  The key disadvantages of this
approach are that the new capabilities that can be added are quite
limited and they can only be added through a specific interface. More
recently programs such as FEMLAB \cite{femlab} have been introduced
that increase the number of points of entry for new capabilities and
provide a broader definition of the interface.  This brings it much
closer to the simulation framework approach.

Simulation frameworks provide an overall structure, building on
particular data formats and other services, that support the effective
development and extension of the framework to provide new capabilities
\cite{BeSh99,BrLa97,DoLa96,De97,StEd04}. Efforts on simulation
framework development have taken advantage of modern programming
languages to provide developers a high degree of flexibility. However,
the choices made on the interface methods, data and algorithmic
services vary substantially among the various frameworks. The
differences correspond to the trade-offs associated with the types and
levels of generality supported and the computational efficiency that
can be obtained.  The framework approach can be particularly effective
when an entirely new simulation code is being developed and one of the
available frameworks is a good match in terms of functionality
supported and level of computational efficiency possible. However, for
developers with an existing code that are focused on incorporating new
capabilities, the framework approach is not ideal because moving
existing capabilities into the framework will be time consuming at
best.

The use of numerical libraries to support the development of numerical
simulation codes has a long history. The area where numerical
libraries have been, and continue to be, most successful is for
execution of computationally intensive core numerical algorithms like
algebraic system solvers, ordinary differential equation and
differential-algebraic solvers (e.g.,
\cite{AsPe98,petsc,BaGr97,eispack,lapack,linpack}). These libraries
provide capabilities for computationally intensive operations that are
most efficiently executed by the careful selection and implementation
of specific numerical algorithms. Although quite successful in their
specific areas, numerical libraries do not support development of
other portions of the code, nor do they support integration with
higher level functionalities.

Recently, the use of component technologies for the development of
simulation codes has started to become more prevalent \cite{cca-app}.
A component is defined to be a software object meant to interact with
other components that encapsulates a specific functionality, employs a
clearly defined interface and conforms to a prescribed behavior common
to all components in an architecture. Typically, each interface is
supported by multiple implementations which allows code developers to
easily experiment with different approaches.  This approach is ideal
in the case where there is already a substantial investment in the
simulation engine and the developers are interested in incorporating
advanced functionality or experimenting with several different,
related approaches.  Several groups are developing component
implementations for different aspects of the numerical solution
process including numerical solvers \cite{petsc,BaGr97}), ODE
integrators, and visualization tools \cite{cca-paper}.  However, more
work is required to increase the number of tools and technologies that
use a component-based approach.

In this paper, we describe ongoing research in the Terascale
Simulation Tools and Technologies (TSTT) center to develop geometry,
mesh and solution field components.  The primary reason for focusing
on this set of components is that it provides access to a broad set of
funtionalities to support simulation automation, solution reliability
in terms of discretization error control,
and the ability to integrate tools to construct multi-physics
simulations.  Simulation automation is supported through the geometry
and mesh components because they provide ready access to CAD-based
geometry definitions and automatic mesh generators. Solution
reliability is supported because these are the components needed to
support the effective creation and adaptive control of meshes. The
fields and mesh components are key to the effective coupling of
multiple simulation codes in the construction of multiphysics
simulations.

Section 2 provides a high level view of the functions and information
flow associated with the execution of mesh-based simulations starting
from a generalized problem definition and indicates the role of the
geometry, mesh and fields components in supporting these processes
which defined in more detail in Section 3. Section 4 provides example
applications of the initial version of these procedures to support
design optimization, adaptive mesh control, mesh quality improvement,
and front tracking. ****** THIS PARAGRAPH WILL NEED UPDATING TO BE
CONSISTENT WITH THE FINAL VERSION AND APPLICATIONS INCLUDED ********



%\begin{verbatim}
%1. Introduction
% - Motivation: there is a significant effort devoted to the development
%   of PDE solvers
%    - often this effort is somewhat peripheral to the main goals at
%      hand; that is the development of a new physics solver
%    - could be more effective if tools for the peripheral parts were
%      available for incorporation
%    - such tools would include automated meshing from general problem 
%      definitions, adaptive methods, multi-physics techniques
%
% - Even when they exist, simulation codes find it difficult to take 
%   advantage of these tools because of the current status of tool availability:
%    - libraries:
%        - typically focused; limited API options for accessing library
%          functionality
%        - difficult to experiment with different libraries that provide
%          similar capabilities (API is library specific)
%        - often don't interoperate with other libraries that provide 
%          related capabilities
%        - language interoperability often sketchy
%        - examples include F,G
%    - frameworks:
%        - typically impose structure on application code; reasonable
%          for new application development, but often difficult to rewrite
%          existing codes to conform
%        - simulation codes need to buy into the entire framework to use
%          the parts of most interest to them
%        - examples include X,Y
%
% - New work in scientific community focuses on the development of components 
%   which addresses some of the short comings of library or framework approaches
%    - Define a component (a software object meant to interact with other
%       components, encapsulating a certain functionality, clearly defined
%       interface and comforms to a prescribed behavior common to all components
%       in an architecture)
%    - Key: define a common API, multiple implementations/services that conform
%    - Useful in the following cases
%        - where there is a specialized analysis engine that cannot be easily 
%          reworked into a framework code and that requires multiple, related 
%          functionalities (perhaps provided by incompatible libraries)
%        - where experimentation with different techniques is of interest
%          (plug and play)
%        - where dynamic access to different, related services is required
%    - Other examples of work in this area
%        - CCA, solvers, etc
%    - TSTT work focuses on mesh/geometry/field tools - the focus of this paper
%\end{verbatim}

\section{Information Flow in Mesh-based simulation}
In mesh-based analysis procedures the
PDEs are solved approximately through a double discretization process
in which the problem's physical domain is discretized into a set of
piecewise components (e.g. a mesh) and the PDEs to be solved are
discretized over the mesh in an appropriate manner. The
discretizations of the PDEs over the mesh are assembled into a fully
discrete system that is solved. The result of this
process yields the construction of a set of discretized solution
fields. Methods covered by such approaches include finite
difference, finite volume, finite element, boundary element and
partition of unity (so-called meshfree) methods.

\subsection{Problem Definition}

To qualify the operations and information involved in executing a
mesh-based simulation, we begin with a qualification of the problem
definition which includes:

\begin{itemize}
\item The domain over which the simulation is to be solved. For the
classes of simulations being considered here the domain includes a
spatial component that is one-, two-, or three-dimensional. The problem
can also be defined over time in which case the domain also includes a
temporal component.

\item The mathematical form governing the simulation (PDE�s,
variational principle, weak form).

\item Specification of the parameters, referred to as physical
attributes, associated with the governing mathematical equations that
includes:

\begin{itemize}
\item material properties
\item forcing functions
\item boundary conditions
\item initial conditions
\end{itemize}
\end{itemize}

From the viewpoint of supporting a numerical simulation, the
domain representation must be able to:

\begin{itemize}
\item Support the construction of a mesh that represents the domain
level discretization used in the simulation.

\item Support the ability to address any geometry interrogation required.

\item Support the proper association of the physical and mathematical
attributes with the mesh.

\item Support the domain evolution in the cases where the domain
changes as part of the solution process.
\end{itemize}

There are multiple sources for the high level definitions of the
spatial component of the domain with CAD models, image data and
cell-based (mesh-based) being the most common. Each of these sources
has one or more representational forms. Historically, CAD systems use
boundary representations. Image data use a volumetric
form such as voxels or octrees. Depending on the configuration of the
cells a variety of implicit and explicit boundary or volumetric
representations have been used.

Except in cases of image data and when all aspects of the simulation
process can be effectively defined in terms of volume entities, it
is generally accepted that the use of a boundary
representation is well suited for the spatial domain definition. There
is a substantial computer-aided design literature on the various
boundary representations. Common to these representations is
the use the abstraction of topological entities and their adjacencies
to represent the model entities of different dimensions. In a
boundary representation the information defining the shape of
the topological entities can be thought of as attribute information
associated with the appropriate entities. The ability to interact with
topological entities provides an
effective means to develop abstract interfaces
allowing the easy integration to multiple domain definition
sources.

An important consideration in the selection of a boundary
representation is its ability to represent the classes of domain
needed. In the case of numerical simulations the domains to be meshed
can be general combinations of 0-, 1-, 2- and 3-D entities in general
configurations. Figure 1 shows a typical analysis domain that may be
used for the structural analysis of a portion of a piping
system. The analysis domain is an idealization
where portions of the pipes are idealized by beams,
the support bracket idealized by a plate
and a full 3-D solid used for the pipe juncture.



Figure 1. Example of a non-manifold model used in simulation.


The proper representation of such geometric domains, as
well as others like multi-material domains, are
referred to as a non-manifold boundary representations \cite{GuCh90,We88}. In
the case of non-manifold models the representation must indicate
how topological entities are used by bounding higher order
entities. For example, each side of a face may be used by a different
region. Therefore, faces have two uses. Another terminology for the
use of a topological entity by higher order topological entities is
co-entities \cite{Ta00}.

Geometric modeling systems maintain tolerance information on how
numerically well the entities fit together.
This is necessitated by the fact that to function properly geometric
modeling systems must employ finite tolerances.
The algorithms and methods within the geometric
modeling system use the tolerance information to
effectively define and maintain a consistent representation of the
geometric model. (What many geometry-based
applications have referred to as dirty geometry is caused by a lack of
knowledge and proper use of the tolerance information \cite{BeWa04}.)

The abstraction of topology provides an effective means to develop
functional interfaces to boundary-based modelers. The ability to
generalize these interfaces is further enhanced by the fact that
the geometry shape information needed by most all
simulation procedures consists of pointwise interrogations 
that can be easily answered in a method independent of 
the modeler shape representation.

The developer of CAD systems support
geometry-based applications through general APIs.
These geometric modeling APIs have been successfully
used to developed automate finite element modeling processes
\cite{BeWa04,ShGe92}.

In some cases the only domain representation is a
mesh. In these cases it is still desirable
to construct a high level topological representation of the problem
domain. In this case the
process of constructing, or updating, the topological entities
associated with the domain geometric model is focused
on determining the appropriate sets of mesh regions, faces, edges, and
vertices to associate with the model regions, faces, edges and
vertices respectively. Algorithms to do this based on mesh based
geometry and/or simulation contact or fracture information
have been developed \cite{KrOr01,PaOr02,WaSh05}.
Once the model topology has been set,
the geometric shape information can be defined in terms of
the mesh facets, or can be made higher order \cite{CiOr00,OwWh01}.

An examination of the properties of analysis attributes indicates
they are tensorial quantities \cite{BeSo83} that
must be defined with respect to a coordinate system. Generalized 
structures and methods can define analysis attributes and associate them
geometric model entities \cite{OBSh02}.

\subsection{Domain Discretization}

The mesh, is a piecewise decomposition of the space/time domain. 
The specifics of the definition of the mesh is a function of the 
methods used to discretize the equations over the mesh entities.

It is common to employ different discretizations for the spatial and
temporal domains.  Since the definition of the spatial mesh is 
typically the more complex of the two, it is the focus of
this discussion. The requirements of the mesh are:

\begin{itemize}
\item To have the appropriately defined �union� of the mesh
entities represent the domain of interest.

\item To maintain, or have access to, the geometric shape information
needed for processes such as differentiation and integration.

\item To support the PDE discretization process over the mesh entities.

\item To maintain relationships of the mesh entities needed to support
the assembly of the complete discrete system and construction of the
solution fields.
\end{itemize}

One common mesh form is the conforming mesh where the intersections of
two mesh entities is null and the intersections of their closure is
either null or the closure of a common boundary mesh entity (face,
edge or vertex). A variant of this is the non-conforming mesh where
the intersections of the closure of two mesh entities is null, or
different parts of the boundaries of the two mesh entities. Other mesh
structures employ mesh patches that can interact in a variety of
ways. Finally, other methods are defined in terms of overlapping
regions (e.g., spheres or cubes). In each of these cases there are
rules on how the mesh entities interact, how
equation discretizations are performed over them, and how the complete
discrete system is assembled.

The geometric shape of the mesh entities must be
understood to support the equation discretization process. In
many methods the mesh geometry is implied from a discrete set of
parameters, which is satisfactory for fixed mesh simulations. An
alternative is to maintain
a linkage to the high level domain geometry. This alternative tends to
be expensive so
a mesh based geometric definition is typically used. However,
in the case of adaptive mesh improvements it is necessary to use the
links back to the original domain geometry to ensure the mesh
geometric approximation improves in a manner consisted with the order
of accuracy provided by the equation discretization process. For
example, as piecewise linear elements approximating curved portions of
the geometry are refined, the new mesh vertices need to be placed on
the curved boundary, or as the polynomial order
of an element is increased, the geometric approximation of the closure
of that entity must be increased to the correct order.

The data model for the mesh must maintain an association with the
domain definition, the discretization functions, the assembled
discrete system and the solution fields. From the perspective of
maintaining its relationship to the geometric domain, the use
of topological entities and their adjacency is ideal
\cite{BeSh97,DeOB01,Ta00}. In this manner it is possible to associate
the mesh entities to the domain entities to obtain needed attributes
and geometric information. This association between the two model
structures is referred to as classification below.

In other cases, such a representation is not ideal.
For example, something like an octree, or some other
spatially-based structure, is appropriate for the partition of unity
(so call meshfree) methods. In the case of structured meshes
maintaining a complete topology down to the individual cell entities
would be overkill.
However, in both of these cases there is information that is well
suited to a topological representation. For example, the boundaries
of the mesh patches in an structured mesh are ideally
defined in terms of a topological structure augmented with the rules
of mesh patch interaction. In the case of partition of unity
methods, maintaining topological entities of the cells of an
octree effectively supports the needed operations \cite{KlSh00}.

Consider the case of using a topological structure for the mesh.
Under the assumption that each topological mesh
entity of dimension $d$, $M^d_i$, is bounded by a set of topological
mesh entities of dimension $d-1$, 
$\left\{ M^d_i \left\{ M^{d-1} \right\} \right\}$, 
the full set of mesh topological entities are:

\begin{equation}
T_M = \left\{ \left\{ M \left\{ M^0 \right\} \right\},~
\left\{ M \left\{ M^1 \right\} \right\},~
\left\{ M \left\{ M^2 \right\} \right\},~
\left\{ M \left\{ M^3 \right\} \right\} \right\}
\end{equation}

where $\left\{ M \left\{ M^{d} \right\} \right\}$, $d=0,1,2,3$, are
respectively the set of vertices, edges, faces and regions which
define the topological entities of the mesh domain. It is
possible to limit the mesh representation to just these entities under
the following restrictions \cite{BeSh97}.

\begin{enumerate}
\item Regions and faces have no interior holes.

\item Each entity of order $d_i$ in a mesh, $M^{d_i}$, may use a particular entity of
lower order, $M^{d_j}$, $d_j<d_i$, at most once.

\item	For any entity $M^{d_i}_i$ there is a unique set of entities of order $d_i-1$,
$\left \{ M^{d_i}_i \left\{M^{d_{i-1}} \right\} \right\}$  that 
are on the boundary of $M^{d_i}_i$.
\end{enumerate}

The first restriction means that regions may be directly represented
by the faces that bound them, faces may be represented by the edges
that bound them, and edges may be represented by the vertices that
bound them. The second restriction allows the orientation of an entity
to be defined in terms of its boundary entities.
For example, the orientation of an edge,
$M^1_i$ bounded by vertices $M^0_j$ and $M^0_k$ is uniquely defined as
going from $M^0_j$ to $M^0_k$ only if $j \neq k$.

The third restriction means that a mesh entity is uniquely specified
by its bounding entities. Most representations including that used in
this paper employ that
requirement. There are representational schemes where this condition
only applies to interior entities; entities on the boundary of the
model may have a non-unique set of boundary entities \cite{BeSh97}.

A key component of supporting mesh-based simulations is the
association of the mesh with respect to the geometric model
\cite{BeSh97,ShGe92}. This association is referred to as
classification in which the mesh topological entities are classified
with respect to the geometric model topological entities upon which
they lie.

{\bf Definition: Classification} - {\it The unique association of mesh
topological entities of dimension $d_i$, $M^{d_i}_i$ to the
topological entity of the geometric model of dimension $d_j$,
$G^{d_j}_j$ where $d_i \leq d_j$, on which it lies is termed
classification and is denoted $M^{d_i}_i \sqsubseteq G^{d_j}_j$
where the classification symbol, $\sqsubseteq$,
indicates that the left hand entity, or set, is classified on the
right hand entity.}

{\bf Definition: Reverse Classification} - {\it For each model
entity, $G^d_j$ , the set of equal order mesh entities classified on that
model entity define the reverse classification information for that
model entity. Reverse classification is denoted as:}

\begin{equation}
RC(G^d_j) = \left\{ M^d_i | M^d_i \sqsubseteq G^d_j \right\}
\end{equation}

The concept of mesh entity classification to a higher level
model can be extended to include additional levels of model
decomposition. Two important cases of this are parallel mesh
partitions and structured mesh partitions. In the cases when these
partitions are non-overlapping the associations are obvious.
The concepts can be extended to the case of
overlapping partitions through the definition of an appropriate
rules of the interaction of entities in the different models.

Mesh shape information can be effectively associated with the
topological entities defining the mesh. In many cases this is limited
to the coordinates of the mesh vertices and, if they exist, higher
order nodes associated with mesh edges, faces or regions. In addition,
it is possible to associate other forms of geometric information with
the mesh entities. For example, the association of Bezier curves and
surface control points with mesh edges and faces for use in p-version
finite elements \cite{LuSh02}. The mesh classification can be can be
used to obtain other needed geometric information such as the
coordinates of a new mesh vertex caused by splitting a mesh edge
classified on a model face.

\subsection{Equation Discretization and the Definition of Solution Fields}

The PDEs being solved are written in terms of dependent
variables that are functions of the
space/time domain. For purposes of this discussion, consider the set of
PDEs being solved are written in the form:

\begin{equation}
{\cal{D}}({\bf u}, \sigma) - f = 0
\end{equation}

where 

\begin{itemize}
\item $\cal{D}$ represents the appropriate differential operators.

\item $\bf{u} (\bf{x},t)$ represents one of more vector dependent variables which are
functions of the independent variables of space, $\bf{x}$, and time, $t$ .

\item $\sigma$ represents one of more scalar dependent variables which are
functions of the independent variables of space, $\bf{x}$, and time, $t$.

\item $f$ represents the forcing functions.
\end{itemize}

(Note that the complete statement of a PDE problem must include a set
of boundary and, for time dependent problems, initial conditions.)

In the double discretization process used in mesh-based PDE solvers,
the dependent variables are discretized over the individual, or
groups of, mesh entities, either by direct operator discretization
(e.g., difference equations) or in terms of a set of basis
function. In both cases this process specifies a set of distribution
functions defining how the discretized variables vary over the mesh
entities and a set of yet to be determined multipliers, called degrees
of freedom (dof). The dof can always be
associated with a single mesh entity while the distribution functions
are associated with one or more mesh entities.
Three common cases that employ different combinations
of interactions between the mesh entities, the dof and the
distributions are:

\begin{itemize}
\item Finite difference based on a vertex stencil: In this case
the distribution functions are difference stencils
written in terms of dof that are the value of the field at
specific neighboring points. The dof are associated with mesh
vertices. The difference stencil is defined over the mesh entities 
that link the vertices involved with the stencil.

\item Finite volume methods: Finite
volume methods are constructed in terms of distribution function
written over individual mesh entities, referred to as cells. In most
cases the field being defined is $C^{-1}$ and the dof are not shared between
neighboring mesh entities. In this case the dof are associated
with the mesh entity the distribution is written over. The
coupling of the dof from different mesh entities is then through
operators acting over common boundary mesh entities.

\item Finite elements with common dof between neighboring elements:
Finite element distribution functions, referred to as shape functions,
are written over individual mesh entities, referred to as elements. In
cases where $C^m,~ m \geq 0$, continuity is required, the
shape functions associated with neighboring elements are made $C^m,~m
\geq 0$, continuous by having common dof associated with the bounding
mesh entities common to the neighboring elements. For example,
a $C^0$ field between two neighboring quadratic
elements in 2-D can be obtained by using the values of the field at one
point on the common edge and at the two
vertices bounding that edge as dof. In this
case the full set of dof used by the element distribution function can
be dof associated with any of the mesh entities in the closure of the
mesh entity of the element. There are other means to meet even higher
order continuity requirements, all
of which require sharing dof on the common boundaries.
\end{itemize}

The process of applying the discretization operation over the
appropriate mesh entities will produce a local contribution to
the complete fully discrete system. The
processes can be stated symbolically as:

\begin{equation}
\cal{D} (\bf{D}^c, \bf{d}^c) - \bf{f}^c = 0 
\end{equation}

where: 

\begin{itemize}
\item $\cal{D}$ represents the discretized differential operators written in terms
of appropriate distribution functions, $D^c$, over the domain of the
contributor $C$ and $\bf{d}^c$ represents the vector of dof associated with that
contributor.

\item $\bf{f}^c$ represents the discretized representation of the known
``forcing functions'' and boundary conditions for that
contributor.
\end{itemize}

The result of the discretization process yields a discrete
representation of the original PDEs that can be written as:

\begin{equation}
\bf{k}^c \bf{d}^c = \bf{f}^c
\label{eq:contrib_matrix}
\end{equation}

where $\bf{k}^c$ is a matrix of parameters for contributor $C$ that multiple the
vector of dof associated with that contributor, $\bf{d}^c$.

The construction of the system contributors can be controlled by the
appropriate traversal of information in the high level problem
definition, or at a level above the mesh such as the mesh patch level
for structured methods.

Note that the solution fields represent the variations of the tensor
variables over the domain of the problem. These fields
must be maintained is a form useful for the application of queries
and manipulation as needed for operations that include:

\begin{itemize}
\item The accurate transfer of the fields to other meshes to provide
input in a multiphysics analysis step, or to maintain the description
of the mesh on an adapted field.

\item The construction of new fields through operations
that may project them on new distribution with higher order
continuity, combine with other fields, etc.
\end{itemize}

\subsection{Discretized System Construction and Solution}

The relationship of the contributor level discretization given in
(\ref{eq:contrib_matrix})
to the complete discrete system is dictated by contributor level
mappings that ``map'' the contributor level dof to the
``assembled'' vector of the dof for the complete system, $\bf{d}$. The
process of constructing the complete system from the contributors is
referred to as the assembly process. Symbolically the complete
discrete system can be written as:

\begin{equation}
\bf{K} \bf{d} = \bf{F}
\end{equation}

where

\begin{itemize}
\item $\bf{K}$ is a system level matrix of parameters.
\item $\bf{d}$ is the complete vector of dof 
\item $\bf{F}$ is the complete right hand side vector.
\end{itemize}

Symbolically the relationship between the contributor level and system
level matrices and vectors can be depicted as:

\begin{equation}
\bf{d} ~=~ A^{N_c}_{c=1}(\bf{d}^c),~K~=~A^{N_c}_{c=1}(\bf{k}^c),~\bf{F}~=~A^{N_c}_{c=1}(\bf{f}^c)
\end{equation}

where 
\begin{itemize}
\item $N_c$ is the number of contributors in the complete system
\item $A^{N_c}_{c=1}$ indicates an assembly operator that is applied to each
contributors contributions and properly maps it to the complete
discrete system.
\end{itemize}

There are a variety of specific representational forms for the
complete systems matrices.  The specific form used is function of the
methods used to perform the computationally intensive process of
solving the discrete system to determine the values of the system dof.
The global algebraic equations are solved to produce the values of the
system dofs. Once the system level dof are determined, the mappings
between the contributor level and system level dof can be used to
complete the specification of the solution fields.





%\begin{verbatim}
%  - Gives the overview of PDE simulation steps - based on the general 
%    discretization docuement from last summer (Mark, can you send that out again?)
%  - Assumptions and definitions
%      - Mesh:  high level view of data model, functionality expected
%      - Geom:  high level view of data model, functionality expected
%      - Field: high level view of data model, functionality expected
%      - Others?
%  - How they are related to each other
%\end{verbatim}

\section{The TSTT Interface Definition efforts}

To support the flow of information in mesh-based simulations a number
of tools and technologies have been developed by different research
groups in academia, industry and the government labs.  For these tools
to have maximum impact it is important that they be interoperable,
interchangeable, and easily inserted into existing application
simulation codes.  Accomplishing this goal will allow easier
experimentation with different, but functionally similar, technologies
to determine which is best suited for a given application.  In
addition, it will provide mechanisms for combining technologies
together to create hybrid solution techniques that use multiple
advanced tools.

To accomplish this goal, we have defined an abstract data model that
encompasses a broad spectrum of mesh types and usage scenarios {\it
and} a set of common interfaces that are implementation and data
structure neutral.  The set of interfaces must be both small enough to
encourage adoption but also flexible enough to support a broad range
of mesh types.  {\it LAF should we move this figure and discussion to
section 2?} Figure \ref{fig:hieracrhy} shows the hierarchical
relationship between the geometric description of the computational
domain and first discretization step in PDE-based simulation.  At the
highest level is the geometric description of the computational
domain; in many cases this domain description will be a CAD boundary
representation.  This domain must be decomposed into a set of mesh
entities. This process may employ a hierarchical decomposition of the
geometric domain. For example, the first level may decompose the
domain into a set of sub-domains that can be meshed using various
meshing strategies.  In particular, hybrid meshes consisting of
different component meshes can be used to discretize different
portions of the geometric domain, or different full geometry meshes
can be used during different stages of the numerical solution.  Each
of these meshes is associated with the full geometry domain so that
any changes made there propogate properly to the associated meshes.
Each mesh can be further decomposed into partitions for solution on a
massively parallel computer.

The TSTT data model abstracts this simulation data hierachy and is
decomposed into three {\it core data types}: the geometric data, the
mesh data, and the field data.  These core data types are associated
with each other through {\it data relation managers}. The data
relation managers control the relationships among two or more of the
core data types, resolve cross references between entities in
different groups, and can provide additional functionality that
depends on multiple core data types.  Work on the mesh data model and
API has progressed the farthest and we describe it some detail in
Section \ref{sec:mesh}.  Preliminary work on the geometry and field
data model and interfaces are discussed as well.

A key aspect of this approach is that we do not enforce any particular
data structure or implementation with our interfaces, only that
certain questions about the mesh, geometry or field data can be
answered through calls to the interface.  To encourage adoption of the
interface we aim to create a small set of interfaces that existing
mesh and geometry packages can support.  The latter point is critical.
The DOE, NSF, DoD and other federal agencies have invested hundreds of
person-years in the development of a wide variety of geometry, mesh
generation and mesh management toolkits.  These software packages will
not be rewritten from scratch to conform to a common API, rather the
API must be data structure neutral and allow for a broad range of
underlying mesh representations. However, only a small set of
functionalities can be covered by a 'core' set of interface functions.
To increase the functionality of the TSTT interface, we define
additional, optional, interfaces for which we will provide reference
implementations based on the core interface methods.  Developers can
implement these functions on their own mesh database as needed.
The allows for incremental adoption of the interface.

One of the foremost challenges inherent in this type of effort include
balancing performance of the interface with the flexibility needed to
support a wide variety of mesh types.  Performance is critical for
kernel computations involving mesh access.  To address this need we
provide a number of different access patterns including array and
iterator-based.  The user may choose the access pattern that is best
suited for their application; the underlying implementation must
provide both styles of access even though only one is likely to be
native.  Further challenges arise when considering the support of many
different scientific programming languages.  This aspect is addressed
through our joint work with the Center for Component Technologies for
Terascale Simulation Science (CCTTSS) \cite{cca-forum} to provide
language independent interfaces by using their SIDL/Babel technology
\cite{babel}.  Preliminary results for the use of SIDL/Babel with
the TSTT mesh interface are given in Section \ref{sec:mesh_perf}.

\subsection{The TSTT Mesh Interface}

\subsubsection{The Mesh Data Model}
\label{sec:mesh}

The TSTT mesh data model is composed of two different types of
entities: mesh entities and entity sets.  To allow the interface to
remain data structure neutral, these entities are uniquely represented
by 32-bit opaque handle which may or may not be invariant through
different calls to the interface in the lifetime of the TSTT mesh.

{\bf Mesh Entity Definition}: TSTT mesh entities are the core of the
TSTT mesh interface and are defined by their entity type and entity
topology.  Allowable entity types are VERTEX (0D), EDGE (1D), FACE
(2D), and REGION (3D).  Allowable entity topologies are listed
WHERE??; each of these topologies has a unique entity type associated
with it.  Higher-dimensional entities are defined by lower-dimensional
entities using canonical ordering relationships.  Mesh geometry shape
information can be associated with the individual mesh entities. For
example, the vertices will have coordinates associated with
them. Higher order mesh entities can also have shape information
associated with them. For example the coordinates of higher-order
nodes can be associated with mesh edges, faces and regions.

%Vertices can return
%coordinate information in blocked or interleaved fashion.

Entity adjacency relationships define how the entities connect to
each other and both first-order and second-order adjacencies are
supported.
\begin{itemize}
\item {\it First-order adjacencies}: For an entity of dimension $d$,
first-order adjacencies return all of the mesh entities of dimension
$q$, which are either on the closure of the entity ($d > q$, downward
adjacency), or which it is on the closure of ($d < q$, upward
adjacency).  %If available, first-order adjacencies can be obtainable
%by either stored adjacencies, local traversal of stored adjacencies of
%an entity's neighborhood, or global mesh level traversal.  

\item {\it Second-order adjacencies}: There are times when applications 
would want not just what bounds an entity or the what an entity
bounds, but the next level of neighbors. Although such information can
always be determined from the appropriate first order adjacencies,
their application is common enough that supporting a second order
adjacency function is useful. A second order adjacency determines the
set of topological entities of a given order adjacent to entities that
that share common boundary entities of the specified order. An example
would be to determine the set of model regions that share a bounding
edge with the given region.

%For an entity of dimension $d$,
%second-order adjacencies describe all of the mesh entities of
%dimension $q$ that share any adjacent entities of dimension $b$, where
%$d \neq b$ and $b \neq q$.  Second-order adjacencies can be derived
%from first-order adjacencies.  Examples include for a given face, a
%set of regions adjacent to the face (first-order upward), a set of
%vertices bounding the face (first-order downward), a set of faces that
%share any vertex of the face (second-order).

\end{itemize}

To determine which adjacencies are supported by an underlying
implementation, an adjacency table is defined which can be
returned by a query through the interface.  Adjacencies are defined be
either immediately available, available through a local traversal,
available through a global traversal, or not available.  If adjacency
information exists, entities must be able to return both upward and
downward adjacency information in the canonical ordering using both
individual and agglomerated request mechanisms.

%%%% ***** MS Comment - To this points the sets are defined as discrete entity sets (as approsed to spaces, etc.) with some specific rules. These sets support specific standard operations. This is very useful and of critical importance to our ability to support applications.  However, I still have some indigestion the things we say sets can do and they "theory" aspects. We support basic Boolean operations (with some care needed) on sets of discrete entities - using the term "set theoretic sense" implies more than that - we should be specific. The use of the sets for most all the examples defined requires a fair amount of additional algorithm development and specific definitions that are not captured directly by basic the set operations. That's fine since we want a general tool that can support lots of things we have not considered and/or do not want to formalize. (There are some we may want to formalize that the current sets do poorly - but we can consider that when (and only when) we decide we want them. On the other hand, some stated examples go well beyond anything the sets come close to supporting - for example supporting the prolongation and restriction operators of multigrid methods with the mesh sets would actually be much harder than most of the normal mechanisms used for these processes. Based on this concern I have made some suggested edits to the text below to try to focus the discussion of sets on what I think we should tell people about them.

{\bf Entity Set Definition:} A TSTT entity set is an arbitrary
collection of TSTT entities that have uniquely defined entity handles.
Each EntitySet may be an unordered set or it
may a (possibly non-unique) ordered list of entities.  When the TSTT
mesh interface is first created in a simulation, a {\it Root Set} is
created and can be populated by string name using the load
functionality.  Example entity sets include a set of vertices, the set
of all faces classified on a geometric face, the set of regions in a
domain decomposition for parallel computing.%, the set of all entities
%in a given level of a multigrid mesh sequence.



Two primary relationships among EntitySets are supported:

\begin{itemize}
\item Entity sets may {\it contain} one or more entity sets.  An
entity set contained in another may be either a subset or an element
of that entity set.  The choice between these two interpretations is
left to the application; TSTT supports both interpretations. If entity
set A is contained in entity set B, a request for the contents of B
will include the entities in A and the entities in sets contained in A
if the application requests the contents recursively.  We note that
the {\it Root Set} cannot be contained in another entity set.

\item {\it Parent/child relationships} between entity sets are used to
represent relations between sets, much like edges connecting nodes in
a graph.  This relationship can be used to indicate that two meshes
have a logical relationship to each other, including multigrid and
adaptive mesh sequences. Because we distinguish between parent and
child links, this is a directed graph. Also, the meaning of cyclic
parent/child relationships is dubious, at best, so graphs must be
acyclic. No other assumptions are made about the graph.
\end{itemize}

Users are able to query entity sets for their entities and entity
adjacency relationships.  Both array- and iterator-based access
patterns are supported.  In addition, entity sets also have "set
operation" capabilities; in particular, you may add and remove
existing TSTT entities to the entity set and you may subtract,
intersect, or unite entity sets.  %In addition, subset and hierarchical
%parent/child relationships among EntitySets are supported.

We note that to be useful to computational simulations, entity sets
can comprise a valid computational mesh; the most simple example of
which is a nonoverlapping, connected set of TSTT entity regions, for
example, the structured and unstructured meshes commonly used in
finite element simulations.  Collections of entity sets can compose,
for example, overlapping and multiblock meshes. In both of these
examples, supplemental information on the interactions of the mesh sets
will be defined and maintained by the application.  Smooth particle
hydrodynamic (SPH) meshes can consist of a collection of TSTT vertices
with no connectivity or adjacency information.

%In addition, entity sets can also be extended to be "modifiable", in
%which case, basic operations that allow applications to change the
%geometry and topology are provided.  

The mesh interface, including the use of mesh sets, is extendable to
include ``modificiation operators'' that change the geometry and topology.
Capabilities include changing
vertex coordinates and adding or deleting entities. No validity checks
are provided with this basic interface so that care must be taken when
using these interfaces.  These interfaces are intended to support
higher-level functionality such as mesh quality improvement, adaptive
schemes, front tracking proceedures, and basic mesh generation
capabilities, all of which would provide validity checking.
Modifiable meshes require interactions with the underlying
geometric model including classifying entities.% and this interaction is described
%in Section \ref{sec:mesh_geom_class}.

Tags are used as containers for user-defined opaque data that can be
attached to TSTT entities and entity sets.  Tags can be
multi-valued which implies that a given tag handle can be associated
with many mesh entities.  In the general case, TSTT tags do not have a
predefined type and allow the user to attach any opaque data to mesh
entities.  To improve ease of use and performance, we support three
specialized tag types: integers, doubles, and Booleans.  Tags have and
can return their string name, size, handle and data (data retrieval is
done in the entity, mesh and entity set interfaces).  Tag data can be
retrieved from TSTT objects by handle in an agglomerated or individual
manner.  The implementation is expected to allocate the memory as
needed to store the tag data.

\subsubsection{Status of the mesh interface}

The TSTT mesh interface has been under development for approximately
two years and several revisions have been made in that time.  Several
implementations are well underway and are supported by mesh management
toolkits such as AOMD (RPI) \cite{ReSh03}, Overture (LLNL)
\cite{overture}, MOAB (SNL) \cite{moab}, NWGrid (PNL) \cite{nwgrid},
and GRUMMP (UBC) \cite{grummp}.  

In addition to the development of underlying implementations, the
TSTT mesh interface has also been used in a variety of contexts as
well.  In particular, it serves as the interface to the Mesquite 
mesh quality improvement and Frontier front tracking


\subsubsection{Preliminary performance results for the mesh interface}

\subsection{Geometry}




\subsection{The ITAPS Fields Interface}
\label{sec:fields}

Simulation fields represent tensor quantities defined in terms of
numerical analysis discretizations in a form useful to support queries
and operations by other functions or simulations. Common examples
where fields are used are (i) multiphysics analysis where the solution
fields from each physics analysis represents a forcing function or
boundary condition for another, (ii) the construction of external
adaptive control loops where the solution fields are used by error
estimation procedures to obtain estimates of the discretization errors
and to construct new mesh size field, and (iii) visualization and 
analysis/postprocessing.

Tensor quantities used in the quantification of problems of
mathematical physics are of order zero or greater and are defined over
a physical space or space/time domain.  Knowledge of the order of a
tensor and the dimension of the spatial domain over which it is defined,
gives the number of components needed to uniquely define the tensor
\cite{BeSo83}. The symmetries, for tensors of order 2 or greater,
define those components that are identical to, or the negative of
(antisymmetric), other components. The components of the tensor are,
in general, functions of the domain parameters as well as other
problem parameters. The ability to understand and use a tensor at any
particular instant requires knowledge of the coordinate system in
which the components of the tensor are referred.

The qualification of a tensor over a domain is called a field. The
field inherits the tensor order and spatial domain dimension from the
tensor along with any symmetries and constraints. The field
discretizes the tensor component values over the domain with
distributions and degrees of freedom (DOFs). The distributions are
defined over the mesh entities (and temporal discretization entities
as needed) and give the variation of the components of the
field. Thus, they must have the same functional domain that the
components of the tensor have.  The DOFs multiply the distributions and
set the magnitude of the variation of the individual distributions.

A complex simulation process can involve a number of fields defined
over various portions of the domain of the simulation. A single field
can be used by a number of different analysis routines that interact,
and the field may be associated with multiple meshes and have
a different relationship with each one.
In addition, different
distributions can be used by a field to discretize its associated
tensor. The ability to have a specific tensor defined over multiple
meshes and/or discretized in terms of multiple distributions is
handled by supporting multiple instances. A field instance has a
single set of distributions over a given mesh. These distributions are
defined over mesh entities which are of same dimension as the tensor
it is discretizing. A field instance can exist in an evaluated form
where the DOF have been determined, or in an unevaluated form where
the DOF are not yet determined.

ITAPS is currently defining interoperable field functions to:
\begin{itemize}
\item construct/load/save a field over a mesh,
\item interrogate the field at specific points and over mesh entities,
\item transform a field from one coordinate system to another,
\item project a field to a different set of basis functions (e.g., projecting a discontinuous 
stress field onto a set of continuous shape functions), and
\item transfer fields between different meshes including the use of different distributions. 
\end{itemize}

%% NOTE - IN THE BIBTEX FILE THIS REFERENCE MEEDS TO HAVE THE INITIALS 
%% P. P. ADDED TO THE 3RD AUTHOR.

%% @book{BeSo83,
%%  author="Beju, I. and Soos, E. and Teodorescu, P.P.", 
%%  title="Euclidean Tensor Calculus with Applications", 
%%  publisher="Abacus Press",
%%  year=1983




%% \subsection{Geometry Interface}
%% \begin{verbatim}
%%       - Data Model
%%       - Functionality
%%       - Status of interface/implementations
%% \end{verbatim}

%% \subsection{Field Interface}


\section{TSTT interface use cases}

\subsection{Adaptive Loop Construction}

Although mesh-based PDE 
codes are capable of providing results to the required levels of
accuracy, the vast majority lack the ability to automatically control
the mesh discretization errors through the application of adaptive
methods \cite{AiOd00,BaSt01,BaRa03}, thus leaving it to the user to
attempt to define an appropriate mesh.

One approach to support the application of adaptive analysis is to
alter the analysis code to include the error estimation and mesh
adaptation methods needed. The advantage of this approach is that the
resulting code can minimize the total computation and data
manipulation time required. The disadvantage is the amount of code
modification and development required to support mesh adaptation
is extensive since it requires extending the data structures
and all the procedures that interact with them. The expense and time
required to do this to existing fixed mesh codes is large and in most
cases considered prohibitive.

The alternative approach is to leave the fixed mesh analysis code
unaltered and to use the interoperable mesh, geometry and field
components to control the flow of information between the analysis
code and a set of other needed components. This approach has been used
to develop multiple adaptive analysis capabilities in which the
interoperable mesh, geometry and field components are used as follows:

\begin{itemize}
\item The geometry interface supports the integration with multiple
CAD systems. The interoperable API of the modeler enables interactions
with mesh generation and mesh modification to obtain all domain
geometry information needed \cite{BeWa04}.

\item The mesh interface provides the services for storing and
modifying mesh data during the adaptive process. The
Algorithm-Oriented Mesh Database \cite{ReSh03} was used for the examples given
here.

\item The field interface \cite{BeSh99} provides the functions to obtain
the solution information needed for error estimation and to support
the transfer of solution fields as the mesh is adapted.
\end{itemize}

One approach to support mesh adaptation is to use error estimators
to define a new mesh size field that is provided to an
automatic mesh generator that creates an entirely new mesh
of the domain. Although a popular approach, it has two
disadvantages. The first is the computational cost of an entire mesh
generation each time the mesh is adapted. The second is that in the
case of transient and/or non-linear problems, it requires global
solution field transfer between the old and new meshes. Such solution
transfer is not only computationally expensive, it can introduce
additional error into the solution which can dictate the ability of
the procedure to effectively obtain the level of solution accuracy
desired. An alternative approach to mesh adaptation is to apply local
mesh modifications \cite{LiSh05} that can range from standard templates, to
combinations of mesh modifications, to localized remeshing. Such
procedures have been developed that ensure the mesh's approximation to
the geometry is maintained as the mesh is modified \cite{LiSh03}. This is the
approach used to adaptive the mesh in the examples presented here.

\subsubsection{Adaptive Loop for Accelerator Design }

SLAC's eigenmode solver Omega3P, which is used in the design of next
generation linear accelerators, has been integrated with adaptive mesh
control \cite{GeLe04} to improve the accuracy and convergence of wall
loss (or quality factor) calculations in accelerator cavities. The
simulation procedure consists of interfacing Omega3P to solid models,
automatic mesh generation, general mesh modification, and error
estimator components to form an adaptive loop. The accelerator
geometries are defined as ACIS solid models \cite{spatial}. Using
functional interfaces between the geometric model and meshing
techniques, the automatic mesh generator MeshSim \cite{simmetrix}
creates the initial mesh. After Omega3P calculates the solution
fields, the error indicator determines a new mesh size field, and the
mesh modification procedures \cite{LiSh05} adapt the mesh.

The adaptive procedure has been applied to a Trispal 4-petal
accelerator cavity. Figure 1 shows the mesh and wall loss distribution
on the cavity surface for initial, first and final adaptive
meshes. The procedure has been shown to reliably produce results of
the desired accuracy for approximately one-third the number of
unknowns the previous user controlled procedure produced \cite{GeLe04}.



Figure 1. Adaptive analysis of a Trispal 4-petal accelerator cavity.


\subsubsection{Metal Forming Simulation}

In 3D metal forming simulations the workpiece undergo large
plastic deformations that result in major changes in the
domain geometry. The meshes of the deforming parts typically need to
be frequently modified to continue the analysis due to large element
distortions, mesh discretization errors and/or geometric approximation
errors. In these cases, it is necessary to replace the deformed mesh
with an improved mesh that is consistent with the current geometry.
Procedures to determine a new mesh size field
considering each of these factors has been developed and used in
conjunction with local mesh modification \cite{WaSh05}. The procedure includes
functions to transfer history dependent field variables as each mesh
modification is performed \cite{WaSh05}.

Figure 2 shows the set-up, initial mesh and final adapted meshes for a
steering link manufacturing problem solved using the DEFORM-3D
analysis engine \cite{Fl04} within a mesh modification-based adaptive
loop. A total stroke of 41.7mm is taken in the simulation. The initial
workpiece mesh consists 28,885 elements. The simulation is completed
with 20 mesh modification steps producing a final mesh with 102,249
elements.



Figure 2. Metal forming example.


\subsection{Mesh Quality Improvement}\label{sec:quality-improvement}

Mesh quality improvement techniques can be applied based on {\it a
priori} geometric quality metrics or {\it a posteriori} solution-based
metrics improvements.  Low-level mesh improvement operations include
vertex relocation, topology modification, vertex insertion, and vertex
deletion.   

The ITAPS center is supporting the development of a stand-alone mesh
quality improvement toolkit, called Mesquite~\cite{Mesquite03}.
Mesquite currently provides state-of-the-art algorithms for vertex
relocation and is flexible enough to work on a
wide array of mesh types ranging from structured meshes to unstructured
and hybrid meshes and a number of different two-and three-dimensional
element types.

% Question for Lori:  Why does Mesquite need to know the number of
% elements of a given type or topology?

Vertex relocation schemes must operate on the surface of the geometric
domain as well as in the interior of the domain to fully optimize the
mesh.  As such, the software must have functional access to both the
high level description of the geometric domain and to individual mesh
entities such as element vertices.  In particular, to operate on
interior vertices, Mesquite queries an ITAPS implementation for vertex
coordinate information, adjacency information, and the number of
elements of a given type or topology.  After determining the optimal
location for a vertex, Mesquite requests that the ITAPS implementation
update vertex coordinate information.  To operate on the surface mesh,
Mesquite must also use ITAPS geometric queries to determine the surface
normal and the closest point on the surface.  Explicit classification
of the mesh vertex against a geometric surface is required, as there
are some cases for which the closest point query will return a point
on the wrong surface, resulting in inverted or invalid meshes.

The ITAPS center is also supporting the development of a simplicial mesh
topology modification tool, which performs face and edge swapping
operations.\cite{TSTT-swap-tool}  This tool has been implemented using
the ITAPS mesh interface, enabling swapping in any ITAPS implementation
supporting triangles (2D) or tetrahedra (3D).

In gathering enough information to determine whether a swap is
desirable, any mesh topology modification scheme must make extensive use
of the ITAPS entity adjacency and vertex coordinate retrieval functions.
Reconfiguring the mesh, when this is appropriate, requires deletion of
old entities and creation of new entities through the ITAPS interface.
In addition, classification operators are again essential.  For
instance, reconfiguring tetrahedra that are classified on different
geometric regions results in tetrahedra that are not classified on
either region, so this case must be avoided.  Likewise, classification
checks make it easy to identify and disallow mesh reconfigurations that
would remove a mesh edge classified on a geometric edge.

In addition to basic geometry, topology and classification information,
a ITAPS implementation must provide additional information for mesh
improvement schemes to operate effectively and efficiently.  For
example, even for simple mesh improvement schemes, the implementation
must be able to indicate which entities may be modified and which may
not.  For mesh improvement schemes to operate on an entire mesh rather
than simply accepting requests entity by entity, an ITAPS implementation
must support some form of iterator.  Furthermore, advanced schemes may
allow the user to input a desired size, orientation, degree of
anisotropy, or even an initial reference mesh; exploiting such features
will require the implementation to associate many different types of
information with mesh entities and pass that information to the mesh
improvement scheme when requested.


\begin{verbatim}
  - SLAC design optimization 
  - Front tracking
\end{verbatim}

\section{Conclusions/Future work}

\section*{Acknowledgments} 

\bibliographystyle{plain}
\bibliography{tstt}

\end{document} 
