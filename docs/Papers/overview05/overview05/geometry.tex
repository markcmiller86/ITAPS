\subsection{The ITAPS Geometry Interface}
\label{sec:geom}

The goal of the geometry interface is to provide access to the
entities defining the geometric domain, the ability to determine
required geometric shape information associated with those entities
and, possibly, the ability to modify the geometric domain. The
geometry interface must account for the fact that the software modules
that provide geometry information are typically independent of 
mesh generators and PDE solvers.
%kkc the
%kkc simulation modules that employ the supplied geometry information to
%kkc make the mesh, and solve PDE's over it.

%kkc An examination of the geometry needs of mesh-based simulation
%kkc applications indicates a large fraction of their needs can be
%kkc satisfied through interface functions keyed by the primary topological
%kkc entities of regions, faces, edges and vertices. 

%lad moved to later
%A large fraction of the geometry needs in mesh-based simulations can
%be satisfied through interfaces keyed by the topological entities
%found in boundary representations: regions, faces, edges and vertices.
%A few situations, particularly those dealing with evolving geometry,
%will have need for the additional topological constructs of loops and
%shells.  Moreover, some interface functions will handle only
%topological entities and their adjacencies, whereas others will also
%provide the geometric shape information associated with the
%topological entities, provide control information, etc.  It is
%therefore useful to have the functions in the geometry interface
%divided into different groups; some of which can be optionally
%supported.

Three types of geometric models will be supported using the 
ITAPS inteface.  These include:

\begin{itemize}
\item Commercial modelers (e.g., Parasolid, ACIS, Granite).

\item Geometric modelers that operate from a utility that reads and operates on models that 
have been written to standard files like IGES and STEP (e.g., an ACIS
model read into Parasolid via a STEP file).

\item Geometric models constructed from an input mesh.
\end{itemize}

The first two geometric modeler types have no difficulty up-loading the model
topology and linking to the shape information.  In the first case,
the modeler already has it, and in the second case, the model structure
is defined within the standard file. In the last case, the input is a
mesh and algorithms must be applied to define the geometric model
topological entities in terms of the sets of appropriate mesh
entities. Such algorithms are not unique and depend on both the level
of information available with the mesh and knowledge of the analysis
process. 
%kkc The mesh functions can be used to load a mesh from which the
%kkc set of mesh entities classified on each geometric model topological
%kkc entities can be constructed using algorithms like that in reference
%kkc [5,6,9]. 
The mesh interface can be used to load a mesh, and algorithms such as
those found in \cite{KrOr01,PaOr02,WaKo04} can be used to construct the
topological entities of the corresponding geometric model. The shape
of the geometric model topological entities can be defined
directly by the mesh geometry of the entities classified on it, or
that information can be enhanced \cite{CiOr00,WaKo04}.

A large fraction of the geometry needs in mesh-based simulations can
be satisfied through interfaces keyed by the topological entities
found in boundary representations: regions, faces, edges and vertices.
A few situations, particularly those dealing with evolving geometry,
will have need for the additional topological constructs of loops and
shells.  Moreover, some interface functions will handle only
topological entities and their adjacencies, whereas others will also
provide the geometric shape information associated with the
topological entities, provide control information, etc.  
%It is therefore useful to have the functions in the geometry interface
%divided into different groups; some of which can be optionally
%supported.

It should be possible to employ the most effective means possible to
determine any geometric parameters that have to be calculated. The
primary complexity that arises is that not all
geometric model forms support the same methods and using the least
common denominator can introduce a large computation penalty over
alternatives that are supported in most cases. The primary example of
this is the use of parametric coordinates for model faces and
edges. The vast majority of the CAD systems employ parametric
coordinates and algorithms such as snapping a vertex to a model face.
Using parametric values can be two orders of magnitude faster
than using the alternative of closest point to a point in
space. Therefore, it is critical that the geometry interface functions
support the use of parametric values while having the ability to deal
with those cases when they are not available. This can be done by
having functions for when one does and does not have a parameterization.
 
The geometric interface functions are grouped by the level of
geometric model information needed to support them and the type of
information they provide \cite{TSTT-software}. The base level includes:

\begin{itemize}
\item Model loading which must load the model and initiate any supporting 
processes. Although the functions are the same for all sources of geometric
models, the implementation of them is a strong function of the model
source. If the source is a CAD API (e.g., ACIS or Parasolid API), the
appropriate API must be initiated and functions mapping to the
geometry interface functions defined. If it is a standard file
structure (e.g., STEP or IGES), the model must be loaded into an
appropriate geometric modeling functionality.  If the source is a mesh
model, it must be loaded, processed and linkage to the mesh geometry
constructed.

\item Topological queries based on the primary topological entities of 
regions, faces, edges, and vertices. The functions in this group include determining
topological adjacencies and entity iterators.

\item Pointwise interrogations which request geometric shape information 
with respect to a point in a single global coordinate system. Typical 
functions include returning the closest point on a model entity, 
getting coordinates, normals, tangents
and curvatures, and requesting bounding boxes of entities.

\item Entity level tags for associating user-defined information with entities. 
\end{itemize}

Other groups of functions increase the functionality and/or the
efficiency of the interface.  Some of these are commonly used while
others are not. Functions of this type that have been defined for the
geometry interface include:
\begin{itemize}
\item Geometric sense information that indicates how face normals and edge 
tangents are oriented.
\item Support of parametric coordinates systems for edges and faces. The functions 
in this group include conversion between global and parametric coordinates,
conversion between parametric coordinates of points on the closure of
multiple entities, and the full set of pointwise geometric
interrogations for a point given its classification and parametric
coordinates.
\item Support of geometric model tolerance information. These functions 
provide access to the geometric modeling tolerances used by the modeling system in
the determination of how closely adjacent entities must be
matched. This information is used to ensure that consistent decisions
are made by mesh-based operations using geometric shape information.
\end{itemize}

Additional functions of value to specific mesh-based applications that have not yet 
been defined include:
\begin{itemize}
\item Support of more complete topological models including shells and loops 
as well as complete non-manifold interactions,
\item Model topology and shape modification functions, and
\item Entity geometric shape information that defines the complete shape 
of model entities.
\end{itemize}

Functional geometry interfaces for mesh-based applications have been
under development and have been in use for a number of years for
automatic mesh generators \cite{BeWa04,ShGe92}. They have also been
used in the support of specific finite element applications such as
determining exact Jacobian information to support {\it p}-version
element stiffness matrix evaluation \cite{BeSh99}.  The current
interoperable geometry interface is being defined and implemented
building on these previous efforts.


