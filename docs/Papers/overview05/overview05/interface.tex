To support the flow of information in mesh-based simulations, a number
of tools and technologies have been developed by different research
groups in academia, industry, and the government labs.  For these tools
to have maximum impact, it is important that they be interoperable,
interchangeable, and easily inserted into existing application
simulation codes.  Accomplishing this goal will allow easier
experimentation with different, but functionally similar, technologies
to determine which is best suited for a given application.  In
addition, it will provide mechanisms for combining technologies
together to create hybrid solution techniques that use multiple
advanced tools.  To accomplish this goal, we have defined an abstract
data model that encompasses a broad spectrum of mesh types and usage
scenarios {\it and} a set of common interfaces that are implementation
and data structure neutral.  Our goal has been to keep the interfaces
small enough to encourage adoption but also flexible enough to support
a broad range of mesh types.

The ITAPS data model partitions the data required by a simulation
into three {\it core data types}: the geometric data, the mesh
data, and the field data.  Interfaces to the data represented by these
abstractions channel the flow of information throughout the
simulation.  For example, ITAPS adaptive mesh refinement services
access solution information for error estimation via the field
interface; modify the mesh using the mesh interface; and query the
geometry interface when creating mesh entites on domain boundaries.
These core data types are associated with each other through {\it data
relation managers}. The data relation managers control the
relationships among two or more of the core data types, resolve cross
references between entities in different groups, and can provide
additional functionality that depends on multiple core data types.
In addition, there are a number of basic functionalities and concepts
that are common to all three of the core data types, for example,
entities, creating sets of entities, and attaching user-defined data
to entities.  We discuss these concepts in Section \ref{sec:utilities}.
Work on the mesh data model and application programming interface
(API) has progressed the farthest, and we describe it in some detail
in Section \ref{sec:mesh}.  Preliminary work on the geometry and field
data model and interfaces are discussed as well in Sections \ref{sec:geom} and
\ref{sec:fields}.

A key aspect of the ITAPS approach is that we do not enforce any
particular data structure or implementation with our interfaces,
requiring only that certain questions about the geometry, mesh, or
field data can be answered through calls to the interface.  To
encourage adoption of the interface, we aim to create a small set of
interfaces that existing mesh and geometry packages can support.  The
latter point is critical.  The DOE, NSF, DoD and other federal
agencies have invested hundreds of person-years in the development of
a wide variety of geometry, mesh generation and mesh management
toolkits.  These software packages will not be rewritten from scratch
to conform to a common API, rather the API must be data structure
neutral and allow for a broad range of underlying mesh, geometry, and
field representations. However, only a small set of functionalities
can be covered by a 'core' set of interface functions.  To increase
the functionality of the ITAPS interface, we define additional,
optional, interfaces for which we will provide reference
implementations based on the core interface methods.  Developers can
incrementally adopt the interface by implementing the optional
functions on their own mesh database as needed.

One of the most challenging aspects of this effort remains balancing
performance of the interface with the flexibility needed to support a
wide variety of mesh types.  Performance is critical for kernel
computations involving mesh and geometry access.  To address this
need, we provide a number of different access patterns including array
and iterator-based.  The user may choose the access pattern that is
best suited for their application; the underlying implementation must
provide both styles of access even though only one is likely to be
native.  Further challenges arise when considering the support of many
different scientific programming languages.  This aspect is addressed
through our joint work with the Common Component Architecture Forum
\cite{cca-forum} to provide
language independent interfaces by using their SIDL/Babel technology
\cite{babel}.  

%%{\tt Insert example here}

\subsection{The ITAPS Basic Interface}
\label{sec:utilities}

The ITAPS data models for mesh, geometry and fields all make use of the
concepts of {\it entities}, {\it entity sets}, and {\it tags}, and we
describe these now in some detail.

ITAPS {\it entities} are used to represent atomic pieces of information
such as vertices in a mesh or edges in a geometric model.  To allow
the interface to remain data structure neutral, entities (as well as
entity sets and tags) are uniquely represented by opaque handles.
Unless entities are added or removed, these handles must be
invariant through different calls to the interface in the lifetime of
the ITAPS interface, in the sense that a given entity will always have
the same handle.  This is required to ensure consistency among the
several different calls that use and return entity handles and to
allow for easy entity handle comparison.  Entities do not have
interface functionality that is separate from mesh, geometry or field
interfaces, and we describe these functionalities in more detail in
the sections that follow.

Entity adjacency relationships define how the entities connect to
each other and both first-order and second-order adjacencies are
supported for the mesh and geometry interfaces.
\begin{itemize}
\item {\it First-order adjacencies}: For an entity of dimension $d$,
first-order adjacencies return all of the entities of dimension
$q$, which are either on the closure of the entity ($d > q$, downward
adjacency), or which it is on the closure of ($d < q$, upward
adjacency).  

\item {\it Second-order adjacencies}:  Many applications require not
only information about first-order adjacencies, but also about the next
level of neighbors. Although such information can always be determined
from the appropriate first-order adjacencies, their application is
common enough that supporting a second-order adjacency function is
useful. A second-order adjacency determines the set of topological
entities of a given type adjacent to entities that share common
boundary entities of the specified type. An example would be
the set of regions that share a bounding edge with the given region.
\end{itemize}

An ITAPS {\it entity set} is an arbitrary collection of ITAPS entities
that have uniquely defined entity handles.  Each entity set may be an
unordered set or it may be a (possibly non-unique) ordered list of
entities.  When an ITAPS interface is first created in a simulation, a
{\it Root Set} is created.  The root set can be populated by string
name using the {\tt load} function call.  The action taken by {\tt
load} is implementation specific and can range from reading mesh data
from a file to generating a mesh on the fly from a named CAD file.

Two primary relationships among entity sets are supported:

\begin{itemize}
\item Entity sets may {\it contain} one or more entity sets.  An
entity set contained in another may be either a subset or an element
of that entity set.  The choice between these two interpretations is
left to the application; ITAPS supports both interpretations. If entity
set A is contained in entity set B, a request for the contents of B
will include the entities in A and the entities in sets contained in A
if the application requests the contents recursively.  We note that
the {\it Root Set} cannot be contained in another entity set.

\item {\it Parent/child relationships} between entity sets are used to
represent relations between sets, much like directed edges connecting nodes in
a graph.  This relationship can be used to indicate that two meshes
have a logical relationship to each other, including multigrid and
adaptive mesh sequences. Because we distinguish between parent and
child links, this is a directed graph. Also, the meaning of cyclic
parent/child relationships is dubious, at best, so graphs must be
acyclic. No other assumptions are made about the graph.
\end{itemize}

Users are able to query entity sets for their entities and entity
adjacency relationships.  Both array- and iterator-based access
patterns are supported.  In addition, entity sets also have "set
operation" capabilities; in particular, existing ITAPS entities may be
added to or removed from the entity set, and sets may be subtracted,
intersected, or united.  

ITAPS {\it tags} are used as containers for user-defined opaque data that
can be attached to ITAPS entities and entity sets.  Tags can be
multi-valued which implies that a given tag handle can be associated
with many different entities.  In the general case, ITAPS tags do not
have a predefined type and allow the user to attach any opaque data to
ITAPS entities.  To improve ease of use and performance, we support
three specialized tag types: integers, doubles, and entity handles.  Tags
have and can return their string name, size, handle and data.  Tag
data can be retrieved from ITAPS entities by handle in an agglomerated
or individual manner.  The ITAPS implementation is expected to allocate
the memory as needed to store the tag data.

\subsection{The ITAPS Mesh Interface}
\label{sec:mesh}

ITAPS {\it mesh entities} are the fundamental building blocks of the
ITAPS mesh interface and correspond to the individual pieces of the
domain decomposition (mesh).  Under the assumption that each
topological mesh entity of dimension $d$, $M^d_i$, is bounded by a set
of topological mesh entities of dimension $d-1$, $\left\{ M^d_i
\left\{ M^{d-1}
\right\} \right\}$, the full set of mesh topological entities are:
\begin{equation}
T_M = \left\{ \left\{ M \left\{ M^0 \right\} \right\},~
\left\{ M \left\{ M^1 \right\} \right\},~
\left\{ M \left\{ M^2 \right\} \right\},~
\left\{ M \left\{ M^3 \right\} \right\} \right\}
\end{equation}
where $\left\{ M \left\{ M^{d} \right\} \right\}$, $d=0,1,2,3$, are
respectively the set of vertices, edges, faces and regions which
define the topological entities of the mesh domain. It is
possible to limit the mesh representation to just these entities under
the following restrictions \cite{BeSh97}.
\begin{enumerate}
\item Regions and faces have no interior holes.
\item Each entity of order $d_i$ in a mesh, $M^{d_i}$, may use a particular entity of
lower order, $M^{d_j}$, $d_j<d_i$, at most once.
\item	For any entity $M^{d_i}_i$ there is a unique set of entities of order $d_i-1$,
$\left \{ M^{d_i}_i \left\{M^{d_{i-1}} \right\} \right\}$  that 
are on the boundary of $M^{d_i}_i$.
\end{enumerate}

The first restriction means that regions may be directly represented
by the faces that bound them, faces may be represented by the edges
that bound them, and edges may be represented by the vertices that
bound them. The second restriction allows the orientation of an entity
to be defined in terms of its boundary entities.  For example, the
orientation of an edge, $M^1_i$ bounded by vertices $M^0_j$ and
$M^0_k$ is uniquely defined as going from $M^0_j$ to $M^0_k$ only if
$j \neq k$. The third restriction means that a mesh entity is uniquely
specified by its bounding entities. Most representations including
that used in this paper employ that requirement. There are
representational schemes where this condition only applies to interior
entities; entities on the boundary of the model may have a non-unique
set of boundary entities \cite{BeSh97}.

Specific examples of mesh entities include, for example, a hexahedron,
tetrahedron, edge, triangle and vertex.  Mesh entities are classified
by their entity type (topological dimension) and entity topology
(shape).  Just as for geometric entities, allowable mesh entity types
are vertex (0D), edge (1D), face (2D), and region (3D).  Allowable
entity topologies are point (0D); line segment (1D); triangle,
quadrilateral, and polygon (2D); and tetrahedron, pyramid, prism,
hexahedron, septahedron, and polyhedron (3D); each of these topologies
has a unique entity type associated with it.  Mesh entity geometry and
shape information is associated with the individual mesh entities. For
example, the vertices will have coordinates associated with them.
Higher-dimensional mesh entities can also have shape information
associated with them. For example the coordinates of higher-order
finite-element nodes can be associated with mesh edges, faces, and
regions.
 
Higher-dimensional entities are defined by lower-dimensional entities
with shape and orientation defined using canonical ordering
relationships.  To determine which adjacencies are supported by an
underlying implementation, an adjacency table is defined which can be
returned by a query through the interface.  The implementation can
report that adjacency information is always, sometimes, or never
available; and to be available at a cost that is constant, logarithmic
(i.e., tree search), or linear (i.e., search over all entities) in the
size of the mesh.  The use of a table allows the implementation to
provide separate information for each upward and downward adjacency
request.  If adjacency information exists, entities must be able to
return information in the canonical ordering using both individual and
agglomerated request mechanisms.

ITAPS {\it mesh entity sets} are extensively used to collect mesh
entities together in meaningful ways, for example, to represent the
set of all faces classified on a geometric face, or the set of regions
in a domain decomposition for parallel computing.  For some
computational applications, it is useful for entity sets to comprise a
valid computational mesh.  The simplest example of this is a
nonoverlapping, connected set of ITAPS region entities, for example,
the structured and unstructured meshes commonly used in finite element
simulations.  Collections of entity sets can compose, for example,
overlapping and multiblock meshes. In both of these examples,
supplemental information on the interactions of the mesh sets will be
defined and maintained by the application.  We note that in other
cases, for example, smooth particle hydrodynamic (SPH) applications,
molecular dynamics, or mesh-free methods, one can use meshes that
consist of a collection of ITAPS vertices with no connectivity or
adjacency information.

The mesh interface, including the use of mesh entity sets, is extendable to
include ``modification operators'' that change the geometry and topology.
Capabilities include changing vertex coordinates and adding or deleting
entities. No validity checks are provided with this basic interface so that
care must be taken when using these interfaces.  These interfaces are intended
to support higher-level functionality such as mesh quality improvement,
adaptive schemes, front tracking proceedures, and basic mesh generation
capabilities, all of which would provide validity checking.  Modifiable meshes
require interactions with the underlying geometric model including classifying
entities.  

Several implementations of the ITAPS mesh interface are well underway and are
supported by mesh management toolkits such as FMDB (RPI)~\cite{ReSh03},
MOAB (SNL)~\cite{moab:http}, NWGrid
(PNNL)~\cite{nwgrid:http}, and GRUMMP (University of British
Columbia)~\cite{GRUMMP:http}.  In addition to the development of underlying
implementations, the ITAPS mesh interface has also been used in a variety of
contexts as well.  In particular, it serves as the interface to the Mesquite
mesh quality improvement and Frontier front tracking tools (see Section 4).


\subsection{Geometry}




\subsection{The ITAPS Fields Interface}
\label{sec:fields}

Simulation fields represent tensor quantities defined in terms of
numerical analysis discretizations in a form useful to support queries
and operations by other functions or simulations. Common examples
where fields are used are (i) multiphysics analysis where the solution
fields from each physics analysis represents a forcing function or
boundary condition for another, (ii) the construction of external
adaptive control loops where the solution fields are used by error
estimation procedures to obtain estimates of the discretization errors
and to construct new mesh size field, and (iii) visualization and 
analysis/postprocessing.

Tensor quantities used in the quantification of problems of
mathematical physics are of order zero or greater and are defined over
a physical space or space/time domain.  Knowledge of the order of a
tensor and the dimension of the spatial domain over which it is defined,
gives the number of components needed to uniquely define the tensor
\cite{BeSo83}. The symmetries, for tensors of order 2 or greater,
define those components that are identical to, or the negative of
(antisymmetric), other components. The components of the tensor are,
in general, functions of the domain parameters as well as other
problem parameters. The ability to understand and use a tensor at any
particular instant requires knowledge of the coordinate system in
which the components of the tensor are referred.

The qualification of a tensor over a domain is called a field. The
field inherits the tensor order and spatial domain dimension from the
tensor along with any symmetries and constraints. The field
discretizes the tensor component values over the domain with
distributions and degrees of freedom (DOFs). The distributions are
defined over the mesh entities (and temporal discretization entities
as needed) and give the variation of the components of the
field. Thus, they must have the same functional domain that the
components of the tensor have.  The DOFs multiply the distributions and
set the magnitude of the variation of the individual distributions.

A complex simulation process can involve a number of fields defined
over various portions of the domain of the simulation. A single field
can be used by a number of different analysis routines that interact,
and the field may be associated with multiple meshes and have
a different relationship with each one.
In addition, different
distributions can be used by a field to discretize its associated
tensor. The ability to have a specific tensor defined over multiple
meshes and/or discretized in terms of multiple distributions is
handled by supporting multiple instances. A field instance has a
single set of distributions over a given mesh. These distributions are
defined over mesh entities which are of same dimension as the tensor
it is discretizing. A field instance can exist in an evaluated form
where the DOF have been determined, or in an unevaluated form where
the DOF are not yet determined.

ITAPS is currently defining interoperable field functions to:
\begin{itemize}
\item construct/load/save a field over a mesh,
\item interrogate the field at specific points and over mesh entities,
\item transform a field from one coordinate system to another,
\item project a field to a different set of basis functions (e.g., projecting a discontinuous 
stress field onto a set of continuous shape functions), and
\item transfer fields between different meshes including the use of different distributions. 
\end{itemize}

%% NOTE - IN THE BIBTEX FILE THIS REFERENCE MEEDS TO HAVE THE INITIALS 
%% P. P. ADDED TO THE 3RD AUTHOR.

%% @book{BeSo83,
%%  author="Beju, I. and Soos, E. and Teodorescu, P.P.", 
%%  title="Euclidean Tensor Calculus with Applications", 
%%  publisher="Abacus Press",
%%  year=1983




