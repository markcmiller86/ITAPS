\subsection{The ITAPS Fields Interface}
\label{sec:fields}

Simulation fields represent tensor quantities defined in terms of
numerical analysis discretizations in a form useful to support queries
and operations by other functions or simulations. Common examples
where fields are used are (i) multiphysics analysis where the solution
fields from each physics analysis represents a forcing function or
boundary condition for another, (ii) the construction of external
adaptive control loops where the solution fields are used by error
estimation procedures to obtain estimates of the discretization errors
and to construct new mesh size field, and (iii) visualization and 
analysis/postprocessing.

Tensor quantities used in the quantification of problems of
mathematical physics are of order zero or greater and are defined over
a physical space or space/time domain.  Knowledge of the order of a
tensor and the dimension of the spatial domain over which it is defined,
gives the number of components needed to uniquely define the tensor
\cite{BeSo83}. The symmetries, for tensors of order 2 or greater,
define those components that are identical to, or the negative of
(antisymmetric), other components. The components of the tensor are,
in general, functions of the domain parameters as well as other
problem parameters. The ability to understand and use a tensor at any
particular instant requires knowledge of the coordinate system in
which the components of the tensor are referred.

The qualification of a tensor over a domain is called a field. The
field inherits the tensor order and spatial domain dimension from the
tensor along with any symmetries and constraints. The field
discretizes the tensor component values over the domain with
distributions and degrees of freedom (DOFs). The distributions are
defined over the mesh entities (and temporal discretization entities
as needed) and give the variation of the components of the
field. Thus, they must have the same functional domain that the
components of the tensor have.  The DOFs multiply the distributions and
set the magnitude of the variation of the individual distributions.

A complex simulation process can involve a number of fields defined
over various portions of the domain of the simulation. A single field
can be used by a number of different analysis routines that interact,
and the field may be associated with multiple meshes and have
a different relationship with each one.
In addition, different
distributions can be used by a field to discretize its associated
tensor. The ability to have a specific tensor defined over multiple
meshes and/or discretized in terms of multiple distributions is
handled by supporting multiple instances. A field instance has a
single set of distributions over a given mesh. These distributions are
defined over mesh entities which are of same dimension as the tensor
it is discretizing. A field instance can exist in an evaluated form
where the DOF have been determined, or in an unevaluated form where
the DOF are not yet determined.

ITAPS is currently defining interoperable field functions to:
\begin{itemize}
\item construct/load/save a field over a mesh,
\item interrogate the field at specific points and over mesh entities,
\item transform a field from one coordinate system to another,
\item project a field to a different set of basis functions (e.g., projecting a discontinuous 
stress field onto a set of continuous shape functions), and
\item transfer fields between different meshes including the use of different distributions. 
\end{itemize}

%% NOTE - IN THE BIBTEX FILE THIS REFERENCE MEEDS TO HAVE THE INITIALS 
%% P. P. ADDED TO THE 3RD AUTHOR.

%% @book{BeSo83,
%%  author="Beju, I. and Soos, E. and Teodorescu, P.P.", 
%%  title="Euclidean Tensor Calculus with Applications", 
%%  publisher="Abacus Press",
%%  year=1983



