\subsection{Fields Interface}

PDE's are written in terms of tensor quantities. Fields are the
discretized representations of tensor quantities over the mesh and
represent the key quantities operated on in the numerical analysis of
PDE�s. A single field can be used by a number of different analysis
routines that interact and the field may be associated with multiple
meshes having alternative relationships between them. The definition
and quantification of a field includes:

\begin{itemize}
\item Specification of the tensor order and dimension which indicate the number of components of the tensor as well as any symmetries of the tensor [3]. 

\item Indication of the portion of the domain, and associated mesh entities, the field acts over.

\item Specification of the distribution functions defined over those mesh entities that indicate 
how the field can vary over the mesh.

\item Specification of the values of the degrees of freedom (dof) that will fix the values of the distribution of the field. 
\end{itemize}

The field interface is being designed to support the qualification and
manipulation of fields over meshes that can arise in simulations in
which the fields must interact with multiple meshes as used in
adaptive and/or multiphysics analysis. Central to this design is
defining the a field in terms of a qualification of the tensor it is
defining and a set of field instances where there is a field instance
defined for each mesh the field is defined over for that portion of
the domain. The information qualifying the tensor includes it order,
dimension, symmetries and coordinate system. This information is
required to define the storage of the tensor, support its
transformation to other systems and to define how various operations
must interact with it. The field instance includes qualification of
the mesh the fields acts over, the distribution functions defined over
the mesh, the dof and how the three components interact.

Efforts are currently underway to define effective mechanism to define
and relate the information defining fields that will effectively
support the needs of mesh-based applications. This will be followed
by the definition of interoperable functions to:
\begin{itemize}
\item Construct a field over a mesh.

\item Load and save fields.

\item Interrogate fields.

\item Manipulate fields on a given mesh.

\item Transfer fields between different meshes including the use of different distributions. 
\end{itemize}


