\subsection{Adaptive Loop Construction}

Although mesh-based PDE 
codes are capable of providing results to the required levels of
accuracy, the vast majority lack the ability to automatically control
the mesh discretization errors through the application of adaptive
methods \cite{AiOd00,BaSt01,BaRa03}, thus leaving it to the user to
attempt to define an appropriate mesh.

One approach to support the application of adaptive analysis is to
alter the analysis code to include the error estimation and mesh
adaptation methods needed. The advantage of this approach is that the
resulting code can minimize the total computation and data
manipulation time required. The disadvantage is the amount of code
modification and development required to support mesh adaptation
is extensive since it requires extending the data structures
and all the procedures that interact with them. The expense and time
required to do this to existing fixed mesh codes is large and in most
cases considered prohibitive.

The alternative approach is to leave the fixed mesh analysis code
unaltered and to use the interoperable mesh, geometry and field
components to control the flow of information between the analysis
code and a set of other needed components. This approach has been used
to develop multiple adaptive analysis capabilities in which the
interoperable mesh, geometry and field components are used as follows:

\begin{itemize}
\item The geometry interface supports the integration with multiple
CAD systems. The interoperable API of the modeler enables interactions
with mesh generation and mesh modification to obtain all domain
geometry information needed \cite{BeWa04}.

\item The mesh interface provides the services for storing and
modifying mesh data during the adaptive process. The
Algorithm-Oriented Mesh Database \cite{ReSh03} was used for the examples given
here.

\item The field interface \cite{BeSh99} provides the functions to obtain
the solution information needed for error estimation and to support
the transfer of solution fields as the mesh is adapted.
\end{itemize}

One approach to support mesh adaptation is to use error estimators
to define a new mesh size field that is provided to an
automatic mesh generator that creates an entirely new mesh
of the domain. Although a popular approach, it has two
disadvantages. The first is the computational cost of an entire mesh
generation each time the mesh is adapted. The second is that in the
case of transient and/or non-linear problems, it requires global
solution field transfer between the old and new meshes. Such solution
transfer is not only computationally expensive, it can introduce
additional error into the solution which can dictate the ability of
the procedure to effectively obtain the level of solution accuracy
desired. An alternative approach to mesh adaptation is to apply local
mesh modifications \cite{LiSh05} that can range from standard templates, to
combinations of mesh modifications, to localized remeshing. Such
procedures have been developed that ensure the mesh's approximation to
the geometry is maintained as the mesh is modified \cite{LiSh03}. This is the
approach used to adaptive the mesh in the examples presented here.

\subsubsection{Adaptive Loop for Accelerator Design }

SLAC's eigenmode solver Omega3P, which is used in the design of next
generation linear accelerators, has been integrated with adaptive mesh
control \cite{GeLe04} to improve the accuracy and convergence of wall
loss (or quality factor) calculations in accelerator cavities. The
simulation procedure consists of interfacing Omega3P to solid models,
automatic mesh generation, general mesh modification, and error
estimator components to form an adaptive loop. The accelerator
geometries are defined as ACIS solid models \cite{spatial}. Using
functional interfaces between the geometric model and meshing
techniques, the automatic mesh generator MeshSim \cite{simmetrix}
creates the initial mesh. After Omega3P calculates the solution
fields, the error indicator determines a new mesh size field, and the
mesh modification procedures \cite{LiSh05} adapt the mesh.

The adaptive procedure has been applied to a Trispal 4-petal
accelerator cavity. Figure 1 shows the mesh and wall loss distribution
on the cavity surface for initial, first and final adaptive
meshes. The procedure has been shown to reliably produce results of
the desired accuracy for approximately one-third the number of
unknowns the previous user controlled procedure produced \cite{GeLe04}.



Figure 1. Adaptive analysis of a Trispal 4-petal accelerator cavity.


\subsubsection{Metal Forming Simulation}

In 3D metal forming simulations the workpiece undergo large
plastic deformations that result in major changes in the
domain geometry. The meshes of the deforming parts typically need to
be frequently modified to continue the analysis due to large element
distortions, mesh discretization errors and/or geometric approximation
errors. In these cases, it is necessary to replace the deformed mesh
with an improved mesh that is consistent with the current geometry.
Procedures to determine a new mesh size field
considering each of these factors has been developed and used in
conjunction with local mesh modification \cite{WaSh05}. The procedure includes
functions to transfer history dependent field variables as each mesh
modification is performed \cite{WaSh05}.

Figure 2 shows the set-up, initial mesh and final adapted meshes for a
steering link manufacturing problem solved using the DEFORM-3D
analysis engine \cite{Fl04} within a mesh modification-based adaptive
loop. A total stroke of 41.7mm is taken in the simulation. The initial
workpiece mesh consists 28,885 elements. The simulation is completed
with 20 mesh modification steps producing a final mesh with 102,249
elements.



Figure 2. Metal forming example.

