The development of effective simulation codes for solving problems in
mathematical physics using mesh-based techniques continues to require
significant effort.  Because the governing equations can be
non-linear, coupled and/or subject to continued refinement, a wide
variety of functions must be addressed for each new problem area.
These include, for example, mesh generation and adaptive mesh control,
equation discretization, the solution of large systems of linear or
non-linear equations, and, potentially, coupling results between
different codes. The execution of many of these steps can be supported
using existing technologies, but coupling technologies developed by
different groups can be quite challenging.  This paper overviews an
approach which addresses this challenge for the generation and control
of meshes as well as the interactions of simulation information
defined on those meshes through the development of interoperable
components.

A number of different approaches have been used to provide tools and
technologies to mesh-based simulation code developers. Depending on
the starting point and needs of the specific code development process,
each has been found to be useful. These approaches can be categorized
into the following four groups:

\begin{enumerate}
\item complete simulation codes that support the integration of specific
      user-defined modules,
\item simulation frameworks that support the overall development process,
\item libraries that support specific aspects of the simulation
      process, and
\item components that encapsulate specific functionalities.
\end{enumerate}

There are many examples of complete simulation codes that provide a
small set of predefined routines that allow users to add specific
capabilities. One such well known example is ABAQUS \cite{abaqus}
which supports a user-defined material routine and a user-defined
finite element routine which allows users to include their own
constitutive relationships and finite element type,
respectively. Although predefined user routine interfaces place
specific limits on the form that can be added, they are particularly
useful if the missing functionality needed is covered by that
routine. For example, the ABAQUS material routine has been
successfully used to integrate hundreds of new material models ranging
from simple curve fits through homogenized constitutive relations
constructed from multiscale analysis.  The key disadvantages of this
approach are that the new capabilities that can be added are quite
limited and they can only be added through a specific interface. More
recently programs such as FEMLAB \cite{femlab} have been introduced
that increase the number of points of entry for new capabilities and
provide a broader definition of the interface.  This brings it much
closer to the simulation framework approach.

Simulation frameworks provide an overall structure, building on
particular data formats and other services, that support the effective
development and extension of the framework to provide new capabilities
\cite{BeSh99,BrLa97,DoLa96,De97,StEd04}. Efforts on simulation
framework development have taken advantage of modern programming
languages to provide developers a high degree of flexibility. However,
the choices made on the interface methods, data and algorithmic
services vary substantially among the various frameworks. The
differences correspond to the trade-offs associated with the types and
levels of generality supported and the computational efficiency that
can be obtained.  The framework approach can be particularly effective
when an entirely new simulation code is being developed and one of the
available frameworks is a good match in terms of functionality
supported and level of computational efficiency possible. However, for
developers with an existing code that are focused on incorporating new
capabilities, the framework approach is not ideal because moving
existing capabilities into the framework will be time consuming at
best.

The use of numerical libraries to support the development of numerical
simulation codes has a long history. The area where numerical
libraries have been, and continue to be, most successful is for
execution of computationally intensive core numerical algorithms like
algebraic system solvers, ordinary differential equation and
differential-algebraic solvers (e.g.,
\cite{AsPe98,petsc,BaGr97,eispack,lapack,linpack}). These libraries
provide capabilities for computationally intensive operations that are
most efficiently executed by the careful selection and implementation
of specific numerical algorithms. Although quite successful in their
specific areas, numerical libraries do not support development of
other portions of the code, nor do they support integration with
higher level functionalities.

Recently, the use of component technologies for the development of
simulation codes has started to become more prevalent \cite{cca-app}.
A component is defined to be a software object meant to interact with
other components that encapsulates a specific functionality, employs a
clearly defined interface and conforms to a prescribed behavior common
to all components in an architecture. Typically, each interface is
supported by multiple implementations which allows code developers to
easily experiment with different approaches.  This approach is ideal
in the case where there is already a substantial investment in the
simulation engine and the developers are interested in incorporating
advanced functionality or experimenting with several different,
related approaches.  Several groups are developing component
implementations for different aspects of the numerical solution
process including numerical solvers \cite{petsc,BaGr97}), ODE
integrators, and visualization tools \cite{cca-paper}.  However, more
work is required to increase the number of tools and technologies that
use a component-based approach.

In this paper, we describe ongoing research in the Terascale
Simulation Tools and Technologies (TSTT) center to develop geometry,
mesh and solution field components.  The primary reason for focusing
on this set of components is that it provides access to a broad set of
funtionalities to support simulation automation, solution reliability
in terms of discretization error control,
and the ability to integrate tools to construct multi-physics
simulations.  Simulation automation is supported through the geometry
and mesh components because they provide ready access to CAD-based
geometry definitions and automatic mesh generators. Solution
reliability is supported because these are the components needed to
support the effective creation and adaptive control of meshes. The
fields and mesh components are key to the effective coupling of
multiple simulation codes in the construction of multiphysics
simulations.

Section 2 provides a high level view of the functions and information
flow associated with the execution of mesh-based simulations starting
from a generalized problem definition and indicates the role of the
geometry, mesh and fields components in supporting these processes
which defined in more detail in Section 3. Section 4 provides example
applications of the initial version of these procedures to support
design optimization, adaptive mesh control, mesh quality improvement,
and front tracking. ****** THIS PARAGRAPH WILL NEED UPDATING TO BE
CONSISTENT WITH THE FINAL VERSION AND APPLICATIONS INCLUDED ********

