In mesh-based analysis procedures the
PDEs are solved approximately through a double discretization process
in which the problem's physical domain is discretized into a set of
piecewise components (e.g. a mesh) and the PDEs to be solved are
discretized over the mesh in an appropriate manner. The
discretizations of the PDEs over the mesh are assembled into a fully
discrete system that is solved. The result of this
process yields the construction of a set of discretized solution
fields. Methods covered by such approaches include finite
difference, finite volume, finite element, boundary element and
partition of unity (so-called meshfree) methods.

\subsection{Problem Definition}

To qualify the operations and information involved in executing a
mesh-based simulation, we begin with a qualification of the problem
definition which includes:

\begin{itemize}
\item The domain over which the simulation is to be solved. For the
classes of simulations being considered here the domain includes a
spatial component that is one-, two-, or three-dimensional. The problem
can also be defined over time in which case the domain also includes a
temporal component.

\item The mathematical form governing the simulation (PDE�s,
variational principle, weak form).

\item Specification of the parameters, referred to as physical
attributes, associated with the governing mathematical equations that
includes:

\begin{itemize}
\item material properties
\item forcing functions
\item boundary conditions
\item initial conditions
\end{itemize}
\end{itemize}

From the viewpoint of supporting a numerical simulation, the
domain representation must be able to:

\begin{itemize}
\item Support the construction of a mesh that represents the domain
level discretization used in the simulation.

\item Support the ability to address any geometry interrogation required.

\item Support the proper association of the physical and mathematical
attributes with the mesh.

\item Support the domain evolution in the cases where the domain
changes as part of the solution process.
\end{itemize}

There are multiple sources for the high level definitions of the
spatial component of the domain with CAD models, image data and
cell-based (mesh-based) being the most common. Each of these sources
has one or more representational forms. Historically, CAD systems use
boundary representations. Image data use a volumetric
form such as voxels or octrees. Depending on the configuration of the
cells a variety of implicit and explicit boundary or volumetric
representations have been used.

Except in cases of image data and when all aspects of the simulation
process can be effectively defined in terms of volume entities, it
is generally accepted that the use of a boundary
representation is well suited for the spatial domain definition. There
is a substantial computer-aided design literature on the various
boundary representations. Common to these representations is
the use the abstraction of topological entities and their adjacencies
to represent the model entities of different dimensions. In a
boundary representation the information defining the shape of
the topological entities can be thought of as attribute information
associated with the appropriate entities. The ability to interact with
topological entities provides an
effective means to develop abstract interfaces
allowing the easy integration to multiple domain definition
sources.

An important consideration in the selection of a boundary
representation is its ability to represent the classes of domain
needed. In the case of numerical simulations the domains to be meshed
can be general combinations of 0-, 1-, 2- and 3-D entities in general
configurations. Figure 1 shows a typical analysis domain that may be
used for the structural analysis of a portion of a piping
system. The analysis domain is an idealization
where portions of the pipes are idealized by beams,
the support bracket idealized by a plate
and a full 3-D solid used for the pipe juncture.



Figure 1. Example of a non-manifold model used in simulation.


The proper representation of such geometric domains, as
well as others like multi-material domains, are
referred to as a non-manifold boundary representations \cite{GuCh90,We88}. In
the case of non-manifold models the representation must indicate
how topological entities are used by bounding higher order
entities. For example, each side of a face may be used by a different
region. Therefore, faces have two uses. Another terminology for the
use of a topological entity by higher order topological entities is
co-entities \cite{Ta00}.

Geometric modeling systems maintain tolerance information on how
numerically well the entities fit together.
This is necessitated by the fact that to function properly geometric
modeling systems must employ finite tolerances.
The algorithms and methods within the geometric
modeling system use the tolerance information to
effectively define and maintain a consistent representation of the
geometric model. (What many geometry-based
applications have referred to as dirty geometry is caused by a lack of
knowledge and proper use of the tolerance information \cite{BeWa04}.)

The abstraction of topology provides an effective means to develop
functional interfaces to boundary-based modelers. The ability to
generalize these interfaces is further enhanced by the fact that
the geometry shape information needed by most all
simulation procedures consists of pointwise interrogations 
that can be easily answered in a method independent of 
the modeler shape representation.

The developer of CAD systems support
geometry-based applications through general APIs.
These geometric modeling APIs have been successfully
used to developed automate finite element modeling processes
\cite{BeWa04,ShGe92}.

In some cases the only domain representation is a
mesh. In these cases it is still desirable
to construct a high level topological representation of the problem
domain. In this case the
process of constructing, or updating, the topological entities
associated with the domain geometric model is focused
on determining the appropriate sets of mesh regions, faces, edges, and
vertices to associate with the model regions, faces, edges and
vertices respectively. Algorithms to do this based on mesh based
geometry and/or simulation contact or fracture information
have been developed \cite{KrOr01,PaOr02,WaSh05}.
Once the model topology has been set,
the geometric shape information can be defined in terms of
the mesh facets, or can be made higher order \cite{CiOr00,OwWh01}.

An examination of the properties of analysis attributes indicates
they are tensorial quantities \cite{BeSo83} that
must be defined with respect to a coordinate system. Generalized 
structures and methods can define analysis attributes and associate them
geometric model entities \cite{OBSh02}.

\subsection{Domain Discretization}

The mesh, is a piecewise decomposition of the space/time domain. 
The specifics of the definition of the mesh is a function of the 
methods used to discretize the equations over the mesh entities.

It is common to employ different discretizations for the spatial and
temporal domains.  Since the definition of the spatial mesh is 
typically the more complex of the two, it is the focus of
this discussion. The requirements of the mesh are:

\begin{itemize}
\item To have the appropriately defined �union� of the mesh
entities represent the domain of interest.

\item To maintain, or have access to, the geometric shape information
needed for processes such as differentiation and integration.

\item To support the PDE discretization process over the mesh entities.

\item To maintain relationships of the mesh entities needed to support
the assembly of the complete discrete system and construction of the
solution fields.
\end{itemize}

One common mesh form is the conforming mesh where the intersections of
two mesh entities is null and the intersections of their closure is
either null or the closure of a common boundary mesh entity (face,
edge or vertex). A variant of this is the non-conforming mesh where
the intersections of the closure of two mesh entities is null, or
different parts of the boundaries of the two mesh entities. Other mesh
structures employ mesh patches that can interact in a variety of
ways. Finally, other methods are defined in terms of overlapping
regions (e.g., spheres or cubes). In each of these cases there are
rules on how the mesh entities interact, how
equation discretizations are performed over them, and how the complete
discrete system is assembled.

The geometric shape of the mesh entities must be
understood to support the equation discretization process. In
many methods the mesh geometry is implied from a discrete set of
parameters, which is satisfactory for fixed mesh simulations. An
alternative is to maintain
a linkage to the high level domain geometry. This alternative tends to
be expensive so
a mesh based geometric definition is typically used. However,
in the case of adaptive mesh improvements it is necessary to use the
links back to the original domain geometry to ensure the mesh
geometric approximation improves in a manner consisted with the order
of accuracy provided by the equation discretization process. For
example, as piecewise linear elements approximating curved portions of
the geometry are refined, the new mesh vertices need to be placed on
the curved boundary, or as the polynomial order
of an element is increased, the geometric approximation of the closure
of that entity must be increased to the correct order.

The data model for the mesh must maintain an association with the
domain definition, the discretization functions, the assembled
discrete system and the solution fields. From the perspective of
maintaining its relationship to the geometric domain, the use
of topological entities and their adjacency is ideal
\cite{BeSh97,DeOB01,Ta00}. In this manner it is possible to associate
the mesh entities to the domain entities to obtain needed attributes
and geometric information. This association between the two model
structures is referred to as classification below.

In other cases, such a representation is not ideal.
For example, something like an octree, or some other
spatially-based structure, is appropriate for the partition of unity
(so call meshfree) methods. In the case of structured meshes
maintaining a complete topology down to the individual cell entities
would be overkill.
However, in both of these cases there is information that is well
suited to a topological representation. For example, the boundaries
of the mesh patches in an structured mesh are ideally
defined in terms of a topological structure augmented with the rules
of mesh patch interaction. In the case of partition of unity
methods, maintaining topological entities of the cells of an
octree effectively supports the needed operations \cite{KlSh00}.

Consider the case of using a topological structure for the mesh.
Under the assumption that each topological mesh
entity of dimension $d$, $M^d_i$, is bounded by a set of topological
mesh entities of dimension $d-1$, 
$\left\{ M^d_i \left\{ M^{d-1} \right\} \right\}$, 
the full set of mesh topological entities are:

\begin{equation}
T_M = \left\{ \left\{ M \left\{ M^0 \right\} \right\},~
\left\{ M \left\{ M^1 \right\} \right\},~
\left\{ M \left\{ M^2 \right\} \right\},~
\left\{ M \left\{ M^3 \right\} \right\} \right\}
\end{equation}

where $\left\{ M \left\{ M^{d} \right\} \right\}$, $d=0,1,2,3$, are
respectively the set of vertices, edges, faces and regions which
define the topological entities of the mesh domain. It is
possible to limit the mesh representation to just these entities under
the following restrictions \cite{BeSh97}.

\begin{enumerate}
\item Regions and faces have no interior holes.

\item Each entity of order $d_i$ in a mesh, $M^{d_i}$, may use a particular entity of
lower order, $M^{d_j}$, $d_j<d_i$, at most once.

\item	For any entity $M^{d_i}_i$ there is a unique set of entities of order $d_i-1$,
$\left \{ M^{d_i}_i \left\{M^{d_{i-1}} \right\} \right\}$  that 
are on the boundary of $M^{d_i}_i$.
\end{enumerate}

The first restriction means that regions may be directly represented
by the faces that bound them, faces may be represented by the edges
that bound them, and edges may be represented by the vertices that
bound them. The second restriction allows the orientation of an entity
to be defined in terms of its boundary entities.
For example, the orientation of an edge,
$M^1_i$ bounded by vertices $M^0_j$ and $M^0_k$ is uniquely defined as
going from $M^0_j$ to $M^0_k$ only if $j \neq k$.

The third restriction means that a mesh entity is uniquely specified
by its bounding entities. Most representations including that used in
this paper employ that
requirement. There are representational schemes where this condition
only applies to interior entities; entities on the boundary of the
model may have a non-unique set of boundary entities \cite{BeSh97}.

A key component of supporting mesh-based simulations is the
association of the mesh with respect to the geometric model
\cite{BeSh97,ShGe92}. This association is referred to as
classification in which the mesh topological entities are classified
with respect to the geometric model topological entities upon which
they lie.

{\bf Definition: Classification} - {\it The unique association of mesh
topological entities of dimension $d_i$, $M^{d_i}_i$ to the
topological entity of the geometric model of dimension $d_j$,
$G^{d_j}_j$ where $d_i \leq d_j$, on which it lies is termed
classification and is denoted $M^{d_i}_i \sqsubseteq G^{d_j}_j$
where the classification symbol, $\sqsubseteq$,
indicates that the left hand entity, or set, is classified on the
right hand entity.}

{\bf Definition: Reverse Classification} - {\it For each model
entity, $G^d_j$ , the set of equal order mesh entities classified on that
model entity define the reverse classification information for that
model entity. Reverse classification is denoted as:}

\begin{equation}
RC(G^d_j) = \left\{ M^d_i | M^d_i \sqsubseteq G^d_j \right\}
\end{equation}

The concept of mesh entity classification to a higher level
model can be extended to include additional levels of model
decomposition. Two important cases of this are parallel mesh
partitions and structured mesh partitions. In the cases when these
partitions are non-overlapping the associations are obvious.
The concepts can be extended to the case of
overlapping partitions through the definition of an appropriate
rules of the interaction of entities in the different models.

Mesh shape information can be effectively associated with the
topological entities defining the mesh. In many cases this is limited
to the coordinates of the mesh vertices and, if they exist, higher
order nodes associated with mesh edges, faces or regions. In addition,
it is possible to associate other forms of geometric information with
the mesh entities. For example, the association of Bezier curves and
surface control points with mesh edges and faces for use in p-version
finite elements \cite{LuSh02}. The mesh classification can be can be
used to obtain other needed geometric information such as the
coordinates of a new mesh vertex caused by splitting a mesh edge
classified on a model face.

\subsection{Equation Discretization and the Definition of Solution Fields}

The PDEs being solved are written in terms of dependent
variables that are functions of the
space/time domain. For purposes of this discussion, consider the set of
PDEs being solved are written in the form:

\begin{equation}
{\cal{D}}({\bf u}, \sigma) - f = 0
\end{equation}

where 

\begin{itemize}
\item $\cal{D}$ represents the appropriate differential operators.

\item $\bf{u} (\bf{x},t)$ represents one of more vector dependent variables which are
functions of the independent variables of space, $\bf{x}$, and time, $t$ .

\item $\sigma$ represents one of more scalar dependent variables which are
functions of the independent variables of space, $\bf{x}$, and time, $t$.

\item $f$ represents the forcing functions.
\end{itemize}

(Note that the complete statement of a PDE problem must include a set
of boundary and, for time dependent problems, initial conditions.)

In the double discretization process used in mesh-based PDE solvers,
the dependent variables are discretized over the individual, or
groups of, mesh entities, either by direct operator discretization
(e.g., difference equations) or in terms of a set of basis
function. In both cases this process specifies a set of distribution
functions defining how the discretized variables vary over the mesh
entities and a set of yet to be determined multipliers, called degrees
of freedom (dof). The dof can always be
associated with a single mesh entity while the distribution functions
are associated with one or more mesh entities.
Three common cases that employ different combinations
of interactions between the mesh entities, the dof and the
distributions are:

\begin{itemize}
\item Finite difference based on a vertex stencil: In this case
the distribution functions are difference stencils
written in terms of dof that are the value of the field at
specific neighboring points. The dof are associated with mesh
vertices. The difference stencil is defined over the mesh entities 
that link the vertices involved with the stencil.

\item Finite volume methods: Finite
volume methods are constructed in terms of distribution function
written over individual mesh entities, referred to as cells. In most
cases the field being defined is $C^{-1}$ and the dof are not shared between
neighboring mesh entities. In this case the dof are associated
with the mesh entity the distribution is written over. The
coupling of the dof from different mesh entities is then through
operators acting over common boundary mesh entities.

\item Finite elements with common dof between neighboring elements:
Finite element distribution functions, referred to as shape functions,
are written over individual mesh entities, referred to as elements. In
cases where $C^m,~ m \geq 0$, continuity is required, the
shape functions associated with neighboring elements are made $C^m,~m
\geq 0$, continuous by having common dof associated with the bounding
mesh entities common to the neighboring elements. For example,
a $C^0$ field between two neighboring quadratic
elements in 2-D can be obtained by using the values of the field at one
point on the common edge and at the two
vertices bounding that edge as dof. In this
case the full set of dof used by the element distribution function can
be dof associated with any of the mesh entities in the closure of the
mesh entity of the element. There are other means to meet even higher
order continuity requirements, all
of which require sharing dof on the common boundaries.
\end{itemize}

The process of applying the discretization operation over the
appropriate mesh entities will produce a local contribution to
the complete fully discrete system. The
processes can be stated symbolically as:

\begin{equation}
\cal{D} (\bf{D}^c, \bf{d}^c) - \bf{f}^c = 0 
\end{equation}

where: 

\begin{itemize}
\item $\cal{D}$ represents the discretized differential operators written in terms
of appropriate distribution functions, $D^c$, over the domain of the
contributor $C$ and $\bf{d}^c$ represents the vector of dof associated with that
contributor.

\item $\bf{f}^c$ represents the discretized representation of the known
``forcing functions'' and boundary conditions for that
contributor.
\end{itemize}

The result of the discretization process yields a discrete
representation of the original PDEs that can be written as:

\begin{equation}
\bf{k}^c \bf{d}^c = \bf{f}^c
\label{eq:contrib_matrix}
\end{equation}

where $\bf{k}^c$ is a matrix of parameters for contributor $C$ that multiple the
vector of dof associated with that contributor, $\bf{d}^c$.

The construction of the system contributors can be controlled by the
appropriate traversal of information in the high level problem
definition, or at a level above the mesh such as the mesh patch level
for structured methods.

Note that the solution fields represent the variations of the tensor
variables over the domain of the problem. These fields
must be maintained is a form useful for the application of queries
and manipulation as needed for operations that include:

\begin{itemize}
\item The accurate transfer of the fields to other meshes to provide
input in a multiphysics analysis step, or to maintain the description
of the mesh on an adapted field.

\item The construction of new fields through operations
that may project them on new distribution with higher order
continuity, combine with other fields, etc.
\end{itemize}

\subsection{Discretized System Construction and Solution}

The relationship of the contributor level discretization given in
(\ref{eq:contrib_matrix})
to the complete discrete system is dictated by contributor level
mappings that ``map'' the contributor level dof to the
``assembled'' vector of the dof for the complete system, $\bf{d}$. The
process of constructing the complete system from the contributors is
referred to as the assembly process. Symbolically the complete
discrete system can be written as:

\begin{equation}
\bf{K} \bf{d} = \bf{F}
\end{equation}

where

\begin{itemize}
\item $\bf{K}$ is a system level matrix of parameters.
\item $\bf{d}$ is the complete vector of dof 
\item $\bf{F}$ is the complete right hand side vector.
\end{itemize}

Symbolically the relationship between the contributor level and system
level matrices and vectors can be depicted as:

\begin{equation}
\bf{d} ~=~ A^{N_c}_{c=1}(\bf{d}^c),~K~=~A^{N_c}_{c=1}(\bf{k}^c),~\bf{F}~=~A^{N_c}_{c=1}(\bf{f}^c)
\end{equation}

where 
\begin{itemize}
\item $N_c$ is the number of contributors in the complete system
\item $A^{N_c}_{c=1}$ indicates an assembly operator that is applied to each
contributors contributions and properly maps it to the complete
discrete system.
\end{itemize}

There are a variety of specific representational forms for the
complete systems matrices.  The specific form used is function of the
methods used to perform the computationally intensive process of
solving the discrete system to determine the values of the system dof.
The global algebraic equations are solved to produce the values of the
system dofs. Once the system level dof are determined, the mappings
between the contributor level and system level dof can be used to
complete the specification of the solution fields.




