\subsection{Geometry Interface}

The goal of the geometry interface is to provide access to the
entities defining the geometric domain, the ability to determine
required geometric shape information associated with those entities
and, possibly, the ability to modify the geometric domain. The
geometry interface must account for the fact that the software modules
that provide geometry information are typically independent of the
simulation modules that employ the supplied geometry information to
make the mesh, and solve PDE�s over it.

An examination of the geometry needs of mesh-based simulation
applications indicates a large fraction of their needs can be
satisfied through interface functions keyed by the primary
topological entities of regions, faces, edges and vertices. A few
situations, particularly those dealing with evolving geometry, will
have need for the additional topological constructs of loops and
shells. It is therefore, useful to have the geometry interface functions placed in groups at different levels of interface. It is also
useful to group the interface functions based on if they deal with
topological entities and their adjacencies only, provide the geometric
shape information associated with the topological entities, provide
control information, etc.

Three types of geometric model API's will be supported in this
manner. These include:

\begin{itemize}
\item Commercial modeler API's (e.g., Parasolid, ACIS, Granite).

\item Geometric modelers that operate off of a utility that reads and operates on models that 
have been written to standard files like IGES and STEP (e.g.,
Overture�s geometry interface, an ACIS model read into Parasolid via a
STEP file, etc.).

\item Geometric models constructed from an input mesh.
\end{itemize}

The first two API types have no difficulty up-loading the model
topology and linking to the shape information since in the first case
the modeler already has it and in the second case the model structure
is defined within the standard file. In the last case the input is a
mesh and algorithms must be applied to define the geometric model
topological entities in terms of the sets of appropriate mesh
entities. Such algorithms are not unique and depend on both the level
of information available with the mesh and knowledge of the analysis
process. The mesh functions can be used to load a mesh from which the
set of mesh entities classified on each geometric model topological
entities can be constructed using algorithms like that in reference
[5,6,9]. The shape of the geometric model topological model entities
can be defined directly by the mesh geometry of the entities
classified on it, or that information can be enhanced [4,9].

It should be possible to employ the most effective means possible to
determine any geometric parameters that have to be calculated. The
primary complexity that arises in meeting this is that not all
geometric model forms support the same methods and using the least
common dominator can introduce huge computation penalty over
alternatives that are supported in most cases. The primary example of
this is the use of parametric coordinates for model faces and
edges. The vast majority of the CAD systems employ parametric
coordinates and algorithms such as snapping a vertex to a model face
using parametric values can be make two orders of magnitude faster
that using the alternative of closest point to a point in
space. Therefore, it is critical that the geometry interface functions
support the use of parametric values while having the ability to deal
with those cases when they are not available. This can be done by
having functions for when one does and does not have parameterization.

The geometric interface functions **** INSERT A REFERENCE TO THE
GEOMETRY INTERFACE DOCUMENT ON THE TSTT WEB PAGE - WE NEED TO PUT IT
THERE IF IT IS NOT THERE ALREADY *****are grouped by the level of
geometric model information needed to support them and the type of
information they provide. The groups defined as the base level
includes:

\begin{itemize}
\item Model loading which must load the model. and initiate any supporting processes. Although the functions are the same for all sources of geometric
models, the implementation of them is a strong function of the model
source. If the source is a CAD API (e.g., ACIS or Parasolid API) the
appropriate API must be initiated and functions mapping to the
geometry interface functions defined. If it is a standard file
structure (e.g., STEP or IGES) the model must be loaded into an
appropriate geometric modeling functionality.  If the source is a mesh
model it must be loaded, processed and linkage to the mesh geometry
constructed.

\item Topological queries based on the primary topological entities of regions, faces, edges and vertices. The functions in this group include determining of
topological adjacencies and entity iterators.

\item Pointwise interrogations which request geometric shape information with respect to a point in a single global coordinate system. Typical functions include returning the closest point on a model entity, getting coordinates, normals, tangents
and curvatures, and requesting bounding boxes of entities.

\item Entity level tags for associating information with entities. 
\end{itemize}

Other groups of functions increase the functionality and/or the efficiency of the interface. 
Some of these are commonly used while others are not. Functions of this type that have 
been defined for the geometry interface include:

\begin{itemize}
\item Geometric sense information that indicates how face normals and edge tangents are oriented.

\item Support of parametric coordinates systems for edges and faces. The functions in this 
group include conversion between global and parametric coordinates,
conversion between parametric coordinates of points on the closure of
multiple entities, and the full set of pointwise geometric
interrogations for a point given its classification and parametric
coordinates.

\item Support of geometric model tolerance information. The functions here provide access 
to the geometric modeling tolerances used by the modeling system in
the determination of how closely adjacent entities must be
matched. This information is used to ensure that consistent decisions
are made by mesh-based operations using geometric shape information.
\end{itemize}

Additional functions that of value to specific mesh-based applications that have not yet 
been defined include:
\begin{itemize}
\item Support of more complete topological models including shells and loops as well and complete non-manifold interactions.

\item Model topology and shape modification functions.

\item Entity geometric shape information that defines the complete shape of model entities.
\end{itemize}

Functional geometry interfaces for mesh-based applications has been under development 
and have been in use for a number of years for automatic mesh generators [1,8]. They have 
also been used in the support of specific finite elements applications such as determining 
exact Jacobian information to support p-version element stiffness matrix evaluation [2]. 
The current interoperable geometry interface is being defined and implemented building 
on these previous efforts. 


