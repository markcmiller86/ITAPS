\documentclass{aiaa-tc}
\usepackage{pslatex}
\usepackage[T1]{fontenc}
\usepackage[latin1]{inputenc}
\pagestyle{empty}
\setcounter{secnumdepth}{3}
\setcounter{tocdepth}{3}
\usepackage{float}
\usepackage{geometry}
\geometry{verbose,letterpaper,tmargin=1in,bmargin=1in,lmargin=1in,rmargin=1in}
\usepackage{multicol}

%%%%%%%%%%%%%%%%%%%%%%%%%%%%%% LyX specific LaTeX commands.
%% Bold symbol macro for standard LaTeX users
\providecommand{\boldsymbol}[1]{\mbox{\boldmath $#1$}}

%%%%%%%%%%%%%%%%%%%%%%%%%%%%%% User specified LaTeX commands.
\renewcommand{\floatpagefraction}{1}
\renewcommand{\textfraction}{0}
\renewcommand{\topfraction}{0.9}
\setcounter{topnumber}{4}
\usepackage{url}

\newenvironment{lyxlist}[1]
 {\begin{list}{}
   {\settowidth{\labelwidth}{#1}
    \setlength{\leftmargin}{\labelwidth}
    \addtolength{\leftmargin}{\labelsep}
    \renewcommand{\makelabel}[1]{##1\hfil}}}
 {\end{list}}
 \newenvironment{lyxcode}
   {\begin{list}{}{
     \setlength{\rightmargin}{\leftmargin}
     \setlength{\listparindent}{0pt}% needed for AMS classes
     \raggedright
     \setlength{\itemsep}{0pt}
     \setlength{\parsep}{0pt}
     \normalfont\ttfamily}%
    \item[]}
   {\end{list}}

\begin{document}

\title{The TSTT Mesh Interface}


\author{Carl Ollivier-Gooch\thanks{Advanced Numerical Simulation
    Laboratory, University of British Columbia} \and Kyle
  Chand\thanks{Center for Applied Scientific Computing, Lawrence
    Livermore National Laboratory} \and Tamara Dahlgren\thanksibid{2}
    \and Lori Freitag Diachin\thanksibid{2} \and Brian
    Fix\thanks{Dept. of Applied Mathematics and Statistics, SUNY
    Stonybrook} \and Jason Kraftcheck\thanks{Parallel Computing
    Sciences, Sandia National Laboratories} \and Xiaolin
    Li\thanksibid{3} \and E. Seegyoung Seol\thanks{Scientific
    Computation Research Center, Rensselaer Polytechnic Institute } \and
    Mark S. Shephard\thanksibid{5} \and Timothy Tautges\thanksibid{4} \and
    Harold Trease\thanks{Pacific Northwest National Laboratory} }

\maketitle
\begin{abstract}

PDE-based numerical simulation applications commonly use basic software
infrastructure to manage mesh, geometry, and discretization data. The
commonality of this infrastructure implies the software is theoretically
amenable to re-use. However, the traditional reliance on library-based
implementations of these functionalities hampers experimentation with
different software instances that provide similar functionality.  This
is especially true for meshing and geometry libraries where applications
often directly access the underlying data structures, which can be quite
different from implementation to implementation.  Thus, using different
libraries interchangeably or interoperably for this functionality has
proven difficult at best and has hampered the wide spread use of
advanced meshing and geometry tools developed by the research community.
To address these issues, the Terascale Simulation Tools and Technologies
center is working to develop standard interfaces to enable the creation
of interoperable and interchangeable simulation tools. In this paper, we
focus on a language- and data-structure-independent interface supporting
query and modification of mesh data conforming to a general abstract
data model. We describe the model and interface, and provide programming
``best practices'' recommendations based on early experience
implementing and using the interface.

\end{abstract}

\section*{List of Abbreviations}
\begin{multicols}{3}
\begin{lyxlist}{Topo}
\item[AI]{Adjacency information enum}
\item[EH]{Entity handle}
\item[ES]{Entity set handle}
\item[ET]{Error type enum}
\item[iter]{Iterator over entities}
\item[SH]{Set handle}
\item[SO]{Storage order enum}
\item[TH]{Tag handle}
\item[TVT]{Tag value type enum}
\item[Topo]{Entity topology enum}
\item[Type]{Entity type enum}
\item[VH]{Vertex handle}
\end{lyxlist}
\end{multicols}

\section{Introduction\label{sec:Introduction}}

Creating simulation software for problems described by partial
differential equations is a relatively common but very time-consuming
task. Much of the effort of developing a new simulation code goes into
writing infrastructure for tasks such as interacting with mesh and
geometry data, equation discretization, adaptive refinement, design
optimization, etc. Because these infrastructure components are common to
most or all simulations, re-usable software for these tasks would
significantly reduce both the time and expertise required to create a
new simulation code.

Currently, libraries are the most common mechanism for software re-use
in scientific computing, especially the highly-successful libraries for
numerical linear algebra\cite{petsc,BaGr97,eispack,lapack,linpack}.
The drawback to software re-use through libraries is the difficulty in
changing from one to another. When a user wishes to add functionality or
simply experiment with a different implementation of the same
functionality in another library, all calls within an application must
be changed to the other API, which likely will not package functionality
in precisely the same way. Another significant challenge with library
use, especially in the context of meshing and geometry libraries, is
that data structures used within the libraries may be radically
different, making changes from one library to another even more onerous.
This time-consuming conversion process can be a significant diversion
from the central scientific investigation, so many application
researchers are reluctant to undertake it. This can lead to the use of
sub-optimal strategies.  For example, new advances developed by the
meshing research community often take years to become incorporated into
application simulations.

To address these issues, the Interoperable Tools for Advanced Petascale
Simulation (ITAPS) center is working to develop interoperable software
tools for meshes, domain geometry, and
discretization\cite{tstt:overview}.  The present paper will discuss our
work in developing a mesh interface.  The most prominent example of
prior research in defining interfaces for meshing is the Unstructured
Grid Consortium, a working group of the AIAA Meshing, Visualization, and
Computing Environments Technical Committee\cite{UGC-web}.  The first
release of the UGC interface\cite{UGC-v1} was aimed at high level mesh
operations, including mesh generation and quality assessment.
Recognizing a need for lower-level functionality, the UGC has developed
a low-level query and modification interface for mesh databases, as well
as an interface for defining generic high-level
services\cite{UGC-v2:paper}.

The ITAPS mesh interface, called iMesh, has a broader scope than the UGC
interface.  In addition to supporting low-level mesh manipulation, the
iMesh interface is also designed to support the requirements of solver
applications, including the ability to define mesh subsets and to attach
arbitrary user data to mesh entities.  In addition, the iMesh interface
is intended to be both language and data structure independent.  In
summary, our initial target is to support low-level interaction between
applications programs --- both meshing and solution applications --- and
external mesh databases regardless of the data structures and
programming language used by each.  In the long term, we expect to also
support high-level operations, including mesh generation, typically as
services built using the iMesh interface.

The fundamental challenge in developing this interface has been the
tension between generality and compactness: our goal has been to define
a set of operations addressing all common uses of mesh data while
minimizing redundancy and avoiding idioms peculiar to a particular
underlying mesh representation.  A common theme in many design decisions
while developing the iMesh interface has been to support common
constructs as simply, directly, and efficiently as possible while still
allowing more sophisticated, less common constructs to be expressed in
the data model and interface.

We began by defining a general abstract data model, focusing on the ways
in which mesh data is used in simulations rather than on how mesh data
is stored by meshing tools.  The data model, described in detail in
Section~\ref{sec:Data-Model}, includes fundamental mesh entities ---
vertices, faces, elements, etc --- and the topological relationships
between them, as well as the concepts of general mesh subsets and
arbitrary data associated with mesh entities.

The mesh interface is built on this data model.  The interface
(Section~\ref{sec:Interface}) supports global and local mesh query, mesh
modification, and collections and tagging of mesh entities.  
% This needs to be said somewhere, but where exactly?  Is this the right
% place? 
The iMesh interface is built on a client-server model, with the
explicit assumption that the client (application) and server (mesh
database) may be written in different programming languages.  To address
cross-language issues, especially with arrays and strings, the iMesh
interface is defined using the Scientific Interface Description Language
(SIDL)\cite{babel:site05,babel:usersguide05}.  This language neutral
description is then processed by an existing interpreter, Babel, to
produce a language-specific client API and server skeleton, as well as
glue code that mediates language translation issues.

Performance data from early usage of the interface suggests there is a
preferred coding style for using the iMesh interface (see
Section~\ref{sec:Programming}).  The interface is already in use in
various meshing tools and simulation applications and on-going
development continues to improve the usability and accessibility of
iMesh-compliant software (see Section ~\ref{sec:Conclusions}).




\section{Data Model\label{sec:Data-Model}}

In the iMesh data model, all mesh primitives --- vertices (0D),
edges (1D), faces (2D), and regions (3D) --- are referred to as
\emph{entities}.  Mesh entities are collected together to form
\emph{entity sets}. All topological and geometric mesh
data\footnote{\emph{Geometric mesh data} is geometric data required to
define shapes of mesh entities. This is distinct from \emph{geometric
model data}, which defines the shapes of the problem domain.} is stored
in a \emph{root entity set} and there is a single root set for each
computational domain; all other entity sets are contained in the root
set. Many implementations will represent the root set as a database
containing all of the mesh entities, with other entity sets containing
handles for these entities.  Any iMesh data object --- an entity or
any entity set including the root set --- can have one or more
\emph{tags} associated with it, so that arbitrary data can be attached
to the object. To preserve data structure neutrality, all iMesh data
objects are identified by opaque handles.


\subsection{Mesh Entities\label{sub:Mesh-Entities}}

All the primitive components of a mesh are defined by the iMesh data
model to be of type \texttt{Entity}. iMesh entities are
distinguished by their entity type (effectively, their topological
dimension) and entity topology; each topology has a unique entity type
associated with it. Examples of entities include a vertex, an edge,
triangular or quadrilateral faces in 2D or 3D, and tetrahedral or
hexahedral regions in 3D. Faces and regions have no interior holes.
Higher-dimensional entities are defined by lower-dimensional entities
using a canonical ordering.


\subsection{Entity Adjacencies\label{sub:Entity-Adjacencies}}

Adjacencies describe how mesh entities connect to each other. For an
entity of dimension $d$, first-order adjacency returns all of the mesh
entities of dimension $q$ which are on the closure of the entity for
downward adjacency ($d>q$), or for which the entity is part of the
closure for upward adjacency ($d<q$). For a particular implementation,
not all first-order adjacencies are necessarily available.  For
instance, in a classic finite element element-node connectivity storage,
requests for faces or edges adjacent to an entity may return nothing,
because the implementation has no stored data to return.  For
first-order adjacencies that are available in the implementation, the
implementation may store the adjacency information directly, or compute
adjacencies by either a local traversal of the entity's neighborhood or
by global traversal of the entity set. Each iMesh implementation
must provide information about the availability of and relative cost of
first-order adjacency queries.

For an entity of dimension $d$, second-order adjacencies describe
all of the mesh entities of dimension $q$ that share any adjacent
entities of dimension $b$, where $d\neq b$ and $b\neq q$. Second-order
adjacencies can be derived from first-order adjacencies. Note that,
in the iMesh data model, requests such as all vertices that are neighbors
to a given vertex are requests for second-order adjacencies.

Examples of adjacency requests include: for a given face, the regions
on either side of the face (first-order upward); the vertices bounding
the face (first-order downward); and the faces that share any vertex
with the face (second-order). 


\subsection{Meshes\label{sub:Meshes}}

To be useful to applications, information
in the root set or one or more of its constituent entity sets is assumed
to be a valid computational mesh, examples of which include: 

\begin{itemize}
\item A non-overlapping, connected set of iMesh entities; for example,
the structured and unstructured meshes commonly used in finite element
simulations (\emph{simple mesh}).  
\item Overlapping grids in which a collection of simple meshes are used
to represent some portion of the computational domain, including
chimera, multiblock, and multigrid meshes (\emph{multiple mesh}). The
interfaces presented here handle these mesh types in a general way;
higher-level convenience functions may be added later to support
specific functionalities needed by these meshes. In this case, each of
the simple meshes is a valid computational mesh, stored as an entity
set.
\item Adaptive meshes in which all entities in a sequence of refined (simple or multiple)
meshes are retained in the root set. The most highly refined adaptation
level typically comprises a simple or multiple mesh. Typically,
different levels of mesh adaptation will be represented by different entity
sets, with many of the entities shared by multiple entity sets.
\item Smooth particle hydrodynamic (SPH) meshes, which consist of a collection
of iMesh vertices with no connectivity or adjacency information. 
\end{itemize}
At the most fundamental level, we consider a static simple mesh. This
mesh provides only basic query capabilities to return entities and their
adjacencies.  This implies that all implementations have a root set, but
not necessarily the subsetting capabilities described in
Section~\ref{sec:Data-Model}.\ref{sub:Entity-Sets}.

In addition, meshes can also be extended to be modifiable, through
support for creation and deletion of mesh entities (see
Section~\ref{sec:Interface}.\ref{sub:Mesh-Modification}).  Modifiable
meshes require a minimal interaction with the underlying geometric model
to uniquely associate mesh entities with geometric model entities of
equal or greater dimension\cite{geoclas}.


\subsection{Entity Sets\label{sub:Entity-Sets}}

The iMesh interface defines a mechanism for creating arbitrary
groupings of entities; these groupings are called \emph{entity sets}.
Each entity set may be a true set (in the set theoretic sense) or it may be
a (possibly non-unique) ordered list of entities; in the latter case,
entities are retrieved in the order in which they were added to the
entity set. An entity set also may or may not be a simple mesh; entity
sets that \emph{are} simple meshes have obvious application in
multiblock and multigrid contexts, for instance. Entity sets (other than
the root set) are populated by addition or removal of entities from the
set. In addition, set boolean operations --- subtraction, intersection,
and union --- are also supported.

Two primary relationships among entity sets are supported. First,
entity sets may contain one or more entity sets (by definition, all
entity sets belong to the root set). An entity set contained
in another may be either a subset or an element (each in the set theoretic
sense) of that entity set. The choice between these two interpretations
is left to the application; the iMesh interface does not impose either
interpretation. Set contents can be queried recursively or non-recursively;
in the former case, if entity set A is contained in entity set B,
a request for the contents of B will include the entities in A (and
the entities in sets contained in A). Second, parent/child relationships
between entity sets are used to represent logical relationships between
sets, including multigrid and adaptive mesh sequences. These logical
relationships naturally form a directed, acyclic graph.

Examples of entity sets include the ordered list of vertices bounding
a geometric face, the set of all mesh faces classified on that geometric
face, the set of regions assigned to a single processor by mesh partitioning,
and the set of all entities in a given level of a multigrid mesh sequence. 


\subsection{Tags\label{sub:Tags}}

Tags are used as containers for user-defined data that can be attached
to iMesh entities, meshes, and entity sets. Different values of a particular
tag can be associated with different mesh entities; for instance,
a boundary condition tag will have different values for an inflow
boundary than for a no-slip wall. In the general case, iMesh tags do
not have a predefined type and allow the user to attach arbitrary
data to mesh entities; this data is stored and retrieved by implementations
as a bit pattern. To improve performance and ease of use, we support
three specialized tag types: integers, doubles, and handles. These
typed tags enable correct saving and restoring of tag data when a
mesh is written to a file.



\section{Interface Description\label{sec:Interface}}

We have defined interfaces for a variety of commonly needed and
supported functionalities for mesh and entity query, mesh modification,
entity set operations, and tags.  In this section we describe the
functionality available through the TSTT mesh interface, including
semantic descriptions of the function calls in the
interface.\footnote{Note that these descriptions do not include detailed
syntax, which can be found in the interface user guide\cite{TSTTB-UG,TSTTM-UG}.}
For listings of allowable values of all TSTT enumerated data types,
see Appendix~\ref{app:TSTT-enum}.

\subsection{Global Queries\label{sub:Mesh-Interface}}

Global query functions can be categorized into two groups: 1)
\emph{database functions}, that manipulate the properties of the
database as a whole and 2) \emph{set query functions}, that query the
contents of entity sets as a whole; these functions require an entity
set argument, which may be the root set as a special case.  These
functions are summarized in Table~\ref{table:Mesh-Int}.

\begin{table}[tbp]
\caption{Functions for Global Queries}\label{table:Mesh-Int}
{\small
\begin{tabular}{|p{1.25in}|p{0.75in}|p{0.75in}|p{223pt}|}
\hline 
Function&
Input&
Output&
Description\tabularnewline
\hline
\hline 
load&
filename, ES&
---&
Loads mesh data from file into entity set\tabularnewline
\hline 
save&
filename, ES&
---&
Saves data from entity set to file\tabularnewline
\hline 
getRootSet&
---&
ES&
Returns handle for the root set\tabularnewline
\hline 
getGeometricDim&
---&
dimen&
Returns geometric dimension of mesh\tabularnewline
\hline 
getDfltStorage&
---&
SO&
Tells whether implementation prefers blocked or interleaved coordinate
data\tabularnewline
\hline 
getAdjTable&
---&
AI table&
Returns table indicating availability and cost of entity adjacency
data\tabularnewline
\hline 
areEHValid&
reset?&
handles changed?&
Returns true if EH remain unchanged since last user-requested status
reset\tabularnewline
\hline 
\hline 
getNumOfType&
ES, Type&
\# of Type&
Returns number of entities of type in ES\tabularnewline
\hline 
getNumOfTopo&
ES, Topo&
\# of Topo&
Returns number of entities of topo in ES\tabularnewline
\hline 
getAllVtxCoords&
ES, SO&
coords, SO&
Returns coords of all vertices in the set and all vertices on the
closure of higher-dimensional entities in the set; storage order can
be user-specified\tabularnewline
\hline 
getEntities&
ES, Type, Topo&
entities&
Returns all entities in ES of the given type and topology\tabularnewline
\hline 
getAdjEntities&
ES, Type, Topo, adj\_Type&
entities&
For all entities of given type and topology in ES, return adjacent entities
of adj\_type\tabularnewline
\hline 
getVtxArrCoords&
vertex handles, SO&
coords, SO&
For all input vertex handles, return coords; storage order can be
user-specified.\tabularnewline
\hline 
getVtxCoordIndex&
ES, Type, Topo, adj\_Type&
indices&
For all entities of given type and topology, find adjacent entities of
adj\_Type, and return the coordinate indices for their vertices. Vertex
ordering matches that in getAllVtxCoords.\tabularnewline
\hline
\end{tabular}
}
\end{table}

Database functions include functions to load and save information to a
file; file format is implementation dependent.  As mesh data is loaded,
entities are stored in the root set, and can optionally be placed into
a subsidiary entity set as well.  TSTT implementations must be able to provide
coordinate information in both blocked (xxx...yyy...zzz...) and
interleaved (xyzxyzxyz...) formats; an application can query the
implementation to determine the implementation's preferred storage
order.  Also, implementations must provide information about the
availability and relative cost of computing adjacencies between entities
of different types.  Finally, each instance of the interface must
provide a handle for the root set.

Set query functions allow an application to retrieve information about
entities in a set.  The entity set may be the root set, which will
return selected contents of the entire database, or may be any
subsidiary entity set.  For example, functions exist to request the
number of mesh entities of a given type or topology; the types and
topologies are defined as enumerations.  Applications can request
handles for all entities of a given type or topology or handles for
entities of a given type adjacent to all entities of a given type or
topology.  Also, vertex coordinates are available in either blocked or
interleaved order. Coordinate requests can be made for all vertices or
for the vertex handles returned by an adjacency call. Finally, indices
into the global vertex coordinate array can be obtained for both entity
and adjacent entity requests.

\subsection{Entity- and Array-Based Query\label{sub:Ent-Interface}}

\begin{table}[tbp]
\caption{Functions for Single Entity Queries}\label{table:Entity}
{\small
\begin{tabular}{|p{1.25in}|p{0.75in}|p{0.75in}|p{223pt}|}
\hline 
Function&
Input&
Output&
Description\tabularnewline
\hline
\hline 
initEntIter&
ES, Type, Topo&
anyData?, iter&
Create an iterator to traverse entities of type and topo in ES; return
true if any entities exist\tabularnewline
\hline 
getNextEntIter&
iter&
anyData?, EH&
Return true and a handle to next entity if there is one; false otherwise\tabularnewline
\hline 
resetEntIter&
iter&
---&
Reset iterator to restart traverse from the first entity\tabularnewline
\hline 
endEntIter&
iter&
---&
Destroy iterator\tabularnewline
\hline 
\hline 
getType&
EH&
Type&
Return type of entity\tabularnewline
\hline 
getTopo&
EH&
Topo&
Return topology of entity\tabularnewline
\hline 
getVtxCoord&
EH&
coords&
Return coordinates of a vertex\tabularnewline
\hline 
getEntAdj&
EH, Type&
entities&
Return entities of given type adjacent to EH\tabularnewline
\hline
\end{tabular}
}
\end{table}

\begin{table}[tbp]
\caption{Functions for Block Entity Queries}\label{table:EntArr}
{\small
\begin{tabular}{|p{1.25in}|p{0.75in}|p{0.75in}|p{223pt}|}
\hline 
Function&
Input&
Output&
Description\tabularnewline
\hline
\hline 
initEntArrIter&
ES, Type, Topo, size&
anyData?, iter&
Create a block iterator to traverse entities of type and topo in ES;
return true if any entities exist\tabularnewline
\hline 
getNextEntArrIter&
iter&
anyData?, EH array&
Return true and a block of handles if there are any; false otherwise\tabularnewline
\hline 
resetEntArrIter&
iter&
---&
Reset block iterator to restart traverse from the first entity\tabularnewline
\hline 
endEntArrIter&
iter&
---&
Destroy block iterator\tabularnewline
\hline 
\hline 
getEntArrType&
EH array&
Type array&
Return type of each entity\tabularnewline
\hline 
getEntArrTopo&
EH array&
Topo array&
Return topology of each entity\tabularnewline
\hline 
getEntAdj&
EH array, type&
entities&
Return entities of type adjacent to each EH\tabularnewline
\hline
\end{tabular}
}
\end{table}

The global queries described in the previous section are used to
retrieve information about all entities in an entity set. While this is
certainly a practical alternative for some types of problems and for
small problem size, larger problems or situations involving mesh
modification require access to single entities or to blocks of
entities. The TSTT mesh interface supports traversal and query functions
for single entities and for blocks of entities; the query functions
supported are entity type and topology, vertex coordinates, and entity
adjacencies.  Tables~\ref{table:Entity} and~\ref{table:EntArr} summarize
these functions.

\subsection{Mesh Modification\label{sub:Mesh-Modification}}


The TSTT mesh interface supports mesh modification by providing a
minimal set of operators for low-level modification; both single entity
(see Table~\ref{table:Modify}) and block versions (see
Table~\ref{table:ModArr}) of these operators are provided.  High-level
functionality, including mesh generation, quality assessment, and
validity checking, can in principle be built from these operators,
although in practice such functionality is more likely to be provided
using intermediate-level services that perform complete unit operations,
including vertex insertion and deletion with topology updates, edge and
face swapping, and smoothing.

\begin{table}[tbp]
\caption{Functions for Single Entity Mesh Modification}\label{table:Modify}
{\small
\begin{tabular}{|p{1.25in}|p{0.75in}|p{0.75in}|p{223pt}|}
\hline 
Function&
Input&
Output&
Description\tabularnewline
\hline
\hline 
createVtx&
coords&
VH&
Create vertex at given location\tabularnewline
\hline 
setVtxCoords&
VH, coords&
---&
Changes coordinates of existing vertex\tabularnewline
\hline 
createEnt&
Topo, handles&
EH, status&
Create entity of given topology from lower-dimensional entities; return
entity handle and creation status\tabularnewline
\hline 
deleteEnt&
EH&
---&
Delete EH from the mesh\tabularnewline
\hline
\end{tabular}
}
\end{table}
\begin{table}[tbp]
\caption{Functions for Block Mesh Modification}\label{table:ModArr}
{\small
\begin{tabular}{|p{1.25in}|p{0.75in}|p{0.75in}|p{223pt}|}
\hline 
Function&
Input&
Output&
Description\tabularnewline
\hline
\hline 
createVtxArr&
coords, SO&
VH array&
Create vertices at given location\tabularnewline
\hline 
setVtxArrCoords&
VH array, coords, SO&
---&
Changes coordinates of existing vertices\tabularnewline
\hline 
createEntArr&
topo, handles&
EH array, status array&
Create entities of given topology from lower-dimensional entities;
return entity handle and status\tabularnewline
\hline 
deleteEntArr&
EH array&
---&
Delete each EH from the mesh\tabularnewline
\hline
\end{tabular}
}
\end{table}


Geometry modification is achieved through functions that change vertex
locations.  Vertex locations are set at creation, and can be changed as
required, for instance, by mesh smoothing or other node movement
algorithms.

Topology modification is achieved through the creation and deletion of
mesh entities.  Creation of higher-dimensional entities requires
specification, in canonical order, of an appropriate collection of
lower-dimensional entities. For instance, a tetrahedron can be created
using four vertices, six edges or four faces, but not from
combinations of these. Upon creation, adjacency information properly
connecting the new entity to its components is set up by the
implementation. Some implementations may allow the creation of
duplicate entities (for example, two edges connecting the same two
vertices), while others will respond to such a creation request by
returning a copy of the already-existing entity.

Deletion of existing entities must always be done from highest to
lowest dimension, because the TSTT interface forbids the deletion
of an entity with existing upward adjacencies (for instance, an edge
that is still in use by one or more faces or regions).

\subsection{Entity Sets\label{sub:Entity-Set-Interface}}

The TSTT entity set interface is divided into three parts: basic set
functionality, hierarchical set relations, and set boolean operations.

\begin{table}[tbp]
\caption{Functions for Basic Entity Set Functionality}\label{table:EntSet}
{\small
\begin{tabular}{|p{1.25in}|p{0.75in}|p{0.75in}|p{223pt}|}
\hline 
Function&
Input&
Output&
Description\tabularnewline
\hline
\hline 
createEntSet&
isList&
SH&
Creates a new entity set (ordered and non-unique if isList is true)\tabularnewline
\hline 
destroyEntSet&
SH&
---&
Destroys existing entity set\tabularnewline
\hline 
isList&
SH&
ordered?&
Return true if the set is ordered and non-unique\tabularnewline
\hline 
\hline 
getNumEntSets&
SH, levels&
\# of sets&
Returns number of entity sets contained in SH\tabularnewline
\hline 
getEntSets&
SH, levels&
SH array&
Returns entity sets contained in SH\tabularnewline
\hline 
addEntSet&
SH1, SH2&
---&
Adds entity set SH1 as a member of SH2\tabularnewline
\hline 
rmvEntSet&
SH1, SH2&
---&
Removes entity sets SH1 as a member of SH2\tabularnewline
\hline 
isEntSetContained&
SH1, SH2&
contained?&
Returns true if SH2 is a member of SH1\tabularnewline
\hline 
\hline 
addEntToSet&
EH, SH&
---&
Add entity EH to set SH\tabularnewline
\hline 
rmvEntFromSet&
EH, SH&
---&
Remove entity EH from set SH\tabularnewline
\hline 
addEntArrToSet&
EH array, SH&
---&
Add array of entities to set SH\tabularnewline
\hline 
rmvEntArrFromSet&
EH array, SH&
---&
Remove array of entities from set SH\tabularnewline
\hline
isEntContained&
SH, EH&
contained?&
Returns true if EH is a member of SH\tabularnewline
\hline
\end{tabular}
}
\end{table}

Basic set functionality, summarized in Table~\ref{table:EntSet}, includes
creating and destroying entity sets; adding and removing entities and
sets; and several entity set specific query functions.  \footnote{Note
that the global mesh query functions (Section~\ref{sec:Interface}.\ref{sub:Mesh-Interface})
and traversal functions (Section~\ref{sec:Interface}.\ref{sub:Ent-Interface}) defined above
can be used with the root set or any other entity set as their first
argument.} Entity sets can be either ordered and non-unique, or
unordered and unique; an ordered set guarantees that query results
(including traversal) will always be given in the order in which
entities were added to the set. The ordered/unordered status of an
entity set must be specified when the set is created and can be queried.

Entity sets are created empty. Entities can be added to or removed from
the set individually or in blocks; for ordered sets, the last of a
number of duplicate entries will be the first to be deleted.  Also,
entity sets can be added to or removed from each other; note that,
because all sets are automatically contained in the root set from
creation, calls that would add or remove a set from the root set are not
permitted.  An entity set can also be queried to determine the number
and handles of sets that it contains, and to determine whether a given
entity or set belongs to that set.

\begin{table}[tbp]
\caption{Functions for Entity Set Relationships}\label{table:SetRel}
{\small
\begin{tabular}{|p{1.25in}|p{0.75in}|p{0.75in}|p{223pt}|}
\hline 
Function&
Input&
Output&
Description\tabularnewline
\hline
\hline 
addPrntChld&
SH1, SH2&
---&
Create a parent (SH1) to child (SH2) relationship\tabularnewline
\hline 
rmvPrntChld&
SH1, SH2&
---&
Remove a parent (SH1) to child (SH2) relationship\tabularnewline
\hline 
isChildOf&
SH1, SH2&
bool&
Return true if SH2 is a child of SH1\tabularnewline
\hline 
getNumChld&
SH, levels&
\# children&
Return number of children of SH\tabularnewline
\hline 
get Chldn&
SH, levels&
SH array&
Return children of SH\tabularnewline
\hline 
getNumPrnt&
SH, levels&
\# parents&
Return number of parents of SH\tabularnewline
\hline 
get Prnts&
SH, levels&
SH array&
Return parents of SH\tabularnewline
\hline
\end{tabular}
}
\end{table}
Hierarchical relationships between entity sets are intended to describe,
for example, multilevel meshes and mesh refinement hierarchies. The
directional relationships implied here are labeled as parent-child
relationships in the TSTT interface. Functions are provided to add,
remove, count, and identify parents and children and to determine if one
set is a child of another; see Table~\ref{table:SetRel}.

\begin{table}[tbp]
\caption{Functions for Entity Set Boolean Operations}\label{table:SetBool}
{\small
\begin{tabular}{|p{1.25in}|p{0.75in}|p{0.75in}|p{223pt}|}
\hline 
Function&
Input&
Output&
Description\tabularnewline
\hline
\hline 
subtract&
SH1, SH2&
SH&
Return set difference SH1-SH2 in SH\tabularnewline
\hline 
intersect&
SH1, SH2&
SH&
Return set intersection of SH1 and SH2 in SH\tabularnewline
\hline 
unite&
SH1, SH2&
SH&
Return set union of SH1 and SH2 in SH\tabularnewline
\hline
\end{tabular}
}
\end{table}
Set boolean operations --- intersection, union, and subtraction --- are
also defined by the TSTT interface; these functions are summarized in
Table~\ref{table:SetBool}. The definitions are intended to be compatible
with their C++ standard template library (STL) counterparts, both for
semantic clarity and so that STL algorithms can be used by
implementations where appropriate. All set boolean operations apply not
only to \emph{entity} members of the set, but also to \emph{set}
members. Note that set hierarchical relationships are not included: the
set resulting from a set boolean operation on sets with hierarchical
relationships will \emph{not} have any hierarchical relationships
defined for it, regardless of the input data. For instance, if one were
to take the intersection of two directionally-coarsened meshes (stored
as sets) with the same parent mesh (also a set) in a multigrid
hierarchy, there is no reason to expect that the resulting set will
necessarily be placed in the multigrid hierarchy at all. On the other
hand, if both of those directionally-coarsened meshes contain a set of
boundary faces, then their intersection will contain that set as well.

While set boolean operations are completely unambiguous for unordered
entity sets, ordered sets make things more complicated. For operations
in which one set is ordered and one unordered, the result set is unordered;
its contents are the same as if an unordered set were created with
the (unique) contents of the ordered set and the operation were
then performed. In the case of two ordered sets, the TSTT specification
follows the spirit of the STL definition, with complications related
to the possibility of multiple copies of a given entity handle in
each set. In the following discussion, assume that a given entity
handle appears $m$ times in the first set and $n$ times in the second
set. 

\begin{itemize}
\item For intersection of two ordered sets, the output set will contain
the $\min\left(m,n\right)$ copies of the entity handle. These will
appear in the same order as in the first input set, with the first
copies of the handle surviving. For example, intersection of the two
sets $A=\textrm{\{$abacdbca$\}}$ and $B=\{ dadbac\}$ will result
in $A\bigcap B=\{ abacd\}$.
\item Union of two ordered sets is easy: the output set is a concatenation
of the input sets: $A\bigcup B=\{ abacdbcadadbac\}$.
\item Subtraction of two ordered sets results in a set containing $\min\left(m-n,0\right)$
copies of an entity handle. These will appear in the same order as
in the first input set, with the first copies of the handle surviving.
For example, $A-B=\{ abc\}$.
\end{itemize}
Regardless of whether the entity members of an entity set are ordered or unordered, the
set members are always unordered and unique, with correspondingly
simple semantics for boolean operations.

\subsection{Tags\label{sub:Tag-Interface}}

\begin{table}[tbp]
\caption{Basic Tag Functions}\label{table:Tags}
{\small
\begin{tabular}{|p{1.25in}|p{0.75in}|p{0.75in}|p{223pt}|}
\hline 
Name&
Input&
Output&
Description\tabularnewline
\hline
\hline 
createTag&
name, \# values, TVT&
TH&
Creates a new tag of the given type and number of values\tabularnewline
\hline 
destroyTag&
TH, force&
---&
Destroys the tag if no entity is using it or if force is true\tabularnewline
\hline 
\hline 
getTagName&
TH&
name&
Returns tag ID string\tabularnewline
\hline 
getTagSizeValues&
TH&
size&
Returns tag size in number of values\tabularnewline
\hline 
getTagSizeBytes&
TH&
size&
Returns tag size in number of bytes\tabularnewline
\hline 
getTagHandle&
name&
TH&
Return tag with given ID string, if it exists\tabularnewline
\hline 
getTagType&
TH&
TVT&
Return data type of this tag\tabularnewline
\hline 
\hline 
getAllTags&
EH&
TH array&
Return handles of all tags associated with entity EH\tabularnewline
\hline 
getAllEntSetTags&
SH&
TH array&
Return handles of all tags associated with entity set SH\tabularnewline
\hline
\end{tabular}
}
\end{table}

Tags are used to associate application-dependent data with a mesh,
entity, or entity set.  Basic tag functionality defined in the TSTT
interface is summarized in Table~\ref{table:Tags}, while functionality
for setting, getting, and removing tag data is summarized in
Table~\ref{table:Tags2}.

When creating a tag, the application must provide its data type and
size, as well as a unique name. For generic tag data, the tag size
specifies how many bytes of data to store; for other cases, the size
tells how many values of that data type will be stored.  The
implementation is expected to manage the memory needed to store tag
data. The name string and data size can be retrieved based on the tag's
handle, and the tag handle can be found from its name. Also, all tags
associated with a particular entity can be retrieved; this can be
particularly useful in saving or copying a mesh.

\begin{table}[tbp]
\caption{Setting, Getting, and Removing Tag Data}\label{table:Tags2}
{\small
\begin{tabular}{|p{1.25in}|p{0.75in}|p{0.75in}|p{223pt}|}
\hline 
Function&
Input&
Output&
Description\tabularnewline
\hline
\hline 
setData&
EH, TH, tagVal&
---&
The value in tag TH for entity EH is set to the first tagValSize bytes
of the array<char> tagVal\tabularnewline
\hline 
setArrData&
EH array, TH, tagVal array&
---&
The value in tag TH for entities in EHarray{[}i{]} is set using data
in the array<char> tagValArray and the tag size\tabularnewline
\hline 
setEntSetData&
SH, TH, tagVal&
---&
The value in tag TH for entity set SH is set to the first tagValSize
bytes of the array<char> tagVal\tabularnewline
\hline 
set{[}Int,Dbl,EH{]}Data&
EH, TH, tagVal&
---&
The value in tag TH for entity EH is set to the int, double, or entity
handle in tagVal; array and entity set versions also exist.\tabularnewline
\hline 
\hline 
getData&
EH, TH&
tagVal&
Return the value of tag TH for entity EH\tabularnewline
\hline 
getArrData&
EH array, TH&
tagVal array&
Retrieve the value of tag TH for all entities in EH array, with data
returned as an array of tagVal's\tabularnewline
\hline 
getEntSetData&
SH, TH&
tagVal&
Return the value of tag TH for entity EH\tabularnewline
\hline 
get{[}Int,Dbl,EH{]}Data&
EH, TH&
tagVal&
Return the value of tag TH for entity EH; array and entity set versions
also exist.\tabularnewline
\hline
\hline 
rmvTag&
EH, TH&
---&
Remove tag TH from entity EH\tabularnewline
\hline 
rmvArrTag&
EH array, TH&
---&
Remove tag TH from all entities in EH array\tabularnewline
\hline 
rmvEntSetTag&
SH, TH&
---&
Remove tag TH from entity set SH\tabularnewline
\hline
\end{tabular}
}
\end{table}

Initially, a tag is not associated with any entity or entity set,
and no tag values exist; association is made explicitly by setting
data for a tag-entity pair. Tag data can be set for single entities,
arrays of entities (each with its own value), or for entity sets.
In each of these cases, separate functions exist for setting generic
tag data and type-specific data. Analogous data retrieval functions
exist for each of these cases. 

When an entity or set no longer needs to be associated with a tag
--- for instance, a vertex was tagged for smoothing and the smoothing
operation for that vertex is complete --- the tag can be removed from that entity
without affecting other entities associated with the tag. When a tag
is no longer needed at all --- for instance, when all vertices have
been smoothed --- the tag can be destroyed through one of two variant
mechanisms. First, an application can remove this tag from all
tagged entities, and then request destruction of the tag. Simpler
for the application is forced destruction, in which the tag is destroyed
even though the tag is still associated with mesh entities, and all
tag values and associations are deleted. Some implementations may
not support forced destruction.

\subsection{Error Handling\label{sub:Error-Handling}}

Like any API, the TSTT interface is vulnerable to errors, either through
incorrect input or through internal failure within an implementation.
For instance, it is an error for an application to request entities with
conflicting types and topologies. Also, an error in the implementation
occurs when memory for a new object cannot be allocated. The TSTT error
interface, summarized in Table~\ref{table:Error}, supports error
handling by defining standard behavior when an error occurs. Severity of
error actions range from ignoring errors through ``throwing'' errors to
aborting on errors.  Applications can set the default action.  Also, the
error interface defines a number of standard error conditions which
could occur in TSTT mesh functions, either because of illegal input or
internal implementation errors.

\begin{table}[tbp]
\caption{Error Handling Functionality}\label{table:Error}
{\small
\begin{tabular}{|p{1.25in}|p{0.75in}|p{0.75in}|p{223pt}|}
\hline 
Name&
Input&
Output&
Description\tabularnewline
\hline
\hline 
set&
ET, desc&
---&
Sets error type and description\tabularnewline
\hline 
getErrorType&
---&
ET&
Retrieves error type\tabularnewline
\hline 
getDescription&
---&
desc string&
Retrieves error description\tabularnewline
\hline 
echo&
label&
---&
Prints label and description string to stderr\tabularnewline
\hline
\end{tabular}
}
\end{table}



\section{Programming Using the TSTT Mesh Interface\label{sec:Programming}}

Early experience with the TSTT interface shows that there are significant
performance penalties created by certain programming practices, including
some practices that naturally arise when translating an existing library-specific
code to use TSTT instead. This section highlights recommended best
practices that come from our early experiences with the interface; more
work in this area is ongoing.

\begin{description}
\item [Use~direct~array~access.]Many TSTT calls return data in SIDL
arrays. Because TSTT uses only one-dimensional SIDL arrays, access to
data in these arrays is simpler and more efficient when using direct
array access rather than calls to the SIDL \texttt{get()} and
\texttt{set()} functions. To use direct array access, the address of the
actual array within the SIDL array structure must be exposed so that it
can be treated directly as a simple array by the application code.
\item [Retrieve~data~in~chunks.]Whenever possible, retrieve data using
the TSTT array access calls instead of entity by entity calls. Experiments
indicate that chunks containing as few as 20 entities were sufficient
to reduce interoperability overhead to less than 5\%, especially when
combined with direct array access~\cite{mcinnes05}.
\item [Re-use~SIDL~arrays.]Especially in inner loops, the time spent
creating and destroying SIDL arrays can represent a significant overhead.
In many cases, this can easily be prevented by moving array creation
and destruction outside the inner loop. In cases where an array is
created within a function call in the original code, the SIDL array
can be made static or created outside the function call and passed
in.
\end{description}




\section{Current Status and Ongoing Work\label{sec:Conclusions}}

The TSTT mesh interface has been implemented into several existing
meshing tools, including FMDB (RPI), GRUMMP (UBC), MOAB (SNL), Frontier
(SUNY SB), Overture (LLNL), and NWGrid (PNNL).  The interface is used by
several TSTT-developed mesh services tools including a mesh quality
improvement toolkit (Mesquite)\cite{Mesquite03}, a face- and
edge-swapping service\cite{TSTT-swap-tool}, and a mesh adaptation service.  In
addition, the TSTT interface is used in several
application codes.  The most notable of these is the joint work between
the TSTT consortium and researchers at the Stanford Linear Accelerator
Center (SLAC).  In this work, researchers are developing the TSTT-based
mesh services needed in design optimization of accelerator cavities and
to insert a mesh adaptation loop into SLAC's linear accelerator design
code\cite{GeLe04}.

To increase the dissemination of TSTT-compliant tools, we are now
working to establish component-level compliance with the standards of
the Common Component Architecture (CCA) Forum\cite{cca-forum}.  The CCA
Forum is defining a component architecture tailored to address all
aspects of high-performance scientific computing.  As part of this
effort, they are creating the Rapid Application Development environment,
in which TSTT mesh implementations will play a key role.

Work continues to improve the functionality and ease of use of the TSTT
interface.  We are working to adapt our current unit test suite for full
TSTTM implementations for use by those requiring correct behavior for
only a subset of functionality to be able to use a particular service.
Also, we expect to leverage current Babel research aimed at improving
software component compliance and usage through interface-level software
contracts\cite{dahlgren:sehpcs04,dahlgren:study04,dahlgren:sehpcs05}.

More information on the TSTT interface, including complete documentation,
can be found at http://www.tstt-scidac.org.



\section*{Acknowledgements}

This work was performed under the auspices of the U.S. Department of
Energy by the University of California Lawrence Livermore National
Laboratory under contract No. W-7405-Eng-48 (UCRL-JRNL-213577); by The
University of British Columbia under Canadian Natural Sciences and
Engineering Research Council (NSERC) Special Research Opportunities
Grant SRO-299160; and by Rensselaer Polytechnic Institute under DOE
grant number DE-FC02-01ER25460.

\bibliographystyle{plain}
\bibliography{../biblio/tstt}

\appendix
\section{Enumerations Defined in the TSTT Mesh
Interface}\label{app:TSTT-enum}
The TSTT interface uses enumerated types for variables that have a
specific, restricted range of values.  These enumerations, and their
possible values, are given in Table~\ref{table:Enums}.  These
enumerations are largely self-explanatory, with the exception of
AdjacencyInfo.  The values of AdjacencyInfo reflect that an
implementation may be able to supply a particular piece of adjacency
information never, always, or only sometimes (for example, an
implementation might choose to store boundary faces but not interior
faces for memory reasons, making it impossible to return the
latter). Also, if adjacency information is available, the cost of
retrieving may be constant time (example: stored data); logarithmic time
(example: tree search); or linear time (example: searching the entire
list of entities).
\begin{table}[htp]
\caption{TSTT Enumerated Types}\label{table:Enums}
{\small
\begin{tabular}{|l|p{5in}|}
\hline 
Enum Name&
Values\tabularnewline
\hline
\hline 
ErrorAction&
SILENT, WARN\_ONLY, THROW\_ERROR\tabularnewline
\hline 
ErrorType&
\raggedright SUCCESS, DATA\_ALREADY\_LOADED, NO\_DATA, FILE\_NOT\_FOUND,
FILE\_ACCESS\_ERROR, NIL\_ARRAY, BAD\_ARRAY\_SIZE, BAD\_ARRAY\_DIMENSION,
INVALID\_ENTITY\_HANDLE, INVALID\_ENTITY\_COUNT, INVALID\_ENTITY\_TYPE,
INVALID\_ENTITY\_TOPOLOGY, BAD\_TYPE\_AND\_TOPO, ENTITY\_CREATION\_ERROR,
INVALID\_TAG\_HANDLE, TAG\_NOT\_FOUND, TAG\_ALREADY\_EXISTS, TAG\_IN\_USE,
INVALID\_ENTITYSET\_HANDLE, INVALID\_ITERATOR\_HANDLE,
INVALID\_ARGUMENT, ARGUMENT\_OUT\_OF\_RANGE, MEMORY\_ALLOCATION\_FAILED, 
NOT\_SUPPORTED, FAILURE \tabularnewline
\hline 
TagValueType&
\raggedright INTEGER, DOUBLE, ENTITY\_HANDLE, BYTES\tabularnewline
\hline
EntityType&
\raggedright VERTEX, EDGE, FACE, REGION, ALL\_TYPES\tabularnewline
\hline 
EntityTopology&
\raggedright POINT, LINE\_SEGMENT, POLYGON, TRIANGLE, QUADRILATERAL, POLYHEDRON,
TETRAHEDRON, HEXAHEDRON, PRISM, PYRAMID, SEPTAHEDRON, ALL\_TOPOLOGIES\tabularnewline
\hline 
StorageOrder&
\raggedright BLOCKED, INTERLEAVED, UNDETERMINED\tabularnewline
\hline 
AdjacencyInfo&
\raggedright UNAVAILABLE, ALL\_ORDER\_1, ALL\_ORDER\_LOGN, ALL\_ORDER\_N, SOME\_ORDER\_1,
SOME\_ORDER\_LOGN, SOME\_ORDER\_N\tabularnewline
\hline 
CreationStatus&
\raggedright NEW, ALREADY\_EXISTED, CREATED\_DUPLICATE, CREATION\_FAILED\tabularnewline
\hline
\end{tabular}
}
\end{table}

\section{Language-Specific Translations of a Typical SIDL Function
Definition from the TSTT Mesh Interface}

One of the advantages of using the Scientific Interface Description
Language (SIDL) is that it eliminates language compatibility issues,
allowing easy use of TSTT servers written in one language with clients
written in another. This appendix gives a simple example of a function
definition in SIDL, and its instantiation in specific programming
languages. See the Babel documentation for complete information on
conversion of SIDL files and use of the Babel-generated interfaces
in client and server code, as well as information about Babel's support
for Java and Python.

We begin with a snippet from the actual TSTT SIDL file defining the
\texttt{getAdjacentEntities} function; \texttt{{}``...''} indicates
omitted definitions. The TSTTM package contains all mesh-specific
interface definitions; tag and entity set functionality are defined
in a separate package (TSTTB), because these functions are also useful
for geometry objects, for instance. TSTTM functions are further divided
into five interfaces: \texttt{Mesh} for global query, \texttt{Entity}
and \texttt{EntArr} for entity- and block-based query, and \texttt{Modify}
and \texttt{ModArr} for single and block modification calls. The \texttt{opaque}
type identifier is used in SIDL to represent opaque handles for objects,
and \texttt{array<type>} represents an array of that type. As we shall
see, Babel converts these meta-types into actual types depending on
the target language. Finally, note that SIDL supports an exception
mechanism through the keyword \texttt{throws}; all TSTT functions
can throw errors, with the precise mechanism being language specific. 

\begin{lyxcode}
package~TSTTM~version~0.7

\{

~~...

~~interface~Mesh~\{

~~~~void~getAdjEntities(~in~opaque~entity\_set,

~~~~~~in~EntityType~entity\_type\_requestor,~

~~~~~~in~EntityTopology~entity\_topology\_requestor,

~~~~~~in~EntityType~entity\_type\_requested,

~~~~~~inout~array<opaque>~adj\_entity\_handles,

~~~~~~out~int~adj\_entity\_handles\_size,

~~~~~~inout~array<int>~offset,

~~~~~~out~int~offset\_size,

~~~~~~inout~array<int>~in\_entity\_set,

~~~~~~out~int~in\_entity\_set\_size)~throws~TSTTB.Error;

~~\};

~~...

\}
\end{lyxcode}
An implementation of the TSTT mesh interface (that is, a mesh database
server) is also declared in a SIDL file. The following example declares
an implementation that guarantees to support the TSTTM global query
and entity interfaces, as well as tags on individual entities.

\begin{lyxcode}
package~MyMeshDB~version~0.7~\{

~~class~MyMesh~implements-all~TSTTM.Mesh,~TSTTM.Entity,~TSTTB.Tag,

~~~~TSTTB.EntTag;

\}
\end{lyxcode}

\subsection*{C++ Instantiation}

The C++ instantiation of SIDL functions is most similar syntactically
to the original SIDL file. A class is defined with the same name as
the interface (\texttt{Mesh}, in this case), with all classes derived
from a base class defined in Babel's runtime library. Primitive types
and enumerations are unchanged in the C++ code. \texttt{opaque}'s
are mapped to \texttt{void{*}}'s, and a \texttt{sidl::array<>} template
class handles array data. Note that any \texttt{sidl::NullIORException}'s
are caught by the glue code between the client and the server.

\begin{lyxcode}
namespace~TSTTM~\{

~~class~Mesh~:~public~::sidl::StubBase~\{~

~~~~void~getAdjEntities~(

~~~~~~/{*}in{*}/~void{*}~entity\_set,

~~~~~~/{*}in{*}/~::TSTTM::EntityType~entity\_type\_requestor,

~~~~~~/{*}in{*}/~::TSTTM::EntityTopology~entity\_topology\_requestor,

~~~~~~/{*}in{*}/~::TSTTM::EntityType~entity\_type\_requested,

~~~~~~/{*}inout{*}/~::sidl::array<void{*}>\&~adj\_entity\_handles,

~~~~~~/{*}out{*}/~int32\_t\&~adj\_entity\_handles\_size,

~~~~~~/{*}inout{*}/~::sidl::array<int32\_t>\&~offset,

~~~~~~/{*}out{*}/~int32\_t\&~offset\_size,

~~~~~~/{*}inout{*}/~::sidl::array<int32\_t>\&~in\_entity\_set,

~~~~~~/{*}out{*}/~int32\_t\&~in\_entity\_set\_size

~~~~)

~~~~throw~(~

~~~~~~::sidl::NullIORException,~::TSTTB::Error

~~~~);


\end{lyxcode}

\subsection*{C Instantiation}

In the C interface for a SIDL function, package and interface names
are prepended to the function name to disambiguate names. The {}``\texttt{self}''
argument is a handle for the mesh database information, and exceptions
are passed as an additional, final argument. SIDL's array type is
instantiated in C as a structure.

\begin{lyxcode}
void~TSTTM\_Mesh\_getAdjEntities(

~~TSTTM\_Mesh~self,

~~void{*}~entity\_set,

~~enum~TSTTM\_EntityType\_\_enum~entity\_type\_requestor,

~~enum~TSTTM\_EntityTopology\_\_enum~entity\_topology\_requestor,

~~enum~TSTTM\_EntityType\_\_enum~entity\_type\_requested,

~~struct~sidl\_opaque\_\_array{*}{*}~adj\_entity\_handles,

~~int32\_t{*}~adj\_entity\_handles\_size,

~~struct~sidl\_int\_\_array{*}{*}~offset,

~~int32\_t{*}~offset\_size,

~~struct~sidl\_int\_\_array{*}{*}~in\_entity\_set,

~~int32\_t{*}~in\_entity\_set\_size,

~~sidl\_BaseInterface{*}~\_ex);
\end{lyxcode}

\subsection*{Fortran 77 Instantiation}

As is the case with C, the Fortran 77 interface for a SIDL function
prepends the package and interface name to the function name; in addition,
{}``\texttt{\_f}'' is appended to the end of the function name.
Because F77 does not support pointers or structures, SIDL opaques,
arrays, exceptions, interfaces, and classes are stored as \texttt{integer{*}8}.
Enumerations are passed as integers, and strings as Fortran arrays
of characters (\texttt{CHARACTER{*}({*})}). 

\begin{lyxcode}
~~~~~~subroutine~TSTTM\_Mesh\_getAdjEntities\_f(self,~entity\_set,~~

~~~~~1~~~~entity\_type~requestor,~entity\_topology\_requestor,~

~~~~~2~~~~entity\_type\_requested,~adj\_entity\_handles,

~~~~~3~~~~adj\_entity\_handles\_size,~offset,~offset\_size,~

~~~~~4~~~~in\_entity\_set,~in\_entity\_set\_size,~exception)~

~~~~~~integer{*}8~self,~entity\_set

~~~~~~integer~entity\_type\_requestor,~entity\_topology\_requestor

~~~~~~integer~entity\_type\_requested

~~~~~~integer{*}8~adj\_entity\_handles,~offset,~in\_entity\_set

~~~~~~integer{*}4~adj\_entity\_handles\_size,~offset\_size,~in\_entity\_set\_size

~~~~~~integer{*}8~exception
\end{lyxcode}

\subsection*{Fortran 90 Instantiation}

Because Fortran 90 has support for derived types and modules, F90
interfaces for SIDL-defined functions are somewhat simpler. Functions
in the TSTTM \texttt{Mesh} interface are declared in F90 in a module
called \texttt{TSTTM\_Mesh}; function names have {}``\texttt{\_s}''
appended. Opaque data in the SIDL interface is passed using long integers,
just as in F77. Class data, including interfaces, arrays, and exceptions,
are passed using defined types.

\begin{lyxcode}
subroutine~getAdjEntities\_s(self,~entity\_set,~~~~~~~~~~~~~~~~~~~~~~~~~~~~~~~~321\&

~~entity\_type\_requestor,~entity\_topology\_requestor,~entity\_type\_requested,~~~\&

~~adj\_entity\_handles,~adj\_entity\_handles\_size,~offset,~offset\_size,~~~~~~~~~~\&

~~in\_entity\_set,~in\_entity\_set\_size,~exception)

~~type(TSTTM\_Mesh\_t)~,~intent(in)~::~self

~~integer~(selected\_int\_kind(18))~,~intent(in)~::~entity\_set

~~integer~(selected\_int\_kind(9))~,~intent(in)~::~entity\_type\_requestor

~~integer~(selected\_int\_kind(9))~,~intent(in)~::~entity\_topology\_requestor

~~integer~(selected\_int\_kind(9))~,~intent(in)~::~entity\_type\_requested

~~type(sidl\_opaque\_1d)~,~intent(inout)~::~adj\_entity\_handles

~~integer~(selected\_int\_kind(9))~,~intent(out)~::~adj\_entity\_handles\_size

~~type(sidl\_int\_1d)~,~intent(inout)~::~offset

~~integer~(selected\_int\_kind(9))~,~intent(out)~::~offset\_size

~~type(sidl\_int\_1d)~,~intent(inout)~::~in\_entity\_set

~~integer~(selected\_int\_kind(9))~,~intent(out)~::~in\_entity\_set\_size

~~type(sidl\_BaseInterface\_t)~,~intent(out)~::~exception
\end{lyxcode}


\end{document}
