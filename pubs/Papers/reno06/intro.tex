\section*{List of Abbreviations}
\begin{multicols}{3}
\begin{lyxlist}{Topo}
\item[AI]{Adjacency information enum}
\item[EH]{Entity handle}
\item[ES]{Entity set handle}
\item[ET]{Error type enum}
\item[iter]{Iterator over entities}
\item[SH]{Set handle}
\item[SO]{Storage order enum}
\item[TH]{Tag handle}
\item[TVT]{Tag value type enum}
\item[Topo]{Entity topology enum}
\item[Type]{Entity type enum}
\item[VH]{Vertex handle}
\end{lyxlist}
\end{multicols}

\section{Introduction\label{sec:Introduction}}

Creating simulation software for problems described by partial
differential equations is a relatively common but very time-consuming
task. Much of the effort of developing a new simulation code goes into
writing infrastructure for tasks such as interacting with mesh and
geometry data, equation discretization, adaptive refinement, design
optimization, etc. Because these infrastructure components are common to
most or all simulations, re-usable software for these tasks would
significantly reduce both the time and expertise required to create a
new simulation code.

Currently, libraries are the most common mechanism for software re-use
in scientific computing, especially the highly-successful libraries for
numerical linear algebra\cite{petsc,BaGr97,eispack,lapack,linpack}.
The drawback to software re-use through libraries is the difficulty in
changing from one to another. When a user wishes to add functionality or
simply experiment with a different implementation of the same
functionality in another library, all calls within an application must
be changed to the other API, which likely will not package functionality
in precisely the same way. Another significant challenge with library
use, especially in the context of meshing and geometry libraries, is
that data structures used within the libraries may be radically
different, making changes from one library to another even more onerous.
This time-consuming conversion process can be a significant diversion
from the central scientific investigation, so many application
researchers are reluctant to undertake it. This can lead to the use of
sub-optimal strategies.  For example, new advances developed by the
meshing research community often take years to become incorporated into
application simulations.

To address these issues, the Terascale Simulation Tools and Technologies
(TSTT) center is working to develop interoperable software tools for
meshes, domain geometry, and discretization\cite{tstt:overview}.  The
present paper will discuss our work in developing a mesh interface.  The
most prominent example of prior research in defining interfaces for
meshing is the Unstructured Grid Consortium, a working group of the AIAA
Meshing, Visualization, and Computing Environments Technical
Committee\cite{UGC-web}.  The first release of the UGC
interface\cite{UGC-v1} was aimed at high level mesh operations,
including mesh generation and quality assessment.  Recognizing a need
for lower-level functionality, the UGC has developed a low-level query
and modification interface for mesh databases, as well as an interface
for defining generic high-level services\cite{UGC-v2:paper}.

The TSTT mesh interface has a broader scope than the UGC interface.  In
addition to supporting low-level mesh manipulation, the TSTT interface
is also designed to support the requirements of solver applications,
including the ability to define mesh subsets and to attach arbitrary
user data to mesh entities.  In addition, The TSTT interface is intended
to be both language and data structure independent.  In summary, our
initial target is to support low-level interaction between applications
programs --- both meshing and solution applications --- and external
mesh databases regardless of the data structures and programming
language used by each.  In the long term, we expect to also support
high-level operations, including mesh generation, typically as services
built using the TSTT interface.

The fundamental challenge in developing this interface has been the
tension between generality and compactness: our goal has been to define
a set of operations addressing all common uses of mesh data while
minimizing redundancy and avoiding idioms peculiar to a particular
underlying mesh representation.

We began by defining a general abstract data model, focusing on the ways
in which mesh data is used in simulations rather than on how mesh data
is stored by meshing tools.  The data model, described in detail in
Section~\ref{sec:Data-Model}, includes fundamental mesh entities ---
vertices, faces, elements, etc --- and the topological relationships
between them, as well as the concepts of general mesh subsets and
arbitrary data associated with mesh entities.

The mesh interface is built on this data model.  The interface
(Section~\ref{sec:Interface}) supports global and local mesh query, mesh
modification, and collections and tagging of mesh entities.  
% This needs to be said somewhere, but where exactly?  Is this the right
% place? 
The TSTT mesh interface is built on a client-server model, with the
explicit assumption that the client (application) and server (mesh
database) may be written in different programming languages.  To address
cross-language issues, especially with arrays and strings, the TSTT
interface is defined using the Scientific Interface Description Language
(SIDL)\cite{babel:site05,babel:usersguide05}.  This language neutral
description is then processed by an existing interpreter, Babel, to
produce a language-specific client API and server skeleton, as well as
glue code that mediates language translation issues.

Performance data from early usage of the interface suggests there is a
preferred coding style for using the TSTT mesh interface (see
Section~\ref{sec:Programming}).  The interface is already in use in
various meshing tools and simulation applications and on-going
development continues to improve the usability and accessibility of
TSTT-compliant software (see Section ~\ref{sec:Conclusions}).

