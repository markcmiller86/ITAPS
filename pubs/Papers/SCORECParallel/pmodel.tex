% This is section{ partition model}. 

A partition model is a separate data model created in FMDB to represent
a mesh partition. It can be viewed as a part of hierarchical domain
decomposition, and is developed between the mesh and the geometric
model. Its purpose is to represent mesh partition in topology and
support mesh-level parallel operations through part boundary links
with ease [2]. 

\textbf{Definition: Partition Model Entity} \footnote{Is the
  dimension necessary for the partition model entity? It is
  implemented as 0 in FMDB, but it can be kept
  for later use, and is different from part denotation $P_i$. }  

A topological entity in the partition model, $P^d_i$, which represents a set of part boundary entities that have the
  same $\mathcal{P}$. Each partition model entity is uniquely
  determined by $\mathcal{P}$.   

\textbf{Definition: Partition Classification}

The unique association of part boundary entity, $M^{d_i}_i$, to the topological
entity of the partition model entity of dimension $d_j$, $P^{d_j}_j$ where
$d_i\leq d_j$, on which it lies is termed partition classification and
is denoted $M^{d_i}_i \subset P^{d_j}_j$.


Each partition model entity stores its id, residence
part(s), and all part boundry mesh entities that are
classified on it.\footnote{I am not sure if partition model entity
  needs owner part.}

\textbf{Definition: Reverse Partition Classification}

For each partition model entity, $P^d_i$, the set of part boundary entities
that are classified on that entity defines the
reverse partition classification for the partition model entity. The
reverse partition classification is denoted as $RC(P^{d_i}_i)=\{M^{d_j}_j|M^{d_j}_j
\sqsubset P^{d_i}_i,{d_j} \leq {d_i}\}$.


\textbf{Definition: Partitioned Object}

A partitioned object is a list of parts on a process. Whenever the
application program needs to access a part, it should access the
partitioned object first.  

A partitioned mesh on one process is a container of sub-meshes of all
parts on the process. It is local to the process. Each part is treated
as a serial sub-mesh, adding the information of part
boundary entities. A process can have multiple parts, i.e. multiple
sub-meshes at one time. The application
program accesses a part mesh through the partitioned mesh on the proper process.   

A common partition model exists on each process, which is shared by
all parts on the process\footnote{Should the partition model only
  store the partition model entity local to that process or all
  partition model entities?}. Each process has a local partitioned mesh.      
