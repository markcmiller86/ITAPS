
% \section{Introduction}

Requirements for modeling and simulation capabilities are increasing
in breadth of physics, fidelity of 
models, complexity of the systems to be simulated, problem size
(e.g. degrees of freedom) \cite{TACC}. These 
problems of interest are multi-physics and multi-scale in nature,
including electronic structures, seismic 
modeling, fusion, system biology, and climate modeling \cite{LLNL}.
The availability of supercomputers 
capable of sustained petaflops performance within the next several
years will create numerous opportunities 
to advance all these scientific applications by computer modeling of
previously intractable problems. The 
availability also challenges the current computer technologies and
algorithms, such as massively parallel 
computing and automatically adaptive simulations. 

In order to support scientific applications in SCOREC software framework at the petascale level, the system is 
required to satisfy the following items. 
\begin{itemize}

\item Balanced work loads on all processors. Even small imbalances
  result in many wasted processors. For 
example, 50,000 processors with one processor 5\% over average workload
				 is equivalent to 2380 idle
				processors 
and 47,620 perfectly balanced processors\cite{devine}.

\item Low interprocessor communication costs, including message
  passing volume and data redistribution costs. 
Processor speeds are increasing faster then network speeds.   

\item Scalable performance on computing algorithms, including scalable
  program execution time and memory 
usage on each processor. 
\end{itemize}

SCOREC has developed an entire software framework, including FMDB and ScorecMeshAdapt,  to 
support massively parallel adaptive, multiphysics and mutliscale
application codes, especially finite element 
method based applications. In the software framework, FMDB is
responsible for combining complex application-
independent capabilities into a single software infrastructure that is
shared by the application codes. 
ScorecMeshAdapt is responsible for parallel adaptivity for distributed,
unstructured meshes for the application codes. 

An overview of SCOREC software framework's capabilities and design abstractions of its
distributed dynamic object-oriented mesh 
data structure and parallel algorithms is presented in the following sections. These
capabilities include management of 
distributed mesh and field data structures, parallel mesh
input/output, interface to data 
partitioning and load balancing, and
transfer of data and fields between 
distributed meshes, and predictive load balancing for parallel
adaptive computations. 
