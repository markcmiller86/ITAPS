\documentclass[10pt]{article}
\usepackage{amsfonts}
\usepackage{amsmath}

%opening
\title{Parallel Interface Design for Petascale Parallel Adaptive Simulations}

\author{Ting Xie, Onkar Sahni, Xiaojuan Luo, Min Zhou,   \\ 
        Andrew C. Bauer, Kenneth E. Jansen, Mark Shephard}

\date{\today}

\begin{document}
\maketitle 

\section{Abstract}

Our main goal is to provide the parallel cababilities of the adaptive
control of meshes of billions of elements on petascale
computers.

This paper focuses on entity group and multiple parts per
process, and presents the whole parallel interface design in
SCOREC software framework (FMDB and MeshAdapt), including management of distributed mesh
and field data structures, parallel data
input/output, interface to data partitioning and
load balancing, transfer of data and fields between 
distributed meshes, and predictive load balancing for parallel
adaptive computations. 
All these interfaces are based on distributed memory (MIMD) computing
and message passing (MPI) communication.  


\section{Introduction}     \section*{List of Abbreviations}
\begin{multicols}{3}
\begin{lyxlist}{Topo}
\item[AI]{Adjacency information enum}
\item[EH]{Entity handle}
\item[ES]{Entity set handle}
\item[ET]{Error type enum}
\item[iter]{Iterator over entities}
\item[SH]{Set handle}
\item[SO]{Storage order enum}
\item[TH]{Tag handle}
\item[TVT]{Tag value type enum}
\item[Topo]{Entity topology enum}
\item[Type]{Entity type enum}
\item[VH]{Vertex handle}
\end{lyxlist}
\end{multicols}

\section{Introduction\label{sec:Introduction}}

Creating simulation software for problems described by partial
differential equations is a relatively common but very time-consuming
task. Much of the effort of developing a new simulation code goes into
writing infrastructure for tasks such as interacting with mesh and
geometry data, equation discretization, adaptive refinement, design
optimization, etc. Because these infrastructure components are common to
most or all simulations, re-usable software for these tasks would
significantly reduce both the time and expertise required to create a
new simulation code.

Currently, libraries are the most common mechanism for software re-use
in scientific computing, especially the highly-successful libraries for
numerical linear algebra\cite{petsc,BaGr97,eispack,lapack,linpack}.
The drawback to software re-use through libraries is the difficulty in
changing from one to another. When a user wishes to add functionality or
simply experiment with a different implementation of the same
functionality in another library, all calls within an application must
be changed to the other API, which likely will not package functionality
in precisely the same way. Another significant challenge with library
use, especially in the context of meshing and geometry libraries, is
that data structures used within the libraries may be radically
different, making changes from one library to another even more onerous.
This time-consuming conversion process can be a significant diversion
from the central scientific investigation, so many application
researchers are reluctant to undertake it. This can lead to the use of
sub-optimal strategies.  For example, new advances developed by the
meshing research community often take years to become incorporated into
application simulations.

To address these issues, the Interoperable Tools for Advanced Petascale
Simulation (ITAPS) center is working to develop interoperable software
tools for meshes, domain geometry, and
discretization\cite{tstt:overview}.  The present paper will discuss our
work in developing a mesh interface.  The most prominent example of
prior research in defining interfaces for meshing is the Unstructured
Grid Consortium, a working group of the AIAA Meshing, Visualization, and
Computing Environments Technical Committee\cite{UGC-web}.  The first
release of the UGC interface\cite{UGC-v1} was aimed at high level mesh
operations, including mesh generation and quality assessment.
Recognizing a need for lower-level functionality, the UGC has developed
a low-level query and modification interface for mesh databases, as well
as an interface for defining generic high-level
services\cite{UGC-v2:paper}.

The ITAPS mesh interface, called iMesh, has a broader scope than the UGC
interface.  In addition to supporting low-level mesh manipulation, the
iMesh interface is also designed to support the requirements of solver
applications, including the ability to define mesh subsets and to attach
arbitrary user data to mesh entities.  In addition, the iMesh interface
is intended to be both language and data structure independent.  In
summary, our initial target is to support low-level interaction between
applications programs --- both meshing and solution applications --- and
external mesh databases regardless of the data structures and
programming language used by each.  In the long term, we expect to also
support high-level operations, including mesh generation, typically as
services built using the iMesh interface.

The fundamental challenge in developing this interface has been the
tension between generality and compactness: our goal has been to define
a set of operations addressing all common uses of mesh data while
minimizing redundancy and avoiding idioms peculiar to a particular
underlying mesh representation.  A common theme in many design decisions
while developing the iMesh interface has been to support common
constructs as simply, directly, and efficiently as possible while still
allowing more sophisticated, less common constructs to be expressed in
the data model and interface.

We began by defining a general abstract data model, focusing on the ways
in which mesh data is used in simulations rather than on how mesh data
is stored by meshing tools.  The data model, described in detail in
Section~\ref{sec:Data-Model}, includes fundamental mesh entities ---
vertices, faces, elements, etc --- and the topological relationships
between them, as well as the concepts of general mesh subsets and
arbitrary data associated with mesh entities.

The mesh interface is built on this data model.  The interface
(Section~\ref{sec:Interface}) supports global and local mesh query, mesh
modification, and collections and tagging of mesh entities.  
% This needs to be said somewhere, but where exactly?  Is this the right
% place? 
The iMesh interface is built on a client-server model, with the
explicit assumption that the client (application) and server (mesh
database) may be written in different programming languages.  To address
cross-language issues, especially with arrays and strings, the iMesh
interface is defined using the Scientific Interface Description Language
(SIDL)\cite{babel:site05,babel:usersguide05}.  This language neutral
description is then processed by an existing interpreter, Babel, to
produce a language-specific client API and server skeleton, as well as
glue code that mediates language translation issues.

Performance data from early usage of the interface suggests there is a
preferred coding style for using the iMesh interface (see
Section~\ref{sec:Programming}).  The interface is already in use in
various meshing tools and simulation applications and on-going
development continues to improve the usability and accessibility of
iMesh-compliant software (see Section ~\ref{sec:Conclusions}).




\section{Partitions}         \subsection{Definitions and Properties}   


\textbf{Definition: Process and Process Rank} \footnote{Process or
  processor? ITAPS\cite{ITAPS} uses term 'processor' instead of 'process'. If
  we assume one process per processor(one core), we can use term
  'processor'.} 

In computing, a process is an instance of a computer program that is
being executed. A program is a collection of instructions, and a
process is the actual execution of those instructions. Multiple
processes may be associated with the same program, but execute
independently. A process has its own variables, stack, and local
memory allocation. At any time, multiple processes may exist on one
physical processor, but cannot be split across multiple processors. 

For an application program run with m processes, the processes are
numbered 0 through m-1. And the assigning number is the rank. The
process with process rank i in the execution is denoted
$P_{r\_i}$.\footnote{To be different from part denotation of $P_i$.}   


\textbf{Definition: Entity}

Entity is the basic unit in the problem domain that the application
program works
with. It has a dimension, and entity of higher dimension can be
bounded by other entities of lower dimension.  
For example, consider a finite element
mesh, an entity is a mesh entity (vertex, edge, face,
or region). A mesh region of tetrahedron
type can be bounded by 4 mesh faces, 6 mesh edges, and 4 mesh
vertices. 

\textbf{Definition: Part Entity}

Part entity is an entity that does not bound any other entities of
higher dimension. The definition of part mesh entity is presented in
section5. 

\textbf{Definition: Entity Group}

Entity group is a group of part entities that are required to stay
together during the process of data partition and migration. The
definitions and operations of mesh entity group are defined in
Section5. 


\textbf{Definition: Part Object} 

A part object is the basic unit of data to which a destination part
can be assigned during the the process of data partition and
migration. It can be a part entity or an entity group. 
 For example, consider a finite element
 mesh, a part object can be a part mesh entity, or a mesh entity
 group. 


\textbf{Definition: Partition}

A partition is a distribution of data into sub-sets called parts
according to some distribution rule, such that one part object is
assigned to a single part. Within a partition, part objects are
assigned to
different parts, and parts are assigned to different processes. The
number of parts is independent from the number of processes. All these
assignments can be changed dynamically. For example, in finite element
space, a mesh partition can be a distribution of a mesh into
sub-meshes.  


\textbf{Definition: Part} 

A part is a collection of subset data. A part is assigned to one and
only one process at any time, i.e, a part can not span across
processes \cite{ITAPS}. A process can contain multiple parts. For example,
consider a finite element mesh, a part is a collection of mesh
entities.  

\textbf{Definition: Part Id}  

For each part, a globally unique part id (short for global part
identifier) can be given over a set of processes within a 
partition. The part with part id i can be denoted $P_i$. \footnote{To
  be 
  consistent with the situation of one part per process, in which the
  value of part id equals to that of the process rank for each part,
  part id will be implemented as a globally unique integer within the
  partition.}  

The application program can access a part though its part id. The
process rank of a process that a part resides can be decided through
the part id. Except the situation of one part per process, the value
of part id is always different from the value of process rank for each
part.  

\textbf{Definition: Part Boundary} 

 A part boundary is a set of data shared by multiple parts, which
 reside in one process or multiple processes. The data on a part
 boundary is in the closure of the part, not belonging to the interior
 of the part. A part can have multiple part boundaries shared by
 different parts. It is also called inter-part boundary. 


\textbf{Definition: Part Boundary Entity}\footnote{Once a part
  boundary
  entity is defined, the information of partition classification and
  residence parts is stored only for a small part of mesh entities.}

Part boundary entity is short for an entity that resides in a part
boundary. Maintaining the information of part boundary entity is very
important to perforam parallel partition and migration within the
partition. For example, in a finite element mesh, the $i^{th}$ part
boundary mesh entity of dimension d is denoted as $M^d_{B\_i}$.


\textbf{Definition: Residence Part} 

For a part object, the residence part is the part where it
resides. For a part boundary entity, the residence part(s) is the
union of residence part(s) where the part object(s) that the part
boundary entity bounds resides. 


\subsection{Functional Requirement Definitions for Partitions}

For a part boundary entity, it must be aware where it is
duplicated. 

\textbf{Definition: Remote Copy} 

Given a part boundary entity $V^d_{B\_i}$ on one part $P_i$, the
memory location of
the part boundary entity duplicated on parts other than part $P_i$. 

The data of a part boundary entity duplicated on
different parts must be consistent at any time. Once a part boundary
entity is
modified on one part, its remote copies must be modified
accordingly.

% remote copy = entity pointer+ part id   

For a part boundary entity, it is beneficial to assign a specific copy
as the owner of the others and let the owner be in charge of
communication or computation between the copies \cite{Seol}. 

\textbf{Definition: Owner Part}

For a part boundary entity, one of the parts sharing the part boundary
entity
will be assigned as its owner part according to some rule dynamically,
such as poor-to-rich ownership, which assigns the poorest part to be
the owner part, where the poorest part is the part that has the least
number of part objects among residence parts of the entity
\cite{Seol}.\footnote{The poor-to-rich assignment may have some problem, since
the poorest part may be not on the the poorest process that has the
least work load.}

\textbf{Definition: Residence Part Operator $\mathcal{P}$} 

An operator that returns a set of part id(s) where part object or part
boundary entity exists. 

For a part object $PO_i$ \footnote{Please refer symbol list in Section4.}, $\mathcal{P}[PO_i]={p}$ where p is the
id of a part where $PO_i$ exists. The part object $PO_i$ can be a part
entity $V^d_{P\_j}$, or an entity group $V^d_{G\_j}$.

For a part boundary entity $V^d_{B\_i}$, 

\begin{equation}
\mathcal{P}[V^{d_i}_{B\_i}]=\{\mathcal{P}[V^{d_j}_{P\_j}|V^d_{B\_i}\in\{\theta(V^{d_j}_{P\_j})]\}\cup\{\mathcal{P}[V^{d_j}_{G\_j}|V^d_{B\_i}\in\{\theta(V^{d_j}_{G\_j})]\}.  
\end{equation}



\section{List of Symbols}  % This is section of {List of Symbols}

\begin{itemize}

\item{V}  The model, $V\in\{G,P,M\}$ where G signifies the
  geometric model, P signifies the partition model, and M signifies
  the mesh model. 

\item{$\{V\{V^d\}\}$} a set of topological entities of dimension d in
  model V. 

\item{$V^d_i$} the $i^{th}$ entity of dimension d in model V. Shorthand
  for $V\{V^d\}_i$, d=0 for a vertex, d=1 for an edge, d=2 for a face,
  and d=3 for a region. In topologry, edges, faces, and regions are
  bounded by the lower order entities. 

\item{$\theta(V^d_i)$} a set of entities on the boundary of $V^d_i$. 

\item{$\{V^d_i\{V^q\}\}$} the set of entities of dimension q in model
  V that are adjacent to $V^d_i$.

\item{$V^d_i\{V^q\}_i$} the $i^{th}$ entity in the set of entities of
  dimension q in model V that are adjacent to $V^d_i$. 

\item{$V^d_{B\_i}$} the $i^{th}$ part boundary entity of dimension d in
  model V within model P.                                               % part boundary entity
 
\item{$V^d_{P\_i}$} the $i^{th}$ part entity of dimension d in model
  V within model P. It does not bound any entity of higher dimension.   % part entity

\item{$V^d_{G\_i}$} the $i^{th}$ entity group of dimension d in model
  V, in which d is the highest dimension of the dimensions of entities
  in the group.              % entity group 
 
% part object
\item{$PO_i$} the $i^{th}$ part object in model V within model
  P. It can be a part entity {$V^d_{P\_j}$}, or an entity group
  {$V^d_{G\_j}$}. \footnote{Do we really need the implementation of
  part object?} 

\item{$\theta(V^d_{G\_i})$} a set of entities on the boundary of
  $V^d_{G\_i}$.        % boundary of an entity group

\item{$P_{r\_i}$} the $i^{th}$ process within model P. 

\item{$P_i$} the $i^{th}$ part within model P. 

\item{$U^{d_i}_i\sqsubset V^{d_j}_j$} classification indicating the
  unique association of entity $U^{d_i}_i$ with entity $V^{d_j}_j$,
  $d_i\leq d_j$, where $U,V\in\{G,P,M\}$ and U is lower than V in
  terms of a hierarchy of domain decomposition. 

\item{$\mathcal{P}[PO_i]$} a set of part id(s) where part object
  $PO_i$ exists. 

\item{$\mathcal{P}[V^d_{P\_i}]$} a set of part id(s) where part
  entity $V^d_{P\_i}$ exists. 

\item{$\mathcal{P}[V^d_{G\_i}]$} a set of part id(s) where entity
  group $V^d_{G\_i}$ exists. 

\item{$\mathcal{P}[V^d_{B\_i}]$} a set of part id(s) where part
  boundary entity $V^d_{B\_i}$ exists. 


\end{itemize}



\section{Mesh Entity Group} \footnote{Only one operation is added in
  this updated document: getBdryEntity.}         


% This is section for entitygroup. 

The definition of mesh entity group is derived from the defintion of
part mesh entity, which is from part entity. 


\textbf{Definition: Part Mesh Entity}

In finite element space, part mesh entity is a mesh entity that does not
bound any higher order mesh entities.In a 3D manifold
mesh, this includes all mesh regions. In a 3D non-manifold mesh, this
includes all mesh regions, the mesh faces that do not bound any mesh
regions; the mesh edges that do not bound any mesh faces, and mesh
vertices that do not bound any mesh edges. The $i^{th}$ part mesh entity of
dimension d is denoted as $M^d_{P\_i}$.


\subsection{Definition of Entity Group}

\textbf{Entity group:} a group of part mesh entities that
need to stay together. It is defined by the application program
according to rules listed below. 

\begin{itemize}
\item Entity group only includes part mesh entities that need to stay
  on the same part. These part mesh entities stay together as a group
  during the life time of entity group.

\item A part mesh entity only associates with a single entity group, and is defined once in the entity group.

\item Entity group information is maintained consistently before and
  after migration on the part it exists.

\item Entity group is dynamic in nature during the execution, i.e.,
  the application program can create and destroy an entity group, or
  add/remove  part mesh entities in an entity group. 
\end{itemize}

\subsection{operations on entity group} 
Once an entity group is defined, the application program has a
handle/pointer to access the entity group and can perform basic
operations on the entity group. The possible operations on entity group are listed below. 

\textbf{Mesh level operation for entity group}

\begin{itemize}
\item Iteration over entity groups in a mesh  

      Loop through all entity groups in a mesh, for example, to get a
      certain part mesh entity. 
\end{itemize}

\textbf{Group level operation for entity group}

\begin{itemize}

\item Create/Delete

To create an entity group, the application program needs to give FMDB
a list of part mesh entities. The application program can delete an
entity group if it is not needed.

\item getBdryEntity 

Get a set of mesh entities that bound the part mesh entities on the boundary of the entity
group. 

\item Clear

Clear the list of part mesh entities in an entity group to get an empty entity group. 

\item Size

Get the number of part mesh entities in an entity group.

\item isEmpty

Check whether or not an entity group is empty, i.e., without any part mesh entities.

\item Union/Subtract

Union: merge the second entity group into the first entity group, and delete the second entity group.

Subtract: retrieve some part mesh entities from an entity group to
build a new entity group, and remove/erase these part mesh entities in the old entity group.

\item Iteration over part mesh entities in an entity group

Loop through part mesh entities of an entity group, for example, to get a
certain part mesh entity.

\item Add/Remove

Append or remove a part mesh entity in an entity group.

\end{itemize}

\textbf{Entity level operation for entity group}
\begin{itemize}
\item GetEntityGroup

Given a part mesh entity, find the entity group it belongs to. If it does not belong to any entity group, the operation returns zero. 
\end{itemize}

In FMDB implementation, this functionality is achieved by attaching
group information (local memory location of the entity group on the
process) to a part mesh entity when the entity is added to entity
group, or to all part mesh entities in the group when an entity group
is created with part mesh entities. When a part mesh entity is removed
from the entity group, or an entity group is removed from the mesh,
the attached group information for the part mesh entity in the group is deleted. 



 

\section{Partition Model}   % This is section{ partition model}. 

A partition model is a separate data model created in FMDB to represent
a mesh partition. It can be viewed as a part of hierarchical domain
decomposition, and is developed between the mesh and the geometric
model. Its purpose is to represent mesh partition in topology and
support mesh-level parallel operations through part boundary links
with ease [2]. 

\textbf{Definition: Partition Model Entity} \footnote{Is the
  dimension necessary for the partition model entity? It is
  implemented as 0 in FMDB, but it can be kept
  for later use, and is different from part denotation $P_i$. }  

A topological entity in the partition model, $P^d_i$, which represents a set of part boundary entities that have the
  same $\mathcal{P}$. Each partition model entity is uniquely
  determined by $\mathcal{P}$.   

\textbf{Definition: Partition Classification}

The unique association of part boundary entity, $M^{d_i}_i$, to the topological
entity of the partition model entity of dimension $d_j$, $P^{d_j}_j$ where
$d_i\leq d_j$, on which it lies is termed partition classification and
is denoted $M^{d_i}_i \subset P^{d_j}_j$.


Each partition model entity stores its id, residence
part(s), and all part boundry mesh entities that are
classified on it.\footnote{I am not sure if partition model entity
  needs owner part.}

\textbf{Definition: Reverse Partition Classification}

For each partition model entity, $P^d_i$, the set of part boundary entities
that are classified on that entity defines the
reverse partition classification for the partition model entity. The
reverse partition classification is denoted as $RC(P^{d_i}_i)=\{M^{d_j}_j|M^{d_j}_j
\sqsubset P^{d_i}_i,{d_j} \leq {d_i}\}$.


\textbf{Definition: Partitioned Object}

A partitioned object is a list of parts on a process. Whenever the
application program needs to access a part, it should access the
partitioned object first.  

A partitioned mesh on one process is a container of sub-meshes of all
parts on the process. It is local to the process. Each part is treated
as a serial sub-mesh, adding the information of part
boundary entities. A process can have multiple parts, i.e. multiple
sub-meshes at one time. The application
program accesses a part mesh through the partitioned mesh on the proper process.   

A common partition model exists on each process, which is shared by
all parts on the process\footnote{Should the partition model only
  store the partition model entity local to that process or all
  partition model entities?}. Each process has a local partitioned mesh.      




\section{Using Zoltan to Perform Partition}    % This is section{Using Zoltan to Do Partition}

FMDB uses Zoltan load-balancing interface functions to perform
partition and repartition of the part objects across the processes. In
order To use Zoltan in FMDB, three steps are needed. 

1. Assign a global identifier to each part object.
2. Set options to control the behavior of Zoltan, the application
program can choose any option to do multilevel
partitioning  
  geometric, graph or hypergraph based partitioner;
  coarse or refine partition; 
  tolerant imbalancings.  
 
3. Provide necessary query functions to provide part object
information to Zoltan (geometric, graph, and hypergraph query functions). 


\subsection{Define global identifier for part object}   
% part object(po) pointer + process rank 

Each part object in the data is required to have global identifier (ID) in
Zoltan partition. And each part object can be represented as local
memory allocation plus the process rank. This global
ID is easy to build and rebuild, and would save the time to search the
corresponding part object based on a global ID. 

% From FMDB's aspect, to desribe how to provide options and query
% functions to Zoltan. 

\subsection{Set Zoltan's Parameters} % I am not sure that I should use
				% graph node or part object. Even
				% though they have the same meaning. 

A specific structure of partition options is provided by FMDB to the
application program, which can set Zoltan's partition parameters
through the partition options. The structure of the partition options
is listed as follows. Currently, FMDB only uses Zoltan graph-based
load balancing interface functions.  

\begin{itemize}

\item{lb\_method:} The partition algorithm used by Zoltan. The default
  value is 'GRAPH'.

\item{lb\_approach:} The desired load balancing approach. 'PARTITION',
  'REPARTITION' or 'REFINE' \cite{Zoltan}. The default value is 'PARTITION'.  

\item obj\_weight\_dim: The number of weights associated with a graph
  node. The default value is zero, indicates all graph nodes have
  equal weight.  

\item edge\_weight\_dim: The number of weights associated with a graph
  edge. The default value is zero, indicates all graph edges have
  equal weight. 

\item num\_global\_partitions:   The total number of partitions to be
  generated in the partition result. Integer values greater than zero
  are accepted \cite{Zoltan}. The default value is the number of processes. 

\item num\_local\_partitions: 
  The number of partitions to be generated on this processor in the
  partition result. Integer values greater than zero
  are accepted. If any process is set with the option, the value on
  processes not setting this option is assumed to be zero \cite{Zoltan}. The
  default value is one for each process.   

\item imblance\_tol: The amount of load imbalance the partitioning
  algorithm should deem acceptable. The default value is 1.1, which
  ndicates that 10\% imbalance is OK; that is, the maximum over the
  average shouldn't exceed 1.1  \cite{Zoltan}.    

\item debug\_level: An integer indicating how much debugging
  information is printed by Zoltan. Higher values produce more output
  and potentially slow down Zoltan's computation \cite{Zoltan}. 

\item timer: The timer with which the application program wish to measure time. Valid
  choices are wall, cpu, and user  \cite{Zoltan}. 


\end{itemize}


\subsection{Provide Query Functions to Zoltan}

In order to use Zoltan graph-based load-balacing interface functions,
FMDB defines a class  'pmZoltanCallbacks' to interact with Zoltan, and provides four query
functions to Zoltan in that class as follows. Please see Appendix for
the details of the class  'pmZoltanCallbacks'. 

\begin{itemize}

\item \textbf{get\_num\_po} a query function returns the number of
  graph nodes that are currently assigned to the processor.

  
\item \textbf{get\_po\_list} a query function returns the information about
  the graph nodes  currently assigned to the processor, including an
  array of global ID(s) for all graph nodes,  and optionally
  their weights. 

\item \textbf{get\_num\_edges\_multi} a query function returns the number of
  graph edges for connected to each graph node returned by query
  function get\_po\_list. 

\item \textbf{get\_edge\_list\_multi} a query function returns arrays of global
  ID(s), process ID(s), and optionally edge weights for graph nodes
  connected to each graph node returned by query
  function get\_po\_list. 

\end{itemize}


\subsection{Construct Graph Nodes and Graph Edges}   %\input{graph.tex}


   


\section{Dynamic Load Balance and Data Migration}  %\input{lb_migrate.tex}

Static load balance and dynamic load balance, 
Data migration algorithm on multiple parts per process, considering
entity group. 
Attached data migration. 

\section{Parallel File I/O}       %
% This section is for parallel file i/o. 



The number of partitioned data files is different from the number of
processes. Multiple partitioned files can be loaded on one process. 

\section{Predictive Load Balance} %predictive load balancing

\subsection{Motivation and Goal}
The motivation for predictive load balancing is to improve the
efficiency of parallel mesh adaptation procedures. The goal is to
balance both the amount of calculation to be carried out during mesh
adaptation and the amount of physical memory used on each processor
for a given input mesh and mesh size field.
\subsection{Introduction}
During the course of parallel mesh modification, without predictive load
balancing, some processors may have nearly all their elements
scheduled for mesh modification which may exceed the limit of the physical memory of
these processors and would either slow down the process or even kill
process. Moreover, if the amount of calculation to be carried out
during mesh adaptation is not balanced, the processors which carry out
more calculations would keep all other ones waiting. By predictive load
balancing procedures, about equal number of elements in the adapted mesh
on each processor may be achieved while
moving smaller number of mesh entities between processors as compared to migration
after mesh adaptation and the process of mesh adaptation could be
scaled better by balancing the amount of calculation on each
processor. To achieve the goal of predictive load balancing, weights
have to be specified to graph nodes and a weighted graph-based
repartitioning is carried out. To balance both the amount of
calculation during mesh adaptation operations and the amount of
physical memory used on each processor multiple weight need to be
specified to graph nodes. One weight is used to estimate the amount of
calculation related to mesh modification operations to be carried out and the
other one is used to estimate the physical memory usage which is
proportional to the number of elements to be
generated. The weighted graph-based repartitioning is carried out before
each stage of mesh modification.

The number of estimated elements can be found as follow: Pain et al. \cite{Pain} uses the ratio of
element volume of the input mesh to the volume of desired element in
the adapted mesh to estimate the number of elements to be
generated. Flaherty et al. \cite{Flaherty} uses error indicator information and
refinement threshold to choose the edges to be marked for splitting
and determine the proper 3-D refinement templates. The number of
elements to be generated is determined by the refinement
templates. For estimating the amount of calculation related to mesh
modification operations, we propose to determine a representative
number of floating point operation for each type of mesh modification operation.
 
\subsection{Review of procedures of SCOREC mesh adapt}
In the course of mesh adaptation, mesh modification procedures
repeatedly modify the mesh to satisfy the mesh metric field. In
particular, it consists of three stages: mesh coarsening, mesh
refinement and shape correction. Generally speaking mesh coarsening
eliminates all the mesh edges shorter than the lower bound of a given
interval in the transformed space, if possible. Mesh refinement splits all the mesh edges longer than the upper bound of the interval.The third stage re-aligns local mesh configurations through the determination and elimination of sliver tetrahedra using local mesh modifications\cite{Li}. The details of mesh adaptation referred to \cite{Li}.

\subsection{Predictive load balancing for each stage}
\subsubsection{Coarsening}
For the stage of mesh coarsening, predictive load balancing is
designed to balance both edge collapse operation and the number of
elements after coarsening. Multiple-weight (two) is specified to graph
nodes.

{\bf (1) Balance of edge collapse operation:}

The input of mesh coarsening stage is a list of short edges. Predictive
load balancing tries to make each processor has about equal number of
short edges (marked for edge collapse). But weights need to be
specified to graph nodes (mesh elements). A relation between short
edges and mesh elements need to be built up. 
The weights of mesh elements are initialized to be zero. Loop over the
short edges in the list. Let \begin{math}M_i^1 \end{math} denote the
current short edge, and \begin{math}L_c \end{math} denote a
representative count of floating point operation of edge collapse. Then \begin{math} L_c/m
\end{math} is added to the weight of tetrahedra in set \begin{math}
  M_i^1\{M^3\} \end{math}, where m is the number of tetrahedra in set \begin{math}
  M_i^1\{M^3\} \end{math}. (Note the element which is not bounded by any
  edges marked for collapsing has weight zero)

{\bf (2) Estimate the number of elements after coarsening:}

The ratio of element volume of the initial mesh to the volume of
desired element in the adapted mesh could be used as weight to balance
the number of elements after coarsening. Loop over the list of short edges, compute the
element volume in the transformed space for each tetrahedron in the set of \begin{math}
M_i^1\{M^3\} \end{math} (where \begin{math}M_i^1 \end{math} is the
current short edge) by \begin{equation}
  V'=min(\sqrt{|T|}dV),i=0,1,2,3\end{equation} where \begin{math} T
  \end{math} is the mesh metric tensor and \begin{math}dV\end{math} is
    the volume of infinitesimal tetrahedron in the physical
    space\cite{Li}. The desired element volume equals one in ideal case. The
    ratios (smaller than one due to coarsening) are specified to the tetrahedra in the
    set respectively if no weight has been specified yet. After looping over the
    short edge list, one is specified as weight to the graph nodes
    which has been assigned no weight(no coarsening operation carried
    out). The values specified to the input elements could be normalized by the smallest one.
\subsubsection{Refinement}
For the stage of mesh refinement, predictive load balancing is
designed to balance the mesh refinement operation and the number of
elements after refinement for each iteration. Again, multiple-weight
(two) is specified to
graph nodes.

{\bf (1) Balance of mesh refinement operation:}

The list of edges marked for split (long edge) is the input of mesh
refinement. Initialize the graph nodes weight to be zero. Loop over the
long edge list, let \begin{math}M_i^1 \end{math} denote the current long edge
and \begin{math}L_{es} \end{math} denote a representative count of floating point operation
of edge split, then \begin{math}L_{es}/m\end{math} is added
  to the weight of tetrahedra in the set of
  \begin{math}M_i^1\{M^3\}\end{math}, where m is the number of
  tetrahedra of the set. Then, loop over the list of faces to be split, let
  \begin{math}M_i^2\end{math} denote the current face, and
    \begin{math}L_{fs}\end{math}denote a representative count of floating point operation
    of face split (based on the subdivision templates determined by
    refinement edges that bound the face), then \begin{math}L_{fs}/m\end{math} is added
  to the weight of tetrahedra in the set of
  \begin{math}M_i^2\{M^3\}\end{math}, where m is the number of
  tetrahedra of the set. Finally, loop over the list of mesh regions to be split,
  let \begin{math}M_i^3\end{math} be the current region, and
  \begin{math}L_{rs}\end{math} denote a representative count of floating point operation of
  region split (based on the subdivision templates), then
  \begin{math}L_{rs}\end{math} is added to the weight of
  \begin{math}M_i^3\end{math}. (Note: the element which is not bounded
    by any edges marked for split has weight zero) 

{\bf (2) Estimate the number of elements after refinement:}

Similar idea to \cite{Flaherty} could be applied to estimate the number of
elements to be generated. The mesh refinement of SCOREC mesh adapt splits long edges in
transformed space using the full set of edge-based refinement
templates. As long as the list of edges to be split is given, the
subdivision templates to each element is determined as well as the
number of elements to be generated. The number of elements to be
generated is specified as weight to each element. One is
specified as weight if no new element to be generated (no edge marked for
split). 

\subsubsection{Shape Correction}

For the stage of shape correction, predictive load balancing is
designed to balance the shape correction operation and the number of
element after correction. Multiple-weight (two) is specified to graph nodes
as well.

{\bf (1) Balance of shape correction operation:}

All the elements with quality below a given threshold are insert into
a dynamics list. Initialize the element weight to be zero. Loop over
the element in the low quality list, specify the floating point
operations of shape correction (analysis of silver tetrahedra and local
modification, particularly edge swap operation) as the weight of the
element.

{\bf (2) Estimate the number of elements after shape correction:}

Shape correction will not dramatically change the number of elements,
default value one could be specified as the weight to each element.

\subsection{Discussion}
To balance the floating point operation and number of elements to be
generated for each stage is expensive. For the coarsening and shape
correction stage, only the initial list is considered (the dynamics
list is updated during operation). For the mesh refinement stage, only
mesh refinement is considered, other mesh modification in this stage
(like edge collapse to new created short edges and projection of newly
created nodes on curved boundaries) are not considered, which may
affect the load balance as well. The count of floating point operation
needs to be determined by testing, which is not aware now. The
proposed predictive load balancing methods are just ideas without
testing, which may or may not work. 

%\end{document}
 

% memory discussion. 


\section{Conclusion}


\begin{thebibliography}{9}
\bibitem{LLNL} Steven F.Ashby. \textit{Scalable Algorithms and Novel
  Paradigms for Petascale Simulation}. Feb, 2006.
\bibitem{TACC} H. Carter Edwards. \textit{TACC Seminar:
SIERRA Framework Supporting Parallel Adaptive Multiphysics
Applications}. May, 2004. 
\bibitem{devine} Karen Devine. \textit{Partitioning and Dynamic Load
  Balancing for Petascale Applications}. SciDAC, 2007.
\bibitem{ITAPS} ITAPS. \textit{ITAPS Parallel Interface: Requirements
  Document}. Jun 12, 2007. 
\bibitem{Seol} Seegyoung Seol. \textit{FMDB:Flexible Distributed Mesh
  Database for Parallel Automated Adaptive Analysis}. PhD dissertation
  in RPI, 2005.
\bibitem{Zoltan} \textit{Zoltan: Parallel Partitioning, Load Balancing and Data-Management
Services}. http://www.cs.sandia.gov/Zoltan/.
\bibitem{R} Loy R. \textit{Autopack User Mannual}. Science Division
  Argonne National Laboratory, 2000. 
\bibitem{Pain} C.C.Pain, A.P.Umpleby, C.R.E.de Oliveira and
  etc. \textit{Tetrahedron mesh optimization and adaptivity for steady-state and transient finite
element calculations}. Computer methods in applied mechanics and
engineering. 2001.
\bibitem{Li} X.Li, \textit{Mesh Modification procedures for general 3-D non-manifold domains}. PhD thesis, Rensselaer Polytechnic Institute, August 2003.
\bibitem{Flaherty} J.E.Flaherty, R.M.Loy, P.C.Scully, M.S.Shephard and
  etc. \textit{Load Balancing and communication Optimization for
  Parallel Adaptive Finite Element Methods}, SCCC'97.

\end{thebibliography}




\end{document}


