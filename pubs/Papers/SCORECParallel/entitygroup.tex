% This is section for entitygroup. 

The definition of mesh entity group is derived from the defintion of
part mesh entity, which is from part entity. 


\textbf{Definition: Part Mesh Entity}

In finite element space, part mesh entity is a mesh entity that does not
bound any higher order mesh entities.In a 3D manifold
mesh, this includes all mesh regions. In a 3D non-manifold mesh, this
includes all mesh regions, the mesh faces that do not bound any mesh
regions; the mesh edges that do not bound any mesh faces, and mesh
vertices that do not bound any mesh edges. The $i^{th}$ part mesh entity of
dimension d is denoted as $M^d_{P\_i}$.


\subsection{Definition of Entity Group}

\textbf{Entity group:} a group of part mesh entities that
need to stay together. It is defined by the application program
according to rules listed below. 

\begin{itemize}
\item Entity group only includes part mesh entities that need to stay
  on the same part. These part mesh entities stay together as a group
  during the life time of entity group.

\item A part mesh entity only associates with a single entity group, and is defined once in the entity group.

\item Entity group information is maintained consistently before and
  after migration on the part it exists.

\item Entity group is dynamic in nature during the execution, i.e.,
  the application program can create and destroy an entity group, or
  add/remove  part mesh entities in an entity group. 
\end{itemize}

\subsection{operations on entity group} 
Once an entity group is defined, the application program has a
handle/pointer to access the entity group and can perform basic
operations on the entity group. The possible operations on entity group are listed below. 

\textbf{Mesh level operation for entity group}

\begin{itemize}
\item Iteration over entity groups in a mesh  

      Loop through all entity groups in a mesh, for example, to get a
      certain part mesh entity. 
\end{itemize}

\textbf{Group level operation for entity group}

\begin{itemize}

\item Create/Delete

To create an entity group, the application program needs to give FMDB
a list of part mesh entities. The application program can delete an
entity group if it is not needed.

\item getBdryEntity 

Get a set of mesh entities that bound the part mesh entities on the boundary of the entity
group. 

\item Clear

Clear the list of part mesh entities in an entity group to get an empty entity group. 

\item Size

Get the number of part mesh entities in an entity group.

\item isEmpty

Check whether or not an entity group is empty, i.e., without any part mesh entities.

\item Union/Subtract

Union: merge the second entity group into the first entity group, and delete the second entity group.

Subtract: retrieve some part mesh entities from an entity group to
build a new entity group, and remove/erase these part mesh entities in the old entity group.

\item Iteration over part mesh entities in an entity group

Loop through part mesh entities of an entity group, for example, to get a
certain part mesh entity.

\item Add/Remove

Append or remove a part mesh entity in an entity group.

\end{itemize}

\textbf{Entity level operation for entity group}
\begin{itemize}
\item GetEntityGroup

Given a part mesh entity, find the entity group it belongs to. If it does not belong to any entity group, the operation returns zero. 
\end{itemize}

In FMDB implementation, this functionality is achieved by attaching
group information (local memory location of the entity group on the
process) to a part mesh entity when the entity is added to entity
group, or to all part mesh entities in the group when an entity group
is created with part mesh entities. When a part mesh entity is removed
from the entity group, or an entity group is removed from the mesh,
the attached group information for the part mesh entity in the group is deleted. 


